% O preambulo deve conter o tipo do trabalho, o objetivo, 
% o nome da instituição e a área de concentração 
\preambulo{Dissertação apresentada ao Programa de Pós-Graduação em Informática
Aplicada (PPGIA) da Universidade de Fortaleza como parte dos requisitos
necessários para a obtenção do grau de Mestre em Informática Aplicada.}
% ---

% ---
% Configurações de aparência do PDF final

% alterando o aspecto da cor azul
\definecolor{blue}{RGB}{41,5,195}

% informações do PDF
\makeatletter
\hypersetup{
      %pagebackref=true,
    pdftitle={\@title}, 
    pdfauthor={\@author},
      pdfsubject={\imprimirpreambulo},
      pdfcreator={LaTeX with abnTeX2},
    pdfkeywords={abnt}{latex}{abntex}{abntex2}{trabalho acadêmico}, 
    %colorlinks=false,          % false: boxed links; true: colored links
      linkcolor=blue,           % color of internal links
      %citecolor=blue,           % color of links to bibliography
      filecolor=magenta,          % color of file links
    urlcolor=blue,
    bookmarksdepth=4
}
\makeatother
% --- 

% --- 
% Espaçamentos entre linhas e parágrafos 
% --- 

% O tamanho do parágrafo é dado por:
\setlength{\parindent}{1.3cm}

% Controle do espaçamento entre um parágrafo e outro:
\setlength{\parskip}{0.2cm}  % tente também \onelineskip

%Altera a capa
% Impressão da Capa
\renewcommand{\imprimircapa}{%
\begin{capa}%
\center

% Logomarca da Universidade
\includegraphics[scale=2]{./img/logo.png}\\
\textbf{FUNDAÇÃO EDSON QUEIROZ}\\
\textbf{UNIVERSIDADE DE FORTALEZA -- UNIFOR}

\vfill
\ABNTEXchapterfont\bfseries\LARGE\imprimirtitulo
\vfill

\ABNTEXchapterfont\large\imprimirautor
\vspace*{1cm}
\vfill\vfill\vfill
\large\imprimirlocal

\large\imprimirdata

\vspace*{1cm}
\end{capa}
}
% ---
% Altera a folha de rosto
\renewcommand{\folhaderostocontent}{
  \begin{center}

    {\ABNTEXchapterfont\large\imprimirautor}

    \vspace*{\fill}\vspace*{\fill}
    \begin{center}
      \ABNTEXchapterfont\bfseries\Large\imprimirtitulo
    \end{center}
    \vspace*{\fill}

    \hspace{.45\textwidth}
    \begin{minipage}{.5\textwidth}
      \SingleSpacing
      \imprimirpreambulo
    \end{minipage}%
    \vspace*{\fill}

    {\large\imprimirorientadorRotulo~\imprimirorientador\par}
    {\large\imprimircoorientadorRotulo~\imprimircoorientador}%
    \vspace*{\fill}

    {\imprimirinstituicao\vspace*{\fill}}

    {\large\imprimirlocal}
    \par
    {\large\imprimirdata}
    \vspace*{1cm}

  \end{center}
}
