\begin{agradecimentos}
Agradeço primeira e imensamente a Deus pela oportunidade de começar este trabalho, 
pela força concedida para que nele perseverasse até o final e pela felicidade em 
vê-lo agora terminado.

Agradeço à minha família, minha esposa Isabela e minha filha Melisa, pelo apoio
incondicional, pela inspiração, motivação e incentivo. Elas sofreram o 
sacrifício de tantos momentos de convivência e partilha, mas por elas é que
esta realização se concretiza. À minha mãe, Dilma, a quem definitivamente devo o 
fato de poder ter chegado até aqui, ela que me ensinou a ter a força em continuar 
mesmo quando tudo é mais difícil e que, mesmo de longe, me protegeu e me animou 
com suas orações poderosas.
 
Obrigado a Chagas, Marcelo Bento e toda minha equipe na Secretaria da 
Fazenda do Ceará, que me deram o suporte necessário para conseguir conciliar as 
obrigações acadêmicas e profissionais, com liberação de faltas e todo o apoio
na realização do trabalho para que os projetos, lá e aqui, continuassem sempre
em andamento.

Agradeço também a confiança e bondade do meu grande amigo e Mestre, Jaime Gama, 
do meu irmão de Equipe de Nossa Senhora, Professor Doutor José Maria Monteiro 
Filho, e meu amigo e colega de SEFAZ, Professor Doutor Paulo Benício,
que gentilmente me concederam as cartas de recomendação que ajudaram a abrir as
portas do PPGIA para que eu ingressasse neste Mestrado.

Meu muito obrigado ao amigo, colega, Mestre e Doutorando Matheus Cunha pela 
colaboração na construção desta Dissertação, pela passagem de experiências, 
imensa ajuda durante todo o curso, pela ``inspiração'' em alguns pontos chave do 
texto ;-) e pelo código contribuído no projeto do Cloud Capacitor.

Agradeço também sobremaneira aos meus orientadores, Professor Doutor Américo Sampaio
e Professor Doutor Nabor Mendonça, pela imensurável colaboração, pela paciência,
dedicação, e, principalmente, por tamanho conhecimento e aprendizado de que pude
desfrutar durante esse período em que me deram a oportunidade de ser aluno.

Obrigado ainda ao Júlio e ao Ronaldo, colegas de mestrado, que juntos
compartilhamos momentos de dificuldade e horas de laboratorio, provas tenebrosas,
prazos, pesquisas, testes, máquinas, manuais. 

\end{agradecimentos}
% ---
