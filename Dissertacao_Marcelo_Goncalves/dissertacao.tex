\documentclass[
  % -- opções da classe memoir --
  12pt,       % tamanho da fonte
  % openright,      % capítulos começam em pág ímpar (insere página vazia caso preciso)
  oneside,      % para impressão em verso e anverso, use twoside.
  a4paper,      % tamanho do papel. 
  % -- opções da classe abntex2 --
  %chapter=TITLE,   % títulos de capítulos convertidos em letras maiúsculas
  %section=TITLE,   % títulos de seções convertidos em letras maiúsculas
  %subsection=TITLE,  % títulos de subseções convertidos em letras maiúsculas
  %subsubsection=TITLE,% títulos de subsubseções convertidos em letras maiúsculas
  % -- opções do pacote babel --
  english,      % idioma adicional para hifenização
  %french,       % idioma adicional para hifenização
  %spanish,      % idioma adicional para hifenização
  brazil        % o último idioma é o principal do documento
  ]{abntex2}

% ---
% Pacotes básicos 
% ---
\usepackage{lmodern}      % Usa a fonte Latin Modern      
\usepackage[T1]{fontenc}    % Selecao de codigos de fonte.
\usepackage[utf8]{inputenc}   % Codificacao do documento (conversão automática dos acentos)
\usepackage{lastpage}     % Usado pela Ficha catalográfica
\usepackage{indentfirst}    % Indenta o primeiro parágrafo de cada seção.
\usepackage{color}        % Controle das cores
\usepackage{graphicx}     % Inclusão de gráficos
\usepackage{microtype}      % para melhorias de justificação
\usepackage{amsmath}      % para fórmulas matemáticas
\usepackage{amssymb}      % para símbolos matemáticos
\usepackage{listings}     % para listagens de código
\usepackage{caption}      % para legendas
\usepackage{subcaption}   % para imagens aninhadas
\usepackage{multirow}     % para row span em tabelas
\usepackage[table,usenames,dvipsnames]{xcolor} % para cores especiais

% ---
% Pacotes de citações
% ---
\usepackage[brazilian,hyperpageref]{backref}   % Paginas com as citações na bibl
\usepackage[alf]{abntex2cite} % Citações padrão ABNT

% --- 
% CONFIGURAÇÕES DE PACOTES
% --- 

% ---
% Configurações do pacote backref
% Usado sem a opção hyperpageref de backref
\renewcommand{\backrefpagesname}{Citado na(s) página(s):~}
% Texto padrão antes do número das páginas
\renewcommand{\backref}{}
% Define os textos da citação
\renewcommand*{\backrefalt}[4]{
  \ifcase #1 %
    Nenhuma citação no texto.%
  \or
    Citado na página #2.%
  \else
    Citado #1 vezes nas páginas #2.%
  \fi}%
% ---
\renewcommand{\lstlistingname}{Listagem}
% ---
% Informações de dados para CAPA e FOLHA DE ROSTO
% ---
\titulo{Processo para Inferência de Desempenho de \\ Configurações para Execução de Aplicações em \\ Ambientes de Nuvem de Infraestrutura}
\autor{Marcelo Canário Gonçalves}
\orientador{Prof. Dr. Américo Tadeu Falcone Sampaio}
\coorientador{Prof. Dr. Nabor das Chagas Mendonça}
\instituicao{%
  Universidade de Fortaleza
  \par
  Programa de Pós-Graduação em Informática Aplicada (PPGIA)}
\local{FORTALEZA}
\data{2014}
\tipotrabalho{Dissertação (Mestrado)}

% O preambulo deve conter o tipo do trabalho, o objetivo, 
% o nome da instituição e a área de concentração 
\preambulo{Dissertação apresentada ao Programa de Pós-Graduação em Informática
Aplicada (PPGIA) da Universidade de Fortaleza como parte dos requisitos
necessários para a obtenção do grau de Mestre em Informática Aplicada.}
% ---

% ---
% Configurações de aparência do PDF final

% alterando o aspecto da cor azul
\definecolor{blue}{RGB}{41,5,195}

% informações do PDF
\makeatletter
\hypersetup{
      %pagebackref=true,
    pdftitle={\@title}, 
    pdfauthor={\@author},
      pdfsubject={\imprimirpreambulo},
      pdfcreator={LaTeX with abnTeX2},
    pdfkeywords={abnt}{latex}{abntex}{abntex2}{trabalho acadêmico}, 
    %colorlinks=false,          % false: boxed links; true: colored links
      linkcolor=blue,           % color of internal links
      %citecolor=blue,           % color of links to bibliography
      filecolor=magenta,          % color of file links
    urlcolor=blue,
    bookmarksdepth=4
}
\makeatother
% --- 

% --- 
% Espaçamentos entre linhas e parágrafos 
% --- 

% O tamanho do parágrafo é dado por:
\setlength{\parindent}{1.3cm}

% Controle do espaçamento entre um parágrafo e outro:
\setlength{\parskip}{0.2cm}  % tente também \onelineskip

%Altera a capa
% Impressão da Capa
\renewcommand{\imprimircapa}{%
\begin{capa}%
\center

% Logomarca da Universidade
\includegraphics[scale=2]{./img/logo.png}\\
\textbf{FUNDAÇÃO EDSON QUEIROZ}\\
\textbf{UNIVERSIDADE DE FORTALEZA -- UNIFOR}

\vfill
\ABNTEXchapterfont\bfseries\LARGE\imprimirtitulo
\vfill

\ABNTEXchapterfont\large\imprimirautor
\vspace*{1cm}
\vfill\vfill\vfill
\large\imprimirlocal

\large\imprimirdata

\vspace*{1cm}
\end{capa}
}
% ---
% Altera a folha de rosto
\renewcommand{\folhaderostocontent}{
  \begin{center}

    {\ABNTEXchapterfont\large\imprimirautor}

    \vspace*{\fill}\vspace*{\fill}
    \begin{center}
      \ABNTEXchapterfont\bfseries\Large\imprimirtitulo
    \end{center}
    \vspace*{\fill}

    \hspace{.45\textwidth}
    \begin{minipage}{.5\textwidth}
      \SingleSpacing
      \imprimirpreambulo
    \end{minipage}%
    \vspace*{\fill}

    {\large\imprimirorientadorRotulo~\imprimirorientador\par}
    {\large\imprimircoorientadorRotulo~\imprimircoorientador}%
    \vspace*{\fill}

    {\imprimirinstituicao\vspace*{\fill}}

    {\large\imprimirlocal}
    \par
    {\large\imprimirdata}
    \vspace*{1cm}

  \end{center}
}


% ---
% compila o indice
% ---
\makeindex
% ---

% ----
% Início do documento
% ----
\begin{document}

% Retira espaço extra obsoleto entre as frases.
\frenchspacing 

% ----------------------------------------------------------
% ELEMENTOS PRÉ-TEXTUAIS
% ----------------------------------------------------------
% \pretextual

% ---
% Capa
% ---
\imprimircapa
% ---

% ---
% Folha de rosto
% (o * indica que haverá a ficha bibliográfica)
% ---
\imprimirfolhaderosto*

% ---
% Inserir a ficha bibliografica
% ---
\input{fichabibliografica}

% ---
% Inserir folha de aprovação
% ---
% Isto é um exemplo de Folha de aprovação, elemento obrigatório da NBR
% 14724/2011 (seção 4.2.1.3). Você pode utilizar este modelo até a aprovação
% do trabalho. Após isso, substitua todo o conteúdo deste arquivo por uma
% imagem da página assinada pela banca com o comando abaixo:
%
% \includepdf{folhadeaprovacao_final.pdf}
%
\begin{folhadeaprovacao}

  \begin{center}
    {\ABNTEXchapterfont\large\imprimirautor}

    \vspace*{\fill}\vspace*{\fill}
    \begin{center}
      \ABNTEXchapterfont\bfseries\Large\imprimirtitulo
    \end{center}
    \vspace*{\fill}
    
%    \hspace{.45\textwidth}
%    \begin{minipage}{.5\textwidth}
%        \imprimirpreambulo
%    \end{minipage}%
%    \vspace*{\fill}
           
   Data de aprovação: 15 de dezembro de 2014.

   \assinatura{\textbf{\mbox{\imprimirorientador}} \\ (Orientador -- UNIFOR)} 
   \assinatura{\textbf{\mbox{\imprimircoorientador}} \\(Coorientador -- UNIFOR)}%
   
   \assinatura{\textbf{\mbox{Prof. Dr. Danielo Gonçalves Gomes}} \\ (Membro -- UFC)}
    \assinatura{\textbf{\mbox{Prof. Dr. Pedro Porfírio Muniz Farias}} \\ (Membro -- UNIFOR)}
   %\assinatura{\textbf{Professor} \\ Convidado 3}
   %\assinatura{\textbf{Professor} \\ Convidado 4}
      
      
   \begin{center}
    \vspace*{\fill}
    %\vspace*{0.5cm}
    {\large\imprimirlocal}
    \par
    {\large\imprimirdata}
    \vspace*{1cm}
   \end{center}
   
   \end{center}
  
\end{folhadeaprovacao}
% ---


% ---
% Dedicatória
% ---
\begin{dedicatoria}
   \vspace*{\fill}
   \centering
   \noindent
   \textit{ À minha esposa, meu Anjo e minha luz, Isabela,\\
   e à minha filha e razão de viver, Melisa.} \vspace*{\fill}
\end{dedicatoria}
% ---


% ---
% Agradecimentos
% ---
\begin{agradecimentos}
Deus, Isabela e Mel, Mãe(orações de longe)\\ 
Chagas, Bento, -- liberação e suporte\\
Jaime Gama, Paulo Benicio, José Maria -- cartas de recomendacao\\
Matheus, -- colaboração\\
Américo e Nabor,\\
Julio e Ronaldo, -- momentos de dificuldade e horas de laboratorio\\
\end{agradecimentos}
% ---


% ---
% Epígrafe
% ---
\begin{epigrafe}
    \vspace*{\fill}
  \begin{flushright}
    \emph{``Porque um dia é preciso parar de sonhar, \\
    tirar os planos da gaveta e, \\
    de algum modo, começar.'' \\
    Amyr Klink}
  \end{flushright}
  \begin{flushright}
    \emph{``O que importa é o conhecimento'' \\
    Sabedoria baiana}
  \end{flushright}
\end{epigrafe}


% ---


% ---
% RESUMOS
% ---
% resumo em português
\setlength{\absparsep}{18pt} % ajusta o espaçamento dos parágrafos do resumo
\begin{resumo}
Um dos principais desafios enfrentados pelos usuários de nuvens que oferecem 
infraestrutura-como-serviço (IaaS) é planejar adequadamente a capacidade dos 
recursos da nuvem necessários às suas aplicações. Este trabalho propõe uma nova 
abordagem para apoiar o planejamento da capacidade de aplicações em nuvens IaaS. 
A nova a abordagem tem como premissa a definição de uma relação de capacidade 
entre as diferentes configurações de recursos oferecidas por um provedor de nuvem, 
com a qual é possível prever (ou ``inferir''), com alto grau de precisão, o 
desempenho esperado de uma aplicação para determinadas configurações de recursos. 
A predição é realizada com base no desempenho observado para outras configurações 
de recursos do mesmo provedor. Dessa forma, a abordagem consegue reduzir, de 
forma significativa, o número total de configurações que precisam ser de fato 
testadas na nuvem (resultados preliminares, obtidos em um ambiente real de nuvem, 
mostram uma redução de mais de 80\% no número total de configurações avaliadas), 
implicando em menores custo e tempo para o processo de planejamento.

 \textbf{Palavras-chaves}: Computação em Nuvem. Planejamento de Capacidade. Inferência de Desempenho.
\end{resumo}

% resumo em inglês
\begin{resumo}[Abstract]
 \begin{otherlanguage*}{english}
One of the main challenges faced by users of infrastructure-as-a-service (IaaS) 
clouds is to correctly plan the resource capacity required for their applications'
needs. This work proposes a new approach to support application capacity planning 
in IaaS clouds. This new approach is based on the definition of a capacity relation 
between different resource configurations offered by a cloud provider which enables 
to predict (or ``infer''), with a high level of accuracy, the expected performance 
of an application for certain resource configurations. The prediction is made based
upon the observed performance for other resource configurations within the same 
provider. The approach significantly reduces the total number of configurations 
effectively tested in the cloud (preliminary results show reductions of over 80\% 
on the number of total tested configurations) resulting in lower costs and time 
for the capacity planning process.

   \vspace{\onelineskip}
 
   \noindent 
   \textbf{Key-words}: Cloud Computing. Capacity Planning. Performance Inference.
 \end{otherlanguage*}
\end{resumo}


% ---
% listas de ilustrações, tabelas, abreviaturas, siglas e símbolos
% ---
\input{listas}

% ---
% inserir o sumario
% ---
\input{sumario}

% ----------------------------------------------------------
% ELEMENTOS TEXTUAIS
% ----------------------------------------------------------
\textual

% ----------------------------------------------------------
% Introdução
% ----------------------------------------------------------
\setcounter{page}{1}
\chapter[Introdução]{Introdução}
% ----------------------------------------------------------
\section{Motivação}
A computação em nuvem é um paradigma computacional que está
transformando a forma de desenvolver e gerenciar aplicações e serviços de
Tecnologia da Informação \cite{Murugesan2014}. Diversas organizações passaram a adotar este paradigma
atraídas pelo seu modelo de negócios onde recursos computacionais (ex:
computação, armazenamento e transferência de dados) podem ser consumidos, sob-demanda, 
como um serviço e pagos de acordo com o consumo efetuado. Um dos principais
desafios enfrentados pelos usuários de nuvens que oferecem 
infraestrutura-como-serviço (IaaS) é planejar adequadamente a capacidade 
dos recursos da nuvem necessários às suas aplicações \cite{Menasce2009}.  

O processo de decisão pela migração de aplicações para o ambiente de nuvens 
computacionais envolve uma série de análises que buscam, entre outras coisas, 
identificar que vantagens a mudança trará de fato \cite{beserra2012cloudstep, 
rodero2010infrastructure}, além de tentar descobrir a melhor maneira de configurar 
a aplicação com os vários recursos oferecidos pelo provedor de nuvem.

Um dos principais recursos computacionais oferecidos por provedores de nuvem que oferecem infraestrutura como serviço (IaaS), que tipicamente representam a parte mais significativa dos gastos dos
clientes, é o serviço de máquina virtual. Em geral, provedores de IaaS cobram um
valor em função do tempo de utilização deste recurso, normalmente medido em horas, 
e esse valor unitário varia conforme o tamanho da máquina virtual (capacidade de 
processamento, memória e espaço de armazenamento). Dessa forma, a apuração do 
custo de operação de uma aplicação em um determinado período de tempo leva em 
conta a quantidade de máquinas virtuais utilizadas bem como seu perfil, ou seja, 
o tamanho e quantidade de recursos usados em cada uma.
 
Para prever o custo de operação de uma aplicação na nuvem, é preciso estimar ou 
medir como a aplicação responderá à demanda submetida em termos de indicadores de 
desempenho. Para se chegar a essa conclusão, faz-se necessário conhecer o 
comportamento da aplicação no ambiente de nuvem para que se identifiquem quais 
perfis de máquinas virtuais oferecidos pelo provedor são capazes de executar a 
aplicação com níveis satisfatórios de desempenho. Somem-se a isso as variações 
da demanda exercidas sobre a aplicação e as diversas possibilidades de variação 
da arquitetura de implantação por meio de procedimentos de escalabilidade.
 
Ao se tomarem procedimentos de escalabilidade vertical (variando-se a quantidade 
de recursos de cada máquina) e/ou de escalabilidade horizontal (variando-se a 
quantidade de máquinas em uma ou mais camadas da aplicação, como dados, apresentação 
e negócio) chega-se a níveis de desempenho e de custo muito diversos. A variação 
da demanda exige que a aplicação também varie em tamanho da implantação, vertical 
ou horizontalmente, conforme a carga aplicada. Quanto mais acentuadas e mais 
frequentes as variações na demanda, mais variações de custo e desempenho serão 
observadas.

Assim, o custo apresenta-se entre os mais difíceis de prever, uma vez que depende 
necessariamente do tamanho da demanda exercida sobre a aplicação além do desempenho 
oferecido e preços cobrados pelo provedor de nuvem de infraestrutura contratado 
\cite{cunha2012ambiente}. Estrategicamente, torna-se fundamental identificar, 
entre as possíveis composições de máquinas virtuais ofertadas em um ou vários 
provedores, quais são as configurações de menor custo capazes de executar a 
aplicação mantendo-se os níveis considerados satisfatórios para os indicadores de desempenho.

Para facilitar o entendimento do problema, suponha o seguinte exemplo. Uma
determinada organização quer utilizar um provedor de nuvem para implantar uma
determinada aplicação web multicamadas. Para saber se uma determinada 
configuração de recursos do provedor (ex: uma máquina virtual de tamanho
pequeno para a camada de aplicação e outra para a a camada de dados) é capaz de
atender a uma demanda específica (ex: carga imposta por 100 usuários
concorrentes), é preciso antes enumerar os indicadores de desempenho que mais
interessam à aplicação (ex: tempo de resposta) e a partir daí estabelecer os
valores aceitáveis para esses indicadores (ex: tempo de resposta abaixo de 10
segundos para uma determinado conjunto de operações requisitadas por 100 usuários concorrentes).
Uma vez estabelecidos os valores aceitáveis, pode-se implantar a aplicação 
sob essa configuração de recursos no ambiente da nuvem e então testar o
desempenho da aplicação. Ao comparar a resposta da aplicação com os valores
dados como aceitáveis para os indicadores, é possível determinar se aquela 
configuração de recursos escolhida é capaz de executar a aplicação a contento e 
ainda calcular o custo por período (por exemplo, mensal ou anual) associado ao uso dessa configuração.

Porém, partindo-se do pressuposto de que o desempenho da aplicação foi satisfatório, 
o que se tem até agora é o custo de uma única configuração capaz de executar a 
aplicação estudada sob um único nível de carga de trabalho. No entanto, cargas de 
trabalho costumam variar em função do tempo em implantações reais, fazendo-se 
necessário, portanto, que esse efeito seja contemplado nos testes por meio da 
medição do desempenho da aplicação submetida a diferentes níveis de carga de 
trabalho.

Analogamente, as diversas configurações de máquinas e recursos, ainda que no mesmo 
provedor, podem responder de maneira diferente sob o mesmo nível de carga de 
trabalho, a depender do momento em que sejam ativados \cite{cunha2011investigating, 
iosup2011performance, jayasinghe2011variations}. Independente do motivo que leve 
a esse comportamento de certa forma imprevisível, é preciso levar em conta nos 
ensaios de avaliação de desempenho essa variabilidade, o que pode pode ser alcançado 
através da repetição dos cenários de teste em horários e dias diferentes.

Um grande problema começa a se desenhar ao seguir essa abordagem: a fase de
experimentação pode atingir patamares elevados de custo, a depender das
necessidades de variação da demanda, da arquitetura de implantação e das 
configurações utilizadas em cada arquitetura implantada \cite{silva2013cloudbench}. 
Ainda que certos provedores IaaS ofereçam descontos ou pacotes de horas grátis 
para novos clientes, em geral esses incentivos são suficientes para custear apenas 
um mês de utilização de uma única máquina virtual muito pequena, provavelmente 
incapaz de suportar a carga de uma aplicação real em produção. Assim, executar 
uma aplicação real, tipicamente implantada em arquitetura de várias camadas, em 
máquinas virtuais de tamanho considerável e por longos períodos de tempo apenas 
para estudar o seu comportamento, pode se traduzir em um custo alto que dificulte ou até mesmo inviabilize 
o próprio projeto de migração dessa aplicação para a nuvem. 

Com o intuito de ajudar a resolver o problema do planejamento de capacidade de aplicações na nuvem, vários trabalhos foram propostos, o quais podem ser classificados de acordo com duas abordagens: preditiva \cite{cloudharmony,
malkowski2010cloudxplor, fittkau2012cdosim, li2011cloudprophet} e empírica 
\cite{jayasinghe2012, silva2013cloudbench, cunhacloud, scheuner2014cloud}.
Trabalhos que adotam a abordagem preditiva têm como objetivo tentar prever o desempenho esperado da aplicação alvo quando executada na nuvem sob determinada configuração de recursos e determinado nível de carga. As técnicas de predição variam conforme o trabalho, sendo as mais comuns o uso de simuladores de nuvem \cite{fittkau2012cdosim} e a analogia com resultados previamente obtidos através da utilização de \emph{benchmarks} \cite{cloudharmony,
malkowski2010cloudxplor}. Uma desvantagem dos trabalhos que seguem a abordagem preditiva é a sua ainda limitada capacidade para prever as melhores configurações dos recursos da nuvem para atender variadas cargas impostas à 
aplicação. Isto se deve principalmente a fatores como: diferenças conceituais entre o comportamento real da aplicação e  
os modelos de simulação, a ausência de \emph{benchmarks} com funcionalidades e demandas similares aos da aplicação, além da instabilidade comuns aos serviços de nuvem. Por outro lado, trabalhos da abordagem preditiva têm a grande vantagem de não onerar os usuários, pois não exigem a implantação e execução da aplicação no ambiente de nuvem real.

Trabalhos que utilizam a abordagem empírica são baseadas na implantação das aplicações alvo na nuvem 
e em testes de carga. Esses trabalhos normalmente oferecem suporte aos usuários
para automatizar as tarefas de instalar, configurar e testar as aplicações no
ambiente da nuvem, eliminando uma série de procedimentos manuais custosos. Por
executarem no ambiente de nuvem real, eles conseguem ajudar a atingir resultados
significativamente melhores no que diz respeito a seleção de recursos
computacionais para cargas de trabalho específicas. No entanto, um problema ainda 
existente nos trabalhos dessa abordagem é a necessidade de se testar exaustivamente uma grande 
quantidade de configurações de recursos e cargas de trabalho, implicando em altos 
custos durante a fase de experimentação.

Visando combinar as vantagens das duas abordagens, este trabalho
propõe uma nova maneira de apoiar o planejamento da capacidade de aplicações
em nuvens IaaS. A nova abordagem tem como premissa a definição de uma relação 
de capacidade entre as diferentes configurações de recursos oferecidas por um 
provedor de nuvem, com a qual é possível prever (ou ``inferir''), com alto grau 
de precisão, o desempenho esperado de uma aplicação para determinadas configurações 
de recursos. A predição é realizada com base no desempenho observado (em testes 
na nuvem) para outras configurações de recursos do mesmo provedor.

\section{Objetivos}
Este trabalho tem como principal objetivo propor uma abordagem de
planejamento de capacidade baseado nas relações de capacidade existentes em
recursos da nuvem e também em um processo de inferência de desempenho 
que utiliza como insumo resultados de execuções reais na nuvem. Os objetivos
específicos do trabalho são:

\begin{itemize}
  \item Definir uma forma de representar as relações de capacidade entre
  configurações de recurso de um provedor de nuvem;
  \item Propor um processo de avaliação de capacidade que seja capaz de 
  prever as melhores configurações de recurso capazes de suportar variadas
  cargas de trabalho;
  \item Fornecer uma implementação concreta do processo proposto;
  \item Avaliar a eficiência e efetividade do processo através de um estudo
  baseado em uma aplicação concreta em um provedor de nuvem real.
\end{itemize}
 
\section{Contribuições}
Com o intuito de resolver a problemática do planejamento de capacidade na nuvem,
atendendo aos objetivos propostos, algumas contribuições podem ser destacadas
deste trabalho.

Primeiramente, é proposto um modelo que representa as relações de capacidade de
configurações de uma nuvem. Este modelo é chamado de Espaço de Implantação
e consiste de uma estrutura em grafo que representa as relações entre os
Tipos de Máquinas Virtuais oferecidos pelo Provedor de nuvem. O grafo é
construído de forma a indicar as relações de capacidade entre os tipos de
instância indicando quais tipos são maiores, menores ou não relacionados com
outros.

Outra contribuição significativa é o Processo de Avaliação de Capacidade por 
Inferência de Desempenho. Esse processo define uma nova abordagem para o 
planejamento de capacidade com base na inferência do comportamento de máquinas 
virtuais. A inferência se dá sob a luz da análise da relação de capacidade 
representada pelo Espaço de Implantação e a partir do resultado da execução de 
testes reais de desempenho em relativamente poucas máquinas dentre o conjunto completo de 
configurações avaliadas.

Para que o Processo saiba identificar quais configurações devem ser efetivamente
testadas e em que ordem, foram definidas Heurísticas para a seleção dessas 
configurações. A cada passo da iteração do Processo, a Heurística selecionada é
acionada para que, a depender do resultado obtido no passo anterior, uma nova
configuração de máquinas virtuais seja levada a testes.

As Heurísticas conferem grande flexibilidade ao Processo, que pode se adaptar a
diferentes comportamentos da Aplicação sob Teste. Além de já ter definido um 
conjunto inicial, o Processo prevê a possibilidade de que novas Heurísticas possam
ser escritas e aplicadas como pontos de extensão à implementação original.

Quando um estudo de capacidade está sendo executado por um profissional, é 
intuitivo que não sejam testadas todas as combinações possíveis de configurações
e cargas de trabalho. Porém, no momento da realização dos testes, a decisão de
qual será a próxima configuração a ser testada é um problema de difícil solução.
Com custos altos envolvidos já durante a fase de execução de testes e com 
possivelmente dezenas de configurações a serem examinadas, como saber qual a
configuração a ser executada a fim de minimizar o custo e maximizar as chances
de obter respostas conclusivas quanto ao planejamento de capacidade em andamento?

Essa é outra contribuição consequente da aplicação da Processo de Inferência de
Desempenho: a minimização do custo de execução de testes e a maximização das 
respostas para o planejamento de capacidade, com altíssimo nível de confiança na
acurácia dessas respostas.

Este trabalho apresenta ainda uma implementação concreta usada em experimentos para 
validar o Processo proposto. Os resultados desses experimentos comprovam sua eficiência 
na redução de custos e sua acurácia ao apontar quais configurações conseguem 
atender a um acordo de nível de serviço estipulado previamente. 

\section{Estrutura da Dissertação}
Além desta Introdução, esta dissertação está organizada em mais cinco capítulos, 
descritos brevemente a seguir.

\begin{description}
  \item[Capítulo 2 -- Fundamentação Teórica e Trabalhos Relacionados] \hfill \\
  Aborda alguns conceitos fundamentais da computação em nuvem e planejamento de
  capacidade, de modo a nivelar o conhecimento do leitor sobre o tema. Além disso, 
  são descritos e analisados criticamente diversos trabalhos relacionados a esta pesquisa.
  \item[Capítulo 3 -- Processo de Avaliação de Capacidade] \hfill \\
  Apresenta os conceitos criados especificamente no âmbito desta pesquisa, explicando 
  a terminologia usada no texto e que permeiam a solução proposta. Descreve em 
  detalhes o Processo proposto para a avaliação de capacidade baseado em
  inferência de desempenho.
  \item[Capítulo 4 -- Implementação do Processo] \hfill \\
  Descreve os detalhes da implementação concreta do processo proposto, mostrando
  como foi construída a biblioteca de programação que fornece a funcionalidade
  necessária à execução de rotinas de avaliação de capacidade. Apresenta também
  o sistema construído com a utilização da biblioteca, validando sua 
  aplicabilidade. 
  \item[Capítulo 5 -- Experimentos e Resultados] \hfill \\
  Onde são apresentados os experimentos criados como parte deste estudo a fim de
  verificar a aplicabilidade do processo de inferência de desempenho. Os resultados 
  dos experimentos são analisados e vêm comprovar a eficiência e acurácia do 
  processo. 
  \item[Capítulo 6 -- Conclusão] \hfill \\
  Faz o fechamento do trabalho, apontando as contribuições trazidas e sugestões
  para trabalhos futuros. 
\end{description}
% ----------------------------------------------------------

% ----------------------------------------------------------

% ----------------------------------------------------------
% Trabalhos Relacionados
% ----------------------------------------------------------
\chapter[Trabalhos Relacionados]{Trabalhos Relacionados}
% ----------------------------------------------------------
Apresentamos  neste capítulo alguns trabalhos cujos objetivos estão alinhados 
com a ideia da avaliação de desempenho de aplicações executadas em ambientes de
computação em nuvem. Esses trabalhos estão agrupados de acordo com a abordagem utilizada para a realização da avaliação de desempenho. São duas abordagens, a primeira é a abordagem preditiva, nesta, os trabalhos não executam diretamente a aplicação alvo no ambiente onde se deseja implantá-la. Já a segunda abordagem é a empírica, nesta as aplicações alvo são implantadas na nuvem e então submetidas a testes de carga. 

Para apoiar a análise crítica de cada trabalho e suas abordagens, definimos um conjunto de critérios a partir dos quais poderemos tanto descrever, quanto comparar as soluções existentes de apoio a avaliação de desempenho do ponto de vista do usuário. Seguem os critérios:

\begin{enumerate}
  \item \textbf{Completude da solução}
  \begin{enumerate}
    \item \textbf{Definição da aplicação} --- flexibilidade da solução para
    definir a aplicação a ser avaliada;
    \item \textbf{Definição da demanda} --- flexibilidade da solução para
    definir os níveis de carga de trabalho sobre os quais a aplicação será
    submetida;
	\item \textbf{Definição dos recursos da nuvem} --- flexibilidade da solução
	para definir as configurações de recursos do provedor sobre as quais a aplicação
	será executada;
	\item \textbf{Definição do acordo de nível de serviço (SLA)} --- flexibilidade
	da solução para definir o SLA desejado.
  \end{enumerate}
  \item \textbf{Efetividade da solução}  
  \begin{enumerate}
    \item \textbf{Eficiência} --- tempo e custo necessários para a solução
    resolver o problema;
    \item \textbf{Acurácia} --- confiabilidade das respostas oferecidas pela
    solução;
	\item \textbf{Complexidade} --- grau de complexidade/esforço exigido do usuário
	da solução.
  \end{enumerate}
\end{enumerate}

Cada abordagem será apresentada em uma subseção, bem como os diversos trabalhos da área que serão resumidos separadamente. Ao final da subseção, será apresentada uma análise crítica que será baseada nos critérios descritos acima.

\section{Ferramentas com Abordagem Preditiva}
As ferramentas que serão apresentadas nesta seção têm em comum o fato de
indicarem a configuração de implantação na nuvem sem executarem avaliações de
desempenho diretamente na aplicação que será migrada para a nuvem. Dessa forma,
elas prevêem a configuração de implantação, por isso estão sendo chamadas de
abordagens preditivas. Muitas das soluções de abordagem preditiva realizam as
suas predições após a caracterização da performance dos recurso da nuvem, o que
é realizado através da execução de \textit{benchmarks}. Já o
\textit{CloudProphet}, apresentado em~\cite{li2011cloudprophet}, coleta como a
aplicação alvo faz uso dos recursos computacionais em um ambiente controlado e
então repete essa utilização em uma nuvem candidata. Dessa forma, não necessita
caracterizar a performance dos recursos da nuvem.

A seguir, apresentaremos seis trabalhos que se destacam na avaliação de
desempenho de aplicações na nuvem usando uma abordagem preditiva, são
eles:~\cite{cloudharmony},~\cite{malkowski2010cloudxplor},~\cite{li2011},~\cite{jung2013cloudadvisor},~\cite{fittkau2012cdosim}
e~\cite{li2011cloudprophet}.

\subsection{Cloud Harmony}
O projeto {\em CloudHarmony}, cujo
objetivo é ``tornar-se a principal fonte independente, imparcial e útil de
métricas de desempenho de provedores de nuvem''~\cite{cloudharmony}, agrega
dados de testes de desempenho realizados desde 2009 em mais de 60 provedores de
nuvem. Além do histórico das avaliações, o {\em CloudHarmony} disponibiliza uma ferramenta para executar novas avaliações de desempenho a qualquer momento, denominada
\textit{Cloud Speed Test},\footnote{\url{http://cloudharmony.com/speedtest}}, a qual permite realizar quatro tipos de teste:

\begin{description}
  \item[\em Download a few large files] --- objetiva determinar o melhor provedor
  para descarregar arquivos grandes, sendo útil para aplicações como {\em video
  streaming};
  \item[\em Download many small files] --- objetiva determinar o melhor provedor
  para descarregar arquivos pequenos, podendo ser útil para hospedar uma página
  web, por exemplo;
  \item[\em Upload] --- útil para avaliar serviços que serão utilizados para
  envio de arquivos;
  \item[\em Test network latency] --- a latência afeta o tempo de resposta da
  aplicação e geralmente está relacionada com a região de onde o teste está
  partindo.
\end{description}

Os resultados disponibilizados pelo {\em CloudHarmony} têm como pontos fortes a
grande quantidade de dados de testes de desempenho disponíveis, além da possibilidade do cliente da
nuvem poder executar novos testes a qualquer tempo. Por outro lado, os testes estão limitados àqueles implementados pela ferramenta de teste, não podendo ser facilmente modificados para contemplar novas métricas ou cenários de avaliação.

\subsection{{\em CloudXplor}}

{\em CloudXplor}~\cite{malkowski2010cloudxplor} é uma ferramenta para
planejamento de configuração de recursos da nuvem baseada em dados empíricos. A ferramenta
foi desenvolvida tomando como base um modelo de planejamento de configuração de
recursos de Tecnologia da Informação (TI), com foco explícito em aspectos econômicos.
Por essa razão, a ferramenta se utiliza de acordos de nível de serviço baseados na relação do custo
da infraestrutura de TI com o valor dos recursos do provedor do serviço. Esse
valor será maior quando o tempo de resposta da aplicação for plenamente atendido
pelo provedor do serviço, e vai diminuindo à medida em que esse tempo de resposta
não é alcançado.

Os dados empíricos precisam ser coletados, previamente, através da execução de diversos experimentos de avaliação de desempenho. Esses dados
são compostos por métricas de sistema (uso de CPU, memória utilizada, tráfego na rede e E/S de disco) e métricas de mais alto nível (tempo de resposta e \textit{throughput}). Após a coleta, os dados dos experimentos são submetidos e analisados pela ferramenta, utilizando um de seus quatro módulos: análise de tempo de
resposta, análise de \textit{throughput}, análise do valor agregado e do custo,
e análise do lucro. Cada um desses módulos filtra os dados, fazendo uso apenas
das informações necessárias para a execução da sua análise. Após a análise dos
 dados, a ferramenta pode ser utilizada para produzir gráficos que ilustram o comportamento da aplicação ao se variar
parâmetros como carga de trabalho e configuração dos componentes da aplicação.


%A ferramenta proposta no trabalho faz uso de dados previamente coletados
%e só então realiza a análise do desempenho de aplicações na nuvem
%em diversos recursos. Por isso deixa a cargo do cliente da nuvem todas as
%atividades para a avaliação de desempenho, pois depende dos dados empíricos que
%são coletados após a avaliação. O CloudXplor é capaz de plotar um gráfico com o
%comportamento da aplicação à medida em que é exposta a variação na demanda, do
%custo e do valor, mas só o faz por que o cliente da nuvem executou cada uma das
%avaliações manualmente. Da mesma forma a variação na demanda vai depender do
%conjunto de avaliações executadas pelo cliente da nuvem, a ferramenta apenas
%plota gráficos do que foi executado pelo cliente. Com relação a definição de
%cenários de avaliação e a definição de parâmetros de desempenho, a ferramenta
%não dá nenhum suporte, uma vez que os cenários dependem das avaliações
%realizadas e que os módulos de análise de custo, valor, tempo de resposta e
% {\em throughput} possuem uma lista dos parâmetros de desempenho possíveis de
%avaliação.


\subsection{CloudCmp}
\citeonline{li2011} apresentam uma ferramenta para apoiar a avaliação e a comparação do
desempenho e do custo dos recursos e serviços de diversos provedores de nuvem
pública, de modo a auxiliar o cliente da nuvem a escolher o provedor mais adequado para a sua
aplicação. Essa ferramenta, denominada {\em CloudCmp}, analisa
os serviços de elasticidade, persistência de dados e rede oferecidos pelos provedores
de interesse, com base em resultados previamente coletados a partir da execução de diversos
{\em benchmarks}: uma versão modificada do {\em SPECjvm2008}~\cite{SPECjvm2008}
para avaliar a característica de elasticidade do provedor; um cliente Java para avaliar os serviços de
armazenamento e persistência de dados; e as ferramentas {\em iperf}\footnote{http://iperf.sourceforge.net. }
e {\em ping} para avaliar os serviços de rede. Após a fase inicial da coleta dos dados, a ferramenta pode ser utilizada para gerar gráficos que auxiliem o
cliente da nuvem a comparar o desempenho dos recursos de cada um
dos provedores nos quais as avaliações foram realizadas, que assim poderá escolher o provedor e os recursos mais apropriados para as necessidades e demandas específicas de suas aplicações. 

%Os gráficos gerados nos
%experimentos apresentados neste trabalho foram comparados com os gerados após o
%estudo do comportamento de três aplicações simples implantadas na nuvem. Essas
%comparações mostraram que as previsões do CloudCmp refletiram o comportamento
%das aplicações testadas.

Segundo \citeonline{li2011}, até a época do trabalho não houve
nenhum provedor de nuvem que se destacasse com relação aos demais. Outra constatação foi de que os resultados
obtidos a partir da execução dos {\em benchmarks} em cada provedor apenas refletiam o momento em que foram coletados, uma vez que a estrutura utilizada pelos provedores para hospedar seus serviços sofre frequentes modificações e a demanda por seus recursos computacionais é bastante variável.

%Como essa solução compara serviços e recursos da nuvem através da análise de
% dados previamente coletados a partir da execução de diferentes {\em benchmarks}, a
%escolha dos provedores e recursos mais apropriados para uma determinada
% aplicação só será eficaz se a aplicação utilizar os recursos da nuvem de forma semelhante à dos {\em benchmarks} avaliados. Além disso, a ferramenta não oferece suporte à execução de novos cenários de avaliação na nuvem, estando limitada àqueles previamente definidos para os respectivos {\em bechmarks}.

\subsection{CloudAdvisor}
O trabalho apresentado em~\cite{jung2013cloudadvisor} ``introduz uma nova plataforma de recomendação de nuvem, chamada {\em CloudAdvisor}''. Essa plataforma destina-se a auxiliar o seu usuário na tarefa de capturar as implicações monetárias e financeiras das configurações de implantação das suas aplicações. Para recomendar a configuração, a platforma recebe como entrada parâmetros de configuração de alto nível como, orçamento, expectativa de performance e economia de energia, os quais estão limitados a uma escala discreta que vai de 0 até 10, onde 0 significa baixa influência e 10 significa alta influência. Uma vez informados os parâmetros de configuração, o {\em CloudAdvisor} irá caracterizar a performance da aplicação alvo em termos de uso dos recursos computacionais e em seguida executará o {\em benchmark} {\em CloudMeter},~\cite{jung2013cloudadvisor}, nas nuvens candidatas, a fim de caracterizar a performance dos recursos dessas nuvens.

Para ilustrar o uso do {\em CloudAdvisor}, os autores implantaram a solução em servidores locais e em três provedores de nuvem pública, foram eles: Windows Azure~\cite{azure}, Rackspace~\cite{rackspace}, and Amazon EC2~\cite{ec2}. Como resultado, foi observado que a taxa de erro da configuração para uma determinada carga foi de 10~\%. No entanto, quando o usuário do ambiente escolhe parâmetros de configuração extremos (por exemplo, configurar o máximo de orçamento, de economia de energia e de nível de performance), essa taxa de erro elevou para 18~\%.

Os autores concluem que o usuário da solução pode explorar diversas opções de configuração da sua aplicação na nuvem utilizando uma interface amigável e sem a necessidade de informar detalhes específicos de configuração. Além disso, mostraram que é possível utilizar uma técnica de caracterização de performance, para uma dada carga, baseada na execução de {\em benchmarks} em nuvens candidatas.

\subsection{CDOSim}\label{subsec:CDOSim}
Uma solução para o desafio da escolha da configuração de implantação de uma aplicação na nuvem foi apresentada em~\cite{fittkau2012cdosim} com o nome de \textit{CDOSim}. Essa solução auxilia o usuário no processo de escolha do que os autores chamaram de opção de implantação na nuvem --- do inglês \textit{Cloud Deployment Option} (CDO)---, uma vez que a análise manual das ``potenciais CDOs é intratável, custosa e consome tempo, devido à heterogeneidade dos ambientes de nuvem'',~\cite{fittkau2012cdosim}. Para a escolha das CDOs, são realizadas simulações que se baseiam no custo e nas propriedades de performance de cada CDO. O custo é informado pelo provedor de nuvem, já a caracterização da performance é realizada através da execução de um \textit{benchmark} para medir a quantidade de \textit{mega integer plus instructions per second} (MIPIPS)~\cite{fittkau2012cdosim}, por opção de configuração. O código desse \textit{benchmark} deve ser gerado para cada linguagem de programação utilizada pela aplicação alvo. Esta, por sua vez, deve passar por um processo de engenharia reversa para modelos KDM, que é descrito em~\cite{perez2011knowledge}, para que o simulator consiga escolher a opção de configuração mais adequada.

Para ilustrar o uso da solução, os autores executaram três tipos de experimentos, um para validar o uso de MIPIPS para caracterizar a performance das opções de configuração, outro para comparar os resultados da simulação com dados reais, e um último experimento para verificar a possibilidade da predição da performance de um provedor de nuvem com base nos dados de outro provedor de nuvem. Esses experimentos foram conduzidos em dois ambientes de nuvem, sendo uma pública, a Amazon EC2, e outra privada. Na nuvem pública foi evidenciado que os valores de MIPIPS dependem da região onde o \textit{benchmark} foi realizado e da carga sobre a máquina física que hospeda a máquina virtual. Já na comparação dos resultados da simulação com os dados reais, os autores mostraram que a taxa de error da utilização de CPU simulada com a utilização de CPU medida, chegou a 30,86~\%. Contudo, a taxa de erro médio global foi abaixo de 22,75~\%, portanto abaixo do limiar estabelecido pelos autores que era de 30~\%. Já a predição da performance de uma instância da Amazon EC2 a partir da performance de uma instância da nuvem privada gerou 15,76~\% como taxa de erro global.

Finalmente, os autores concluem que a simulação pode auxiliar no processo de escolha da opção de implantação com maior performance e menor custo e que os resultados da simulação são razoavelmente próximos dos valores reais.

\subsection{CloudProphet}\label{subsec:CloudProphet}
Em \cite{li2011cloudprophet} os autores apresentam o \textit{CloudProphet}, um sistema de
predição de desempenho de aplicações em ambiente de nuvem computacional baseado 
na metodologia de ``rastrear e reproduzir'' (\textit{trace and replay}).

O \textit{CloudProphet} não testa a aplicação do cliente de fato no ambiente de nuvem. De
modo contrário, ele injeta na implantação original da aplicação um módulo que 
registra um rastreamento detalhado dos eventos de utilização de recursos de CPU,
armazenamento e rede em cada componente da aplicação durante um período de 
execução habitual em seu ambiente de produção.

Em um passo seguinte, outro módulo faz uma extração das relações de dependência 
entre os eventos coletados, ordenando as transações executadas nos diversos 
componentes.

O terceiro passo é a reprodução dos eventos coletados durante a fase de 
rastreamento. Essa reprodução consiste fazer com que o ambiente de nuvem 
computacional que se deseja avaliar execute as transações representadas nos dados
do rastreamento a partir de requisições que partem de clientes simulados.
   
O objetivo do \textit{CloudBench}, segundo os autores é eliminar o custo e o trabalho 
envolvidos na migração da aplicação real para a nuvem para a execução de testes 
antes que seja de fato tomada a decisão em favor dessa migração.

Os autores argumentam que a simples implantação da aplicação no ambiente de um
serviço de nuvem computacional já incorre em custos, que podem ser altos a 
depender do tamanho ou da arquitetura da aplicação. Além disso, a tarefa de
migração pode ser bastante trabalhosa conforme o número e a diversidade dos 
componentes da aplicação, que podem acarretar dificuldades de configuração e
compatibilidade no novo ambiente.

\section{Ferramentas com Abordagem Empírica}
As ferramentas que serão apresentadas nesta seção têm em comum o fato de utilizarem como aplicação alvo a prória aplicação que se deseja implantar na nuvem. Portanto, é preciso inicialmente realizar uma implantação na nuvem para que seja dado início ao processo de análise de desempenho. Por isso, cada ferramenta oferece um mecanismo para que o seu usuário possa definir como a aplicação deve ser implantada e configurada. Além da definição da aplicação e dos recursos da nuvem que serão utilizados, essas ferramentas também permitem que sejam definidas a demanda que será imposta a cada aplicação e o acordo de nível de serviço. Dessa forma, é possível definir, por exemplo, o número de usuários simultâneos e o tempo de resposta esperado para uma transação.

Uma vez que que as ferramentas que realizam a abordagem empírica possibilitam ao usuário muita liberdade na definição da aplicação, demanda, recurso da nuvem e SLA, essas soluções têm o mais alto grau de completude. Além disso, como faz-se uso da própria aplicação alvo para a avaliação de desempenho, a acurácia dos resultados apresentados pelas ferramentas é a mais elevada. A seguir, apresentaremos quatro trabalhos que se destacam na avaliação de desempenho de aplicações na nuvem usando a abordagem empírica, são eles:~\cite{jayasinghe2012},~\cite{silva2013cloudbench},~\cite{cunhacloud} e~\cite{scheuner2014cloud}.

\subsection{Expertus}
Devido à complexidade e às implicações da escolha da configuração para a implantação de uma aplicação na nuvem, em~\cite{jayasinghe2012} os autores apresentam o \textit{Expertus}, que é descrito como ``um framework flexível de geração de código para automatizar testes de performance de aplicações distribuídas em nuvem de infraestrutura''. Essa geração automática de código é realizada a partir de {\em templates} especificados na forma de documentos XML~\cite{jayasinghe2012}. Os templates utilizados nas avaliações de desempenho devem ser escritos pelo usuário e servem de entrada para o ambiente, que realiza diversas transformações nessa entrada até a forma de \textit{shell scripts}. Esses \textit{scripts}, por fim, possuem os comandos para a configuração da avaliação de desempenho na aplicação alvo.

Como demonstração da usabilidade da ferramenta, os autores apresentaram em~\cite{jayasinghe2012} resultados de experimentos realizados com duas aplicações alvo. Cada uma das aplicações foi avaliada com duas opções de sistemas de gerenciamento de bancos de dados, o que demonstrou também como diferentes opções de configuração poderiam ser utilizadas nas aplicações. Além da demonstração da usabilidade, os autores realizaram experimentos para evidenciar a magniture e tipos de \textit{scripts} que podem ser gerados pela ferramenta. Como exemplo da magnitute, para a realização de experimentos com 48 nós, o total de linhas de scripts geradas pelo \textit{Expertus} girou em torno de 15 mil. Por fim, os autores demonstraram o que chamaram de ``riqueza da ferramenta'', que foi comprovada através da execução de experimentos em 5 nuvens (por exemplo, Amazon EC2 e Open Cirrus~\cite{avetisyan2010open}).

Dessa forma, pode-se concluir que a ferramenta apresentada minimiza a ocorrência de falhas humanas na avaliação de desempenho de uma aplicação implantada em diversos nós. Além disso, os mesmo experimentos podem ser repetidos em diferentes provedores de nuvem pública. De modo que mais cenários de implantação podem ser considerados para a escolha do mais adequado para a aplicação.

\subsection{CloudBench}
\cite{silva2013cloudbench} descreve o \textit{CloudBench} como um arcabouço para automação da avaliação de desempenho de ambientes de nuvem computacional sob o modelo IaaS. As abstrações apresentadas neste trabalho permitem que um experimento seja especificado através de uma lista de diretivas as quais descrevem os itens que compõem o experimento. São exemplos desses itens, objetos como a aplicação alvo, as instâncias de máquinas virtuais utilizadas, e as métricas de desempenho que são tanto relativas à aplicação alvo, quanto ao serviço do provedor de nuvem (por exemplo, latência de provisionamento).
%Para a análise dos resultados, o ambiente coleta dados relativos às métricas de desempenho observadas, através %da abstração de experimentos, aplicações e máquinas virtuais.
%Além disso, o ambiente prevê métricas que dizem respeito não só à aplicação, mas também ao provedor, como latência de provisionamento.
%, bem como a execução dos 
%testes e a coleta de dados relativos às métricas de desempenho observadas, através 
%da abstração de experimentos, aplicações e máquinas virtuais.   

Para a realização dos experimentos, o \textit{CloudBench} faz a implantação automática da aplicação a ser executada para efeito de testes. Portanto, o acompanhamento é realizado desde a criação da máquina virtual no ambiente até a coleta dos dados de desempenho e desligamento das máquinas. Essas características fazem do CloudBench uma ferramenta muito poderosa para a
automação de testes e coleta de dados para análise das execuções. Suas ferramentas
de monitoramento fornecem informações com grandes níveis de detalhamento a respeito
de cada componente implantado e usado nos testes, proporcionando excelente embasamento
para a tomada de decisão.

Entretanto, embora o CloudBench tenha um escopo de solução muito mais amplo, 
voltado para a avaliação de desempenho tanto da aplicação do cliente como do 
provisionamento de máquinas pelo provedor, seu alvo no momento da execução de 
testes está restrito a \textit{benchmarks} pré-definidos, não permitindo a execução de 
uma aplicação real no ambiente testado.

%Neste sentido, o arcabouço que propomos com este trabalho se diferencia pelo fato
%de ser agnóstico em relação à aplicação que deverá ser testada, assim como quanto 
%às métricas que a ela concernem, conferindo ao usuário da solução a oportunidade 
%de avaliar o comportamento da aplicação de seu interesse implantada no ambiente
%pretendido e sob a perspectiva que lhe for mais conveniente, inclusive em termos
%de arquitetura de implantação. 

\subsection{Cloud Crawler}
Este trabalho apresenta um ambiente programável para apoiar os usuários de nuvens IaaS na realização de testes automáticos de desempenho de aplicações na nuvem. As principais contribuições do ambiente são: a linguagem declarativa {\em Crawl}, com a qual os usuários podem especificar, através de uma notação simples e de alto nível de abstração, uma grande variedade de cenários de avaliação de desempenho de uma aplicação na nuvem; e o motor de execução {\em Crawler}, que automaticamente executa e coleta os resultados dos cenários descritos em {\em Crawl} em um ou mais provedores. Essas duas ferramentas são denominadas conjuntamente de {\em Cloud Crawler}~\cite{cunhacloud}.

Para iniciar os testes de desempenho de uma aplicação através do ambiente {\em Cloud Crawler}, os componentes dessa aplicação precisam ser declarados em um \textit{script} da linguagem {\em Crawl}. Compõem esse \textit{script Crawl}, por exemplo, o provedor de nuvem, os tipos de máquinas virtuais e as máquinas virtuais que serão utilizadas nas avaliações, além disso, métricas de desempenho e a demanda imposta à aplicação também irão compor o cenário de avaliação que é declarado no \textit{script Crawl}. Finalizada essa etapa de declaração, o usuário do ambiente irá submeter o \textit{script crawl} para o motor de execução {\em Crawler}. Esse motor irá iniciar todas as máquinas virtuais, caso seja necessário, irá proceder com a modificação do tipo de máquina virtual, de acordo com o que estiver declarado. Após a inicialização de cada máquina virtual, o motor pode executar alguma configuração nessa máquina, por exemplo, a configuração do endereço ip de um banco de dados, ou a configuração do total de memória utilizado por uma máquina virtual java. Todas as configurações necessárias para a aplicação executar na nuvem devem estar declaradas no {\em script Crawl} que foi submetido para o motor. Quando a última máquina virtual é configurada, o motor {\em Crawler} executa um por um os cenários de avaliação, com suas respectivas demandas, e ao mesmo tempo coleta as métricas de desempenho especificadas. As métricas de desempenho podem ser tanto métricas de sistema, como percentual de CPU utilizado e de memória RAM, quanto métricas de aplicação, como o tempo de resposta de uma aplicação WEB.

A fase de mapeamento dos componentes da aplicação é realizada apenas uma vez, enquanto que a submissão para o motor de execução pode ser repetida ao critério do usuário. Ambientes como o {\em Cloud Crawler} permitem que os seus usuários testem suas aplicaçãoes em difirentes cenário de implantação e possibilitam que o mesmo entenda o comportamento da sua aplicação à medida em que ela é submetida a diferentes demandas e implantada em diferentes configurações, porém, a qualidade da avaliação de desempenho dependerá da qualidade dos cenários de testes que os usuários declararem, uma vez que o ambiente não decide qual será a nova configuração testada. O ambiente apenas segue aquilo que foi declarado pelo usuário.

\subsection{Cloud WorkBench}
Uma vez que a escolha da infraestrutura computacional ótima para hospedar uma determinada aplicação na nuvem se trata de uma tarefa que ``exige a avaliação de custos e performances de diferentes combinações de configurações''~\cite{scheuner2014cloud}. Onde os autores propõem uma arquitetura e uma implementação concreta dessa arquitetura para automatizar a realização de avaliações em serviços da nuvem. O {\em Cloud WorkBench}, nome dado à solução apresentada neste trabalho, adota noções de Infraestrutura como Código, do inglês {\em Infrastructure-as-Code} (IaC)~\cite{huttermann2012devops}, para a realização dessas avalições. Dessa forma, as ações necessárias para o provisionamento dos recursos utilizados pela aplicação encontram-se todas codificadas.

Para ilustrar o uso do \textit{Cloud WorkBench}, foi realizado um pequeno experimento para avaliar a velocidade de escrita sequencial em disco de recursos da nuvem. Nesse experimento, foram utilizados três perfis de recursos computacionais da Amazon EC2 na região Irlanda (\textit{eu-west-1}), em servidores utilizando o sistema operacional Ubuntu 14.04. Para cada perfil de recurso, foram realizadas entre 8 e 12 execuções do benchmark FIO\footnote{http://git.kernel.org/cgit/linux/kernel/git/axboe/fio.git} 2.1.10. Conforme descrito em~\cite{scheuner2014cloud}, o experimento evidenciou que há diferenças na performance dos perfis de recursos utilizados. Esse diferença poderia se refletir na performance de uma aplicação que fizesse muita escrita em disco.

Após a análise dos resultados, os autores concluem que o \textit{Cloud WorkBench} suporta a realização de experimentos em nuvens de infraestrutura, e que toda a complexidade da configuração do ambiente pode ficar codificada. O que diminui a ocorrência de erros decorrentes de eventuais intervenções manuais.

\section{Análise Crítica}

\subsection{Abordagem Preditiva}
As soluções de abordagem preditiva possuem desempenho bastante variados nos
critérios de desempenho estabelecidos no início deste capítulo, porém elas
acabam convergindo na efetividade, que comproente eficiência, acurácia e
complexidade. No critério de eficência, as soluções se destacam, possuem
alta eficiência, uma vez que não requerem a alocação de recursos de nuvem para
realizarem as predições. No entanto, o \textit{CloudProphet} é a única solução
de abordagem preditiva com baixa eficiência, pois requer a alocações de recursos da nuvem para
avaliar todas as demandas e configurações. Já com relação à acurácia, as
soluções tem um desempenho moderado, com distinção para a solução apresentada
em~\cite{fittkau2012cdosim}, que possui baixa acurácia, conforme fica
evidenciado nos resultados apresentados na subseção~\ref{subsec:CDOSim}.
Finalmente, no que diz respeito à complexidade, a solução com menor
complexidade, portanto com maior destaque, é a \textit{CloudHarmony} a qual
permite que os testes sejam iniciados e que as pesquisas de resultados
anteriores sejam realizadas através de uma interface amigável, sem a necessidade de intervenções do usuário.
Já as demais possuem complexidade moderada.
 
Ainda com base nos critérios de avaliação, as soluções apresentadas
em~\cite{malkowski2010cloudxplor,cloudharmony} possuem baixa completude, pois
não permitem que sejam definidos aplicação, demanda, recurso da nuvem e SLA. Já
o \textit{CloudAdvisor} apresentando em~\cite{jung2013cloudadvisor}, permite a
definição da aplicação, e da carga, porém não permite a escolha do SLA e não
faz referência a respeito do uso de nuvens públicas diferentes das apresentadas
nos resultados. Dessa forma, com relação à completude, o \textit{CloudAdvisor}
tem um desempenho moderado. Das abordagens preditivas, o \textit{CloudProphet}
é a solução que se destaque no critério da completude, pois permite que o
usuário defina aplicação, demanda, recurso da nuvem e SLA desejado.

\subsection{Abordagem Empírica}
Apesar das soluções de abordagem empírica terem destaque no critério da completude, no que diz respeito à efetividade, elas possuem desempenho moderado nos critérios relacionados à eficiência e à complexidade, enquanto que possuem alta acurácia. No que diz respeito à eficiência, essas ferramentas poderiam fazer uso de resultados anteriores para evitar a execução de testes que claramente poderiam ser evitados, por exemplo, em uma situação em uma demanda é submetida à aplicação que está sendo executada em uma máquina virtual com o mais baixo nível de recurso computacional, seria coerente afirmar que essa mesma demanda pode ser atendida por máquinas com perfis computacionais mais robustos. Já no que diz respeito à complexidade, cada trabalho faz uso de uma estratégia particular para a definição da aplicação. Dessa forma, o usuário do ambiente precisa se adaptar à sintaxe de cada solução, e eventualmente, configurar imagens contendo os componentes da aplicação que será avaliada.

%\section{Outros Trabalhos e Considerações}

%******

%\textbf{Caracteristicas gerais do Cloud Capacitor}

%Foco em minimização dos custos de execução dos testes

%Proposta de análise do impacto da escalabilidade horizontal e/ou vertical nos
% resultados dos testes

%Julgamento de qual tipo de máquina apresenta melhor custo x desempenho para
% cadafaixa de workload

%Flexibilidade de implementação da lógica de avaliação através do arcabouço
% deheurísticas

%******


\section{Fundamentação Teórica}
% ----------------------------------------------------------
Esta seção apresenta alguns fundamentos básicos ligados à Computação em Nuvem
(\emph{Cloud Computing}) a fim de nivelar o conhecimento do leitor a respeito
das tecnologias envolvidas bem como terminologias e conceitos utilizados neste
trabalho.

\subsection{Computação em Nuvem}
Embora não haja uma conceituação padronizada, a análise da literatura disponível 
em torno da computação em nuvem permite caracterizá-la como um modelo de computação 
distribuída em que usuários rapidamente provisionam e liberam sob demanda recursos 
computacionais virtualizados configuráveis de maneira escalável e elástica através 
de um serviço provido pela rede ou Internet 
\cite{foster2009cloud,cearley2010case,mell2011nist}. 

A partir desse texto é possível destacar três ideias chave: \textbf{escalável} --- 
relacionada à percepção de recursos ilimitados; \textbf{serviço} --- similar ao 
fornecimento de serviços utilitários como energia elétrica ou gás, que são pagos 
apenas quando utilizados; \textbf{rede e Internet} --- relativa ao meio através do qual o 
consumidor terá acesso ao serviço; \textbf{rápido provisionamento e a liberação de recursos}
 --- como forma de garantir escalabilidade, e o acesso aos recursos através da rede, não
necessariamente a Internet.

Em~\cite{vaquero2008break}, os autores discutem o paradigma da computação em
nuvem através da identificação e estudo de 20 definições disponíveis na literatura, 
para então extrair uma definição mais abrangente, contendo as características comuns 
a várias das definições analisadas:

\begin{citacao}
A computação em nuvem ``consiste em um grande \emph{pool} de recursos virtualizados 
(tais como hardware, plataformas de desenvolvimento e/ou serviços) facilmente 
utilizáveis e acessíveis. Esses recursos podem ser dinamicamente reconfigurados 
para se ajustarem a uma carga variável, permitindo também a sua utilização ótima. 
Esse \emph{pool} de recursos é tipicamente explorado em um modelo de pagar-pelo-uso, 
no qual o provedor de infraestrutura oferece garantias por meio de acordos de nível 
de serviço (SLAs) customizados.''\end{citacao}

Essa última definição menciona o termo \textbf{acordo de nível de serviço}, uma 
característica particularmente importante para aplicações sensíveis a flutuações 
de desempenho. Outra característica mencionada é o modelo de \textbf{pagar-pelo-uso}, 
o que permite que uma aplicação se adapte a variações na demanda com a adição de 
novos recursos ou a liberação de recursos existentes, restringindo seus gastos 
com a nuvem apenas a recursos que de fato necessita e utiliza.

\subsubsection{Modelos de Implantação}
O serviço de disponibilização de recursos computacionais ao usuário, conforme
carcaterizado na seção anterior, pode ser implantado de maneiras diferentes,
a depender da sua localização em relação à entidade corporativa representada pelo
usuário e ao meio pelo qual esse usuário tem acesso a esse 
serviço~\cite{armbrust2009above,Zhang2010}.
 
\begin{description}
\item[Nuvem privada] \hfill \\ É construída para ser utilizada por uma única 
organização e, geralmente, a alocação de seus recursos se dá por uma rede interna. 
Esse modelo de nuvem oferece todos os benefícios normalmente atribuídos à computação 
em nuvem, como recursos virtualizados e escalabilidade das aplicações. A exceção 
é o seu alto custo de manutenção, já que a própria organização é responsável pelo 
gerenciamento da infraestrutura física. Atualmente há diversas soluções que permitem 
a criação e o gerenciamento de uma nuvem privada dentro de uma organização, com 
destaque para soluções de código aberto como OpenStack~\cite{openstack}, 
Eucalyptus~\cite{eucalyptus} e Open-Nebula~\cite{opennebula}.
\item[Nuvem pública] \hfill \\ A nuvens públicas constituem o modelo de nuvem 
mais conhecido, e disponibilizam recursos de nuvem para o público em geral. Esses 
recursos normalmente estão disponíveis através da Internet e são gerenciados por 
uma organização (provedor da nuvem), que pode cobrar pela sua utilização. 
\item[Nuvem híbrida] \hfill \\ No modelo de implantação híbrido, recursos de uma 
nuvem pública são alocados em conjunto com os de uma nuvem privada. Um exemplo de 
aplicabilidade desse modelo é o caso dos recursos da nuvem privada terem se exaurido,
possivelmente num momento de pico de utilização de uma aplicação. Neste caso, 
novos recursos podem ser alocados a partir de uma nuvem pública. O modelo de 
nuvem híbrida é implementado em algumas soluções como OpenNebula~\cite{opennebula}, 
que permite a integração de recursos de uma nuvem privada com recursos da nuvem 
da Amazon, e Righscale~\cite{rightscale}, que oferece um serviço de nuvem híbrida 
comercial.
\end{description}

Este trabalho tem seu foco totalmente voltado ao planejamento de capacidade em
nuvens públicas e, portanto, este foi o modelo de implantação adotado na execução
dos experimentos descritos no Capítulo~\ref{chap:resultados}. 

\subsubsection{Tipos de Serviço de Nuvem}
Outra forma de classificar as soluções de computação em nuvem é quanto aos tipos 
dos serviços oferecidos. Os três tipos de serviços de nuvem mais comumente citados 
são: infraestrutura como serviço (IaaS), plataforma como serviço (PaaS), e 
software como serviço (SaaS)~\cite{armbrust2010view,Zhang2010}. 

\begin{description}
\item[SaaS - Software as a Service] \hfill \\ Caracterizam-se por ter como cliente o
próprio usuário final do serviço publicado, que normalmente consiste em uma
aplicação hospedada na nuvem. Nas nuvens SaaS, é o provedor da nuvem quem se
responsabiliza pelo desenvolvimento e atualização dos serviços (aplicações) 
fornecidos aos usuários. Estão cada vez mais presentes no dia a dia dos usuários 
de computadores. Os exemplos mais comuns são serviços de email, discos virtuais, 
CRM e editores online, que muitas vezes são ofertados sem custo inicial para seus 
usuários. No caso do email, um dos serviços mais conhecidos é o GMail, que oferece 
uma determinada capacidade de armazenamento gratuitamente, com o usuário tendo a 
opção de pagar para estender essa capacidade inicial caso necessite de mais espaço. 
Serviços de discos virtuais, como Dropbox e GoogleDrive, também seguem uma política 
de preços similar. Já serviços de CRM - \emph{Customer Relationship Management}, 
como \cite{salesforce} e o \cite{zoho}, oferecem um período de teste gratuito 
para o usuário, após o qual o serviço passa a ser pago. O \cite{github} é um 
serviço gratuito de controle de versionamento de arquivos para repositórios 
públicos. Entretanto, o usuário tem a opção de pagar pelo serviço para ter direito 
a repositórios privados.
\item[PaaS - Platform as a Service] \hfill \\ Nas nuvens PaaS o desenvolvimento 
das aplicações fica sob responsabilidade dos clientes, cabendo ao provedor da 
nuvem oferecer uma plataforma de execução apropriada ao cliente. Nesse tipo de 
nuvem, o provedor é responsável por manter o ambiente operacional, garantindo a 
disponibilidade dos serviços e realizando os ajustes necessários para que as 
aplicações desenvolvidas pelos clientes atendam à demanda de seus usuários. 
Dois dos provedores de nuvem PaaS mais conhecidos são Microsoft e Google, que 
oferecem os serviços Windows Azure~\cite{azure} e Google AppEngine~\cite{appengine}, 
respectivamente. No Windows Azure, o cliente da nuvem tem acesso às plataformas 
.NET, Node.js, Java, PHP, Python e Ruby, além das plataformas móveis iOS, Android
e Windows Phone. Já no Google App Engine o desenvolvedor pode escolher entre 
as linguagens Java, Python, PHP e Go. No entanto, uma aplicação Java ou Phyton 
já existente dificilmente poderia ser portada para um desses serviços sem sofrer 
modificações, pois em abos há restrições ou a necessidade de configurações específicas
em suas APIs quando comparadas às plataformas tradicionais.
\item[IaaS - Infrastructure as a Service] \hfill \\ Nuvens to tipo IaaS oferecem 
aos clientes recursos básicos de infraestrutura, como processamento, armazenamento 
e rede. Nessa modalidade de serviço, o usuário tem à sua disposição um conjunto
de recursos virtuais, geralmente agrupados por sua capacidade ou finalidade, para
que escolha e monte a configuração que mais se adeque às necessidade de desempenho
da sua aplicação. Entre esses recursos estão máquinas virtuais, discos,
CPUs de voltadas a processamento gráfico e redes de alta velocidade. Comparado 
aos outros dois tipos de serviço de nuvem, os serviços do tipo IaaS oferecem muito 
mais flexibilidade de configuração pelos clientes, que assim podem implantar e 
executar na nuvem as aplicações de sua escolha, inclusive os próprios sistemas 
corporativos com pouca o nenhuma modificação. Alguns dos provedores de nuvens 
IaaS mais conhecidos são Amazon, com seu serviço EC2 - Elastic Compute 
Cloud~\cite{ec2}, e Rackspace~\cite{rackspace}.
\end{description}

Uma vez que o objetivo deste trabalho está voltado para o planejamento de capacidade
de recursos de nuvens, a modalidade de serviço utilizada é a IaaS. Os estudos 
desenvolvidos focam no teste de desempenho de uma aplicação implantada em um
provedor de nuvem de infraestrutura, onde o objetivo é identificar quais dentre
os recursos disponibilizados são capazes de executar a aplicação obdedecendo ao 
acordo de nível de serviço estabelecido. Assim, apresentamos a seguir uma breve
descrição dos recursos oferecidos em um provedor de nuvem de infraestrutura. 

\subsubsection{Recursos de IaaS}
A fim de esclarecer melhor quais recursos um usuário de nuvem IaaS tem a seu
dispor, esta seção mostra algumas das oções disponibilizadas pelo serviço EC2
da Amazon. A descrição está focada neste provedor por ter sido o escolhido para
a execução dos testes de desempenho que dão suporte aos experimentos realizados
neste trabalho.

O serviço EC2 caracteriza-se principalmente pelo fornecimento de máquinas virtuais
que o usuário pode criar, ligar e desligar livremente, pagando pelo tempo de
utilização, ou seja, pelo tempo em que as máquinas estiverem ligadas. Essas 
máquinas são instanciadas a partir de configurações cujo poder computacional é 
pré determinado pelo provedor, mas especificamente, quantidade de memória, 
quantidade de núcleos virtuais de CPU e poder de processamento total da CPU.

Os tipos de máquinas virtuais do serviço EC2 são categorizados segundo sua 
finalidade e segundo seus atributos de hardware. Há categorias para máquinas
que priorizam mais memória em detrimento de poder computacional, outras para 
máquinas cujo característica é o grande poder de processamento e até categorias
específicas de máquinas voltadas ao processamento gráfico ou em \emph{clusters}.
A Tabela~\ref{table:maquinas_ec2} mostra uma parte dos tipos de máquinas virtuais
oferecidos pelo serviço EC2 da Amazon, com os preços cobrados por hora e suas
características de hardware.

\begin{table}
  \centering
  \begin{tabular}{|c|c|c|c|c|}
    \hline
%    \multirow{2}{*}{Tipos de Máqunas} & Núcleos & ECU\footnotemark & \parbox[m]{2cm}{\center{Memória\\(GiB)}} & US\$ / hora\\
    Tipos de Máquinas & Núcleos & ECU\footnotemark & \parbox[m]{2cm}{\center{Memória\\(GiB)}} & US\$ / hora\\
%    \cline{2-5}
    \hline    
      \multicolumn{5}{|c|}{Propósito Geral} \\
      \hline      
      t2.micro  & 1 & Variável & 1 & 0.013 \\
      t2.small  & 1 & Variável & 2 & 0.026 \\
      t2.medium & 2 & Variável & 4 & 0.052 \\
      m3.medium & 1 & 3 & 3.75 & 0.070 \\
      m3.large  & 2 & 6.5 & 7.5 & 0.140 \\
      m3.xlarge & 4 & 13  & 15  & 0.280 \\
      m3.2xlarge & 8 & 26 & 30  & 0.560 \\
      \hline      
      \multicolumn{5}{|c|}{Otimizadas para Computação} \\
      \hline      
      c3.large  & 2 & 7 & 3.75 & 0.105 \\
      c3.xlarge & 4 & 14 & 7.5 & 0.210 \\
      c3.2xlarge & 8 & 28 & 15 & 0.420 \\
      c3.4xlarge & 16 & 55 & 30 & 0.840 \\
      c3.8xlarge & 32 & 108 & 60 & 1.680 \\
      \hline      
      \multicolumn{5}{|c|}{Otimizadas para Processamento Gráfico (GPU)} \\
      \hline      
      g2.2xlarge & 8 & 26 & 15 & 0.650 \\
      \hline      
      \multicolumn{5}{|c|}{Otimizadas para Memória} \\
      \hline      
      r3.large & 2 & 6.5 & 15 & 0.175 \\
      r3.xlarge & 4 & 13 & 30.5 & 0.350 \\
      r3.2xlarge & 8 & 26 & 61 & 0.700 \\
      r3.4xlarge & 16 & 52 & 122 & 1.400 \\
      r3.8xlarge & 32 & 104 & 244 & 2.800 \\
    \hline    
  \end{tabular}
  \caption{\label{table:maquinas_ec2}Relação parcial de preços do EC2 na regiao US East em Dezembro de 2014}
\end{table}

\footnotetext{\emph{ECU - EC2 Compute Unit} é uma medida própria criada pela Amazon 
para relativizar o poder de processamento de suas máquinas para permitir uma 
comparação entre os vários tipos de máquinas oferecidos.}

Cada máquina pode ser individualmente configurada com opções diferentes de armazenamento
em massa. Há opções de armazenamento volátil, armazenamento em volumes magnéticos
comuns ou discos de estado sólido - \emph{SSDs}. Essas opções, embora não alterem
a categorização da máquina, alteram o preço final da hora de utilização, acarretando
um custo adicional ao valor base da hora da configuração padrão da máquina. Além
disso, a utilização de alguns recursos acarreta mais um custo adicional, como o
tráfego de dados pela rede. A Amazon cobra uma tarifa por \emph{gigabyte} trafegado
para fora da rede local do provedor. Isso significa que mesmo o tráfego de rede 
entre centros de dados diferentes pode ser cobrado.

Os preços cobrados pela Amazon para a utilização de uma máquina virtual variam 
conforme o centro de dados escolhido pelo usuário no momento da criação da máquina.
Essa variação ocorre tanto para o preço base das máquinas como para recursos adicionais
e consumíveis, como tráfego de dados.

A Amazon possui centros de dados em todos os continentes, oferecendo redundância
e buscando minimizar problemas com latência de rede. Atualmente existem 3 centros de
dados nos Estados Unidos, sendo 1 em Virginia, 1 no Oregon e 1 na Califórnia; 2 
centros de dados na Europa, sendo 1 em Frankfurt e 1 na Irlanda; 3 centros na Ásia,
com 1 em Cingapura, 1 em Tóquio e 1 em Sydney; e 1 centro de dados na América do
Sul, em São Paulo. Por motivos de custo, para os testes executados nos experimentos 
deste trabalho foi utilizado o centro de dados da região US East (Virginia), o
mais barato entre todos, inclusive mais barato que o centro de dados no Brasil.   

% ----------------------------------------------------------

% ----------------------------------------------------------

% ----------------------------------------------------------
% Formalização
% ----------------------------------------------------------
%\chapter[Formalização da Solução]{Formalização da Solução}
% ----------------------------------------------------------
A fim de que uma análise criteriosa possa ser feita, apresentamos a seguir um 
conjunto de definições e terminologias que baseiam o entendimento da 
construção do trabalho e também a avaliação dos resultados, bem como a eficiência 
e eficácia da solução proposta. Os conceitos aqui explicitados são tomados de 
forma a permitir um estudo agnóstico quanto a aplicações, plataformas e provedores
utilizados durante a execução das ferramentas desenvolvidas neste trabalho.

Apresentamos também um formalismo que visa à generalização da metodologia 
utilizada neste trabalho e a permitir a melhor descrição do raciocínio lógico 
envolvido no desenvolvimento das heurísticas criadas.

\section{Definições e Terminologias}
Apresentamos a seguir as definições que permeiam o conhecimento necessário para 
a análise dos problemas estudados e soluções propostas. Mostramos também as 
terminologias ou nomenclaturas que criamos para designar esses conceitos a fim 
de facilitar a comunicação e o entendimento por parte do leitor.

\subsection{Aplicação sob Teste}
A Aplicação sob Teste é um sistema computacional, possivelmente implementado em 
arquitetura multicamadas, para o qual se deseja observar o comportamento em um 
ambiente de computação em nuvem e ao qual estão ligadas uma ou mais Métricas de 
Desempenho.

\subsection{Métrica de Desempenho}
Uma característica ou comportamento mensurável de forma automatizada e 
comparável a um Valor de Referência capaz de indicar o grau de sucesso de uma 
execução da Aplicação. Ex. tempo de resposta, quadros por segundo, etc. Métricas 
podem ser minimizáveis ou maximizáveis, a depender do objetivo da métrica quanto 
ao resultado desejado. Por exemplo, “tempo de resposta” é uma métrica 
minimizável, uma vez que geralmente se deseja que uma Aplicação responda a uma 
requisição com o menor tempo de resposta possível nos resultados. Contrariamente, 
uma métrica “quadros renderizados por segundo”, no domínio da computação gráfica, 
é uma métrica maximizável, pois quanto mais quadros são renderizados por unidade 
de tempo, maior a qualidade percebida pelo usuário.

\subsection{Valor de Referência ou SLA}
Um valor predefinido como minimamente aceitável como resultado apresentado por 
uma Métrica após uma Execução da Aplicação sob Teste. Este valor, também 
referenciado neste trabalho como SLA (Service Level Agreement), serve como base 
de comparação para que as Heurísticas desenvolvidas e apresentadas saibam se a 
Aplicação é capaz de ser executada sobre um determinado arranjo de máquinas 
virtuais e sob uma determinada Carga de Trabalho a ela imposta.

\subsection{Provedor}
Consideramos neste trabalho a figura do provedor como representando uma empresa 
que fornece infraestrutura computacional como serviço cobrado financeiramente 
por fração de tempo de utilização. Alguns provedores fornecem conjuntamente a 
modalidade de plataforma como serviço. Nós, porém, não estamos considerando essa 
modalidade neste trabalho, interessando-nos apenas os serviços de infraestrutura, 
notadamente a disponibilização de máquinas virtuais.

\subsection{Tipos de Máquinas Virtuais}
Provedores costumam classificar as máquinas virtuais fornecidas conforme suas 
características, de modo a manter uma linha de produtos discreta e finita. 
Normalmente essa classificação se dá em termos de quantidade de memória RAM, 
quantidade de espaço em disco e capacidade computacional, neste caso, quer seja 
em termos relativos a um valor padrão tomado como base, quer seja em termos 
absolutos, como número de CPUs virtuais.

\subsection{Categorias de Máquinas Virtuais}
Tipos de Máquinas Virtuais podem ser agrupados em Categorias, conforme suas 
características físicas, plataforma e/ou arquitetura de hardware e a natureza do 
uso a que se destinam. Dentro de uma mesma Categoria, os Tipos de Máquinas 
Virtuais variam apenas na quantidade de cada um dos recursos especificados para 
a Categoria e no preço cobrado pelo uso das máquinas virtuais.

Como exemplo, podemos citar uma Categoria de máquinas destinadas a armazenamento 
de arquivos, onde as máquinas devem privilegiar o espaço de armazenamento em 
massa. Dentro dessa categoria, a principal diferença entre os Tipos de Máquinas 
Virtuais se dá em função da quantidade de espaço em disco disponibilizado, 
enquanto características como memória RAM e CPU teriam pequenas variações. 
Outras Categorias podem enfatizar o consumo de banda de rede ou processamento 
paralelo de alto desempenho.

\subsection{Configurações}
Chamamos de Configuração um conjunto de máquinas virtuais pertencentes ao mesmo 
Tipo de Máquinas Virtuais e, portanto, de uma mesma Categoria. Configurações são
usadas para implantar uma camada arquitetural da Aplicação sob Teste (apresentação, 
negócio, persistência, etc.) e representam o estado de uma determinada camada da 
aplicação quanto à sua escalabilidade, vertical ou horizontal.

Por exemplo, suponhamos avaliação do comportamento de uma Aplicação cuja camada de
negócios está implantada em arquitetura de cluster de servidores de aplicação. 
Variando a quantidade de máquinas que compõem esse cluster, obtemos diferentes níveis
de escalabilidade horizontal para os quais podemos avaliar o desempenho da Aplicação.
Agora, suponhamos que podemos usar uma, duas, três ou quatro máquinas em paralelo como
componentes do cluster da camada de negócios da Aplicação sob Teste. Assim, teríamos
então quatro Configurações diferentes, a primeira com uma instância na camada de negócio,
a segunda Configuração com duas instâncias, a terceira com três e a quarta com quatro 
instâncias de máquinas do mesmo Tipo de Máquina Virtual. A Aplicação sob Teste seria 
executada quatro vezes, cada uma das quais utilizando uma dessas Configurações. Os 
resultados dessas Execuções refletem o efeito da escalabilidade horizontal no desempenho
geral da Aplicação.

Analogamente, poderiam ser usadas Configurações criadas a partir de Tipos de Máquinas
Virtuais diferentes, umas mais potentes que as outras. A comparação dos resultados obtidos
nesse cenário nos dão insumos para avaliar o efeito da escalabilidade vertical sobre o
desempenho da Aplicação.

As heurísticas desenvolvidas e apresentadas por este trabalho comparam 
implantações de diferentes Configurações em uma determinada camada da Aplicação 
sob Teste estudada. Isso permite que sejam feitas avaliações como a viabilidade 
financeira da escalabilidade vertical face ao desempenho possivelmente obtido 
com a escalabilidade horizontal.

\subsection{Espaço de Implantação}
Chamamos de Espaço de Implantação o conjunto limitado de Configurações tomadas 
para execução da Aplicação sob Teste em uma sessão de avaliação.
 
Idealmente, uma Aplicação deveria ser testada sob todos os Tipos de Máquinas 
Virtuais fornecidos pelo Provedor (cobrindo todo o espaço de escalabilidade 
vertical) com o maior número possível de combinações de quantidade de instâncias 
(cobrindo o espaço de escalabilidade horizontal). Porém, se muitos Tipos de 
Máquinas Virtuais forem necessários e se o intervalo de número de instâncias 
solicitado for muito grande, o tempo de duração da sessão e o custo da muitas 
execuções podem se tornar proibitivos.

Assim, o processo de especificação de um Espaço de Implantação consiste em 
selecionar uma lista de Tipos de Máquinas Virtuais entre os oferecidos pelo 
Provedor e designar o melhor valor para o número máximo de instâncias que serão 
usadas na criação das Configurações. Isso faz com que ambos os espaços de 
escalabilidade vertical e horizontal sejam limitados, de forma a controlar 
melhor os custos e permitir que sejam executados testes mais objetivos e de 
acordo com a meta de Carga de Trabalho a ser atendida pela Aplicação.

\subsection{Carga de Trabalho}
A Carga de Trabalho, ou Workload, representa o tamanho da demanda que será 
imposta à Aplicação sob Teste em uma execução. A unidade de medida da Carga é 
dependente do domínio da Aplicação, como a duração do vídeo em uma aplicação
de transcodificação de arquivos multimídia ou o tamanho do arquivos de entrada
para uma aplicação de compactação de arquivos. Entretanto, para efeito deste 
trabalho, essa unidade de medida da Carga é irrelevante, uma vez que a 
responsabilidade pela execução dos testes e, por conseguinte, pela geração da 
carga, é delegada a um módulo à parte dentro do sistema de avaliação, como um
software de \emph{benchmarking}.

\subsection{Execução}
Damos o nome de Execução ao evento de utilização de uma Configuração para 
executar a Aplicação sob Teste submetida a uma determinada Carga de Trabalho. 
Dessa forma, a avaliação dos Resultados de uma Execução nos dará uma ideia de 
como a Aplicação responderá às requisições de certo número de usuários (Workload) 
após ser implantada num ambiente de nuvem com certo grau de escalabilidade 
horizontal (número de máquinas virtuais usadas). 

Tomemos como exemplo uma aplicação web muito comum, um blog. São requisições 
comuns a um blog o acesso à página principal, uma consulta às postagens de uma
categoria e o acesso a uma postagem específica. Agora suponhamos uma Configuração
composta de três instâncias de máquinas virtuais do Provedor Rackspace, do Tipo 
"Performance 1", executando uma instalação do blog Wordpress, muito usado hoje em 
dia. Uma Execução dessa aplicação seria a imposição da Carga de Trabalho correspondente
a um conjunto de requisições disparadas por 100 usuários simultâneos sobre essa Configuração. 

\section{Formalismos}
Dada uma Aplicação sob Teste A, precisamos identificar, dentre um conjunto de 
cenários de implantação e execução da aplicação em ambiente de nuvem 
computacional sob a modalidade de infraestrutura como serviço, sob quais desses 
cenários a aplicação é executada com sucesso e, dentre esses cenários, quais os 
de menor custo.

De modo a facilitar uma descrição formal do raciocínio lógico empregado na
solução proposta, mostramos a seguir a notação que representa os conceitos 
descritos na seção anterior. Ao final, será apresentada uma descrição formal
do modelo vislumbrado como solução para o problema apresentado.

\subsection{Métricas de Desempenho}
Seja $M(A)$ um conjunto de Métricas de Desempenho de interesse da Aplicação sob 
Teste $A$. Seja $|M(A)|$ a quantidade de Métricas de Desempenho $\in M(A)$ avaliadas 
para a Aplicação sob Teste $A$.

$M(A) = \{m_1, m_2, \dotsc, m_{|P|}\}\; | \; m_n \in M, 1 \leq n \leq |M| \iff m_n $ 
é uma Métrica de Desempenho a ser avaliada para a Aplicação sob Teste $A$.

\subsection{Provedor}
Seja $P$ um conjunto de Provedores. Seja $|P|$ a quantidade de Provedores 
pertencentes a $P$.

$P = \{p_1, p_2, \dotsc, p_{|P|}\}\; | \; p_n \in P, 1 \leq n \leq |P| \iff p_n $ 
é um Provedor de infraestrutura como serviço.

\subsection{Tipos de Máquinas Virtuais}
Seja $T(P)$ um conjunto de Tipos de Máquinas Virtuais fornecidos pelo provedor $P$.

Seja $|T(P)|$ a quantidade de Tipos de Máquinas Virtuais pertencentes a $T(P)$.

$T(P) = \{t_1, t_2, \dotsc, t_{|T(P)|}\}\; | \; t_n \in T(P), 1 \leq n \leq |T(P)|
 \iff t_n $ é um Tipo de Máquina Virtual fornecido pelo provedor P.

Definimos $i(t)$ como uma instância de Máquina Virtual do Tipo $t \in T(P)$. 

Definimos $t(i)$ como o Tipo de Máquina Virtual $t \in T(P)$ usado para criar a 
instância $i$.

Definimos \$$(t)$ como o valor de custo unitário por hora de utilização de uma 
instância do Tipo $t$.

Definimos $cpu(t)$ como o tamanho da capacidade computacional do Tipo $t$.

Definimos $mem(t)$ como o tamanho da capacidade de memória do Tipo $t$.

Dados dois Tipos de Máquinas Virtuais $t_1$ e $t_2$:

$t_1 = t_2 \iff cpu(t_1) = cpu(t_2) \land mem(t_1) = mem(t_2)$.

$t_1 < t_2 \iff cpu(t_1) < cpu(t_2) \land mem(t_1) < mem(t_2)$.

$t_1 > t_2 \iff cpu(t_1) > cpu(t_2) \land mem(t_1) > mem(t_2)$.

$cpu(t_1) < cpu(t_2) \land mem(t_1) > mem(t_2) \iff$ relação indeterminada entre $t_1$ e $t_2$.

$cpu(t_1) > cpu(t_2) \land mem(t_1) < mem(t_2) \iff$ relação indeterminada entre $t_1$ e $t_2$.

\subsection{Categorias de Máquinas Virtuais}
Seja $Cat(P)$ um conjunto de Categorias de Máquinas Virtuais fornecidas pelo 
provedor $P$.

$Cat(P) = \{cat_1, cat_2, \dotsc, cat_{|Cat(P)|}\}\; | \; cat_n \in Cat(P), 
1 \leq n \leq |Cat(P)| \iff cat_n $ é uma Categoria de Máquina Virtual fornecida 
pelo provedor P.

$cat_n = \{T(P)_1, T(P)_2, \dotsc\, T(P)_{|cat_n|}\}\; | \; T(P)_m \subset T(P), 
1 \leq m \leq |cat_n| \iff T(P)_m $ é um grupo de Tipos de Máquina Virtual
fornecidos pelo provedor P.

Definimos $cat(t_n) \in Cat(P)$ como a Categoria atribuída ao Tipo $t_n \in T(P)$ 
pelo Provedor $P$.

\subsection{Configurações}
Seja $c$ um grupo com um certo número de Máquinas Virtuais $i(t)$.

Dizemos que $c$ é uma Configuração se $\forall \; i(t) \in c, t$ é constante.

Definimos $t(c)$ como o Tipo de Máquina Virtual usado para criar as instâncias 
que compõem $c$. Portanto, temos que $t(c) = t(i), \forall \; i(t) \in c$. 

Definimos $\#(c)$ como o tamanho da Configuração $c$, ou seja, a quantidade de 
instâncias $i(t)$ usadas para formar $c$.

Definimos \$$(c)$ como o custo total da configuração, dado por \$$(t) \times \#(c)$.

O conjunto $C(P)$ é o conjunto de todas as Configurações $c$ possíveis de serem 
formadas a partir dos Tipos $t \in T(P)$ e com um número arbitrário de instâncias 
$i(t)$.

Dadas duas Configurações $c_1$ e $c_2$:

$c_1 = c_2 \iff t(c_1) = t(c_2) \land \#(c_1) = \#(c_2)$.

$ c_1 > c_2 \iff \left\{
  \begin{array}{l l}
    t(c_1) = t(c_2) \land \#(c_1) > \#(c_2)\\
    ou\\
    t(c_1) > t(c_2) \land \#(c_1) = \#(c_2)
  \end{array} \right.$
  
$ c_1 < c_2 \iff \left\{
  \begin{array}{l l}
    t(c_1) = t(c_2) \land \#(c_1) < \#(c_2)\\
    ou\\
    t(c_1) < t(c_2) \land \#(c_1) = \#(c_2)
  \end{array} \right.$

\subsection{Espaço de Implantação}
Seja um conjunto de Tipos de Máquinas Virtuais $T(DS_x^y) \subset T(P)$.

Seja um conjunto de Configurações $C_x^y$ tal que:
 
$\forall \; t \in T(DS_x^y), \; \forall \; i \; | \; x \leq i \leq y, \; \exists \; c \in C_x^y \; | \; t(c) = t \; \land \; \#(c) = i$.
 
O Espaço de Implantação $DS_x^y(P)$ é um grafo cujo conjunto de vértices é dado 
por $C_x^y$ e cujas arestas são dadas pelas seguintes operações:

\begin{enumerate}
  \item Ordenar $C_x^y$ pelo atributo $preço$ de cada elemento do conjunto (\$$(c)$).
  \item $preço_{\to} = \forall\; c_1, c_2 \in C_x^y,\; c_1 \to c_2 \iff \$(c_1) < \$(c_2) \land \nexists\; c_3\;|\; \$(c_1) < \$(c_3) < \$(c_2)$
  \item Ordenar $C_x^y$ pelo atributo $CPU$ de cada elemento do conjunto ($cpu(c)$).
  \item $cpu_{\to} = \forall\; c_1, c_2 \in C_x^y,\; c_1 \to c_2 \iff cpu(c_1) < cpu(c_2) \land \nexists\; c_3\;|\; cpu(c_1) < cpu(c_3) < cpu(c_2)$
  \item Ordenar $C_x^y$ pelo atributo $Memória$ de cada elemento do conjunto ($mem(c)$).
  \item $mem_{\to} = \forall\; c_1, c_2 \in C_x^y,\; c_1 \to c_2 \iff mem(c_1) < mem(c_2) \land \nexists\; c_3\;|\; mem(c_1) < mem(c_3) < mem(c_2)$
\end{enumerate}

\subsection{Cargas de Trabalho}
Seja $W$ um conjunto finito de valores que representam Cargas de Trabalho. Seja 
$|W|$ a quantidade de Cargas de Trabalho pertencentes a $W$.

$W = \{w_1, w_2, \dotsc, w_{|W|}\}\; | \; w_n \in W, 1 \leq n \leq |W| \iff w_n $ 
é um valor válido como carga a ser submetida dentro do domínio de conhecimento 
da Aplicação sob Teste.

\subsection{Resultados}
Um Resultado $r$ é uma quádrupla no formato $\{m, v, cpu, mem\}$ onde:

$m \in M(A)$ é a Métrica de Desempenho medida no Resultado;

$v$ é o valor medido para a Métrica;

$cpu$ é o percentual de CPU utilizada para a obtenção de $v$; e

$mem$ é o percentual de memória RAM utilizada para a obtenção de $v$.

\subsection{Execuções}
Seja $E(A)$ um conjunto de Execuções da Aplicação sob Teste $A$. Seja $|E|$ a 
quantidade de Execuções pertencentes a $E(A)$.

$E(A) = \{e_1, e_2, \dotsc, e_{|E|}\}\; | \; e_n \in E(A), 1 \leq n \leq |E(A)| \iff e_n $ 
é uma tripla no formato $\{c, w, r\}\; |\; c \in DS_x^y(P), w \in W$ e onde $r$ é 
o Resultado obtido a partir da Execução da Aplicação na configuração $c$, 
submetida à Carga de Trabalho $w$.

\subsection{Avaliação da Execução}
Seja $\alpha$ um valor de referência definido como parâmetro de sucesso para uma 
Métrica de Desempenho quando da Execução de um teste da Aplicação.

Seja $e \in E(A)$ uma Execução da Aplicação $A$

Seja $r \in e$ o Resultado da Execução $e$

Seja $m \in r$ a Métrica de Desempenho avaliada no Resultado $r$

Seja $v \in r$ o valor medido para a Métrica $m$ no Resultado $r$

Seja $atende(r, \alpha)$ uma função tal que:

$m\; é\; minimizável \iff atende(r,\alpha) = \left\{
  \begin{array}{l l}
    \{1, \delta\}, v \leq \alpha\\
    \{0, \delta\}, v > \alpha
  \end{array} \right.$

$m\; é\; maximizável \iff atende(r,\alpha) = \left\{
  \begin{array}{l l}
    \{0, \delta\}, v < \alpha\\
    \{1, \delta\}, v \geq \alpha
  \end{array} \right.$

Em ambos os casos, $\delta$ representa a distância entre o valor de Resultado 
obtido para a Métrica $m$ na Execução $e$ (ou seja, o valor $e[r[v]]$) e o Valor
de Referência ou SLA $\alpha$. 

A representação dessa distância é dada por $\delta = \{grau\; de\; desvio,
\; direção\; do\; desvio\}$, onde o $grau\;de\;desvio$ indica se esse desvio é 
$alto$, $médio$ ou $baixo$, e a $direção\;do\;desvio$ indica se o resultado 
obtido está $abaixo$ ou $acima$ do SLA.  

Os limites de comparação do Resultado com o SLA para a classificação do grau de 
desvio são parâmetros e podem ser modificados conforme a tolerância da Aplicação
a flutuações de desempenho do ambiente de nuvem. Os valores desses parâmetros 
são dados em termos percentuais.

As informações trazidas por $\delta$ podem influenciar o modo como as Heurísticas
de Avaliação de Capacidade caminham sobre o Espaço de Implantação na busca pelas 
melhores Configurações. Seu uso, porém, é opcional e fica a critério da própria 
Heurística.

Apresentamos a seguir uma especificação de funcionamento das Heurísticas, 
detalhando seu papel dentro deste trabalho, insumos de entrada e as operações
que devem fornecer como interface para sua manipulação e controle.

% Conforme conference call de que participaram Nabor Mendonça,
% Américo Sampaio e Marcelo Gonçalves em 05/12/2014, a secao de 
% definicoes foi movida para dentro do capitulo de 
% processo. A secao de Formalismos deste capítulo será parte da 
% Tese de % Doutorado de Matheus Cunha.
% ----------------------------------------------------------

% ----------------------------------------------------------
% Avaliação de Capacidade
% ----------------------------------------------------------
\chapter[Avaliação de Capacidade]{Avaliação de Capacidade}
% ----------------------------------------------------------
Dadas as definições e formalizações propostas no anteriormente, faz-se necessária
a especificação de uma lógica de manipulação das entidades e operações descritas.

Passamos agora a descrever uma proposta de processo de avaliação de capacidade
que visa a buscar as Configurações de menor preço capazes de executar uma determinada 
Carga de Trabalho. Esse processo foi implementado como parte de um sistema 
computacional ao qual demos o nome de \emph{Cloud Capacitor} e que descreveremos 
no capítulo seguinte.  

O processo prevê um conjunto de dados de entrada, ao menos uma execução da 
Aplicação sob Teste no ambiente de nuvem de infraestrutura almejado para 
hospedá-la, e, por fim, a análise do desempenho obtido pela Aplicação a partir 
de suas execuções. Com base nos dados de desempenho, o processo passa por diversos
pontos de decisão que podem levar a novas execuções da Aplicação em diferentes 
cenários. Ao final do processo, é fornecida como saída uma lista de Configurações, 
ordenadas por preço, capazes de executar a Aplicação sob Cargas de Trabalho específicas.

Este capítulo estuda em detalhes todas as fases do processo de avaliação de capacidade
proposto, explicando quais são os dados de entrada necessários, como é orquestrada
a execução da Aplicação no ambiente de nuvem alvo e quais as decisões pelas quais
o processo tem que passar até determinar quais são as Configurações de menor custo
capazes de executar a Aplicação.

\section{Dados de Entrada}

O principal parâmetro esperado pelo processo de avaliação de capacidade é o Valor
de Referência de Desempenho, ao qual também nos referimos como SLA 
(\emph{Service Level Agreement}). Esse valor será usadp para determinar 
se a Aplicação atingiu os requisitos mínimos de desempenho exigidos, conforme
veremos na descrição do funcionamento do processo, mais adiante.

Além do SLA, o processo precisa também conhecer quais são as Cargas de Trabalho
às quais deverá submeter a Aplicação sob Teste durante seu funcionamento. Nem todas
as Cargas de Trabalho serão submetidas de fato. Isso vai depender do conjunto de
decisões tomadas pelo processo com base na comparação do desempenho da Aplicação
com o SLA.

Para que a Aplicação seja executada, é preciso que o processo conheça quais são
as Configurações disponibilizadas no Provedor de nuvem para esse fim. Para isso,
o processo deve ser alimentado com uma lista de Tipos de Máquinas Virtuais que
serão utilizadas na execução da Aplicação, bem como a quantidade máxima de 
instâncias usadas para compor cada Configuração. Através desses dados o processo
passa a conhecer então o Espaço de Implantação disponível para os testes de 
desempenho, composto por uma lista de Configurações geradas a partir da lista de
 Tipos de Máquinas Virtuais disponíveis e do número máximo de instâncias.

  
As Estratégias devem implementar um conjunto de operações pré-definidas no processo 
que as torna capazes de se comunicar com o processo de avaliação de capacidade. Tais 
operações dão suporte ao processo nos momentos em que devem ser analisados os dados
dos resultados das execuções da Aplicação. Diferentes estratégias podem tomar 
decisões diferentes a partir dos mesmos dados de resultado.

As Heurísticas de Avaliação de Capacidade, conforme propostas neste trabalho, 
são implementações de estratégias que buscam encontrar, a partir do Resultado 
de uma Execução da Aplicação sob Teste sob uma Carga de Trabalho inicial em uma 
Configuração inicial, se existe uma Configuração mais barata que seja capaz de 
executar a mesma Carga de Trabalho.

O funcionamento de uma Heurística é totalmente baseado no caminhamento sobre o
Espaço de Implantação e sobre um a faixa de valores de Cargas de Trabalho. O 
tamanho de cada passo nesse caminho depende da lógica empregada pela Heurística 
e da avaliação do Resultado obtido. Cada passo dado pela Heuristica pode resultar 
em uma nova Execução dos testes ou em uma conclusão a respeito da capacidade da 
Configuração avaliada (e outras similares) de executar a Carga de Trabalho a um 
custo viável. Assim, a inteligência de uma Heurística está na forma como ela 
decide caminhar sobre o Espaço de Implantação e na faixa de Cargas de
Trabalho apresentados.

Apresentamos a seguir o modelo de como uma Heurística deve funcionar, a 
especificação do arcabouço de implementação construído nesse trabalho, começando
por descrever as operações que uma Heurística deve executar. 

\section{Operações Iniciais}
Para que uma Heurística de Avaliação de Capacidade seja compatível no âmbito deste trabalho, 
deve apresentar um conjunto mínimo de operações esperadas para que a lógica da
avaliação se complete e o resultado final obtido possa ser considerado válido e
comparável com os resultados obtidos por outras Heurísticas.

Além disso, as operações constituem a interface pela qual o controlador das 
sessões de avaliação pode configurar as Heurísticas e informar-lhe os dados 
necessários ao controle da sua execução.
 
Apresentamos esse conjunto mínimo de operações nas subseções a seguir, que 
representam o arcabouço necessário para a construção de uma Heurística de 
Avaliação de Capacidade.

\subsection{Selecionar Carga de Trabalho Inicial}
Este trabalho tem como premissa a necessidade de se identificar quais as 
Configurações mais baratas em um Provedor capazes de executar diversos níveis de
Cargas de Trabalho, tendo como objetivo a otimização de custos para a execução de
uma Aplicação so Teste.

Assim, pressupomos que exista uma faixa de valores para os níveis de Cargas de 
Trabalho a que a Aplicação é costumeiramente submetida e que seja de conhecimento
prévio dos responsáveis pela Aplicação. 

De posse dessa faixa de valores de Cargas de Trabalho, uma Heurística deve ser 
capaz de escolher, de acordo com sua estratégia de trabalho, um valor inicial de
Carga de Trabalho a ser imposta sobre a Aplicação. A Carga de Trabalho escolhida
deve ser retornada para o controlador da sessão, de forma que este possa coordenar
a Execução dos testes.
 
\subsection{Selecionar Configuração Inicial}
Analogamente, a fim de que as atividades da sessão de avaliação possam ter início,
é necessário que a Heurística de Avaliação de Capacidade usada selecione uma
Configuração inicial.

A escolha da Configuração inicial é feita a partir das Configurações disponíveis
no Espaço de Implantação previamente configurado pelo responsável pela avaliação.
A Heurística deverá avaliar o conjunto de configurações disponíveis quanto ao seu
preço, número de instâncias em cada, etc, de forma a escolher uma Configuração 
que considere mais adequada à sua estratégia para o início da sessão de avaliação.

\section{Operações de Controle}
Tendo em mãos uma Carga de Trabalho e uma Configuração iniciais, o controlador
da sessão de avaliação de capacidade pode ordenar uma Execução de testes, onde
serão coletados dados de desempenho relevantes para a Aplicação sob Teste.

Após a primeira Execução, um Resultado contendo os dados de desempenho colhidos 
é avaliado pelo controlador e, conforme sua decisão, novas Execuções podem se 
fazer necessárias. Neste caso, a Heurística deve ser novamente invocada, desta 
vez a selecionar uma nova Carga de Trabalho ou uma nova Configuração a partir do
Espaço de Implantação. Essa interação deve se repetir até que o controlador 
conclua os testes e dê por encerrada a sessão de avaliação.

Abaixo descrevemos as operações que a Heurística deve prover para que permita ao
controlador a correta operação dos testes e da sessão.

\subsection{Selecionar Nova Configuração}
Depois da cada execução de testes, o controlador estará de posse de um Resultado,
contendo os dados de desempenho da Aplicação executada sob a Carga de Trabalho e
a Configuração selecionadaa. A depender dos dados desse Resultado, o controlador
pode decidir executar novos testes em outra Configuração.

Para isso, a Heurística deve ser usada para selecionar a próxima Configuração a 
ser testada com a Aplicação. Com base no Resultado obtido pela Execução anterior,
a Heurística usará sua lógica de navegação para determinar a distância a ser 
caminhada no Espaço de Implantação em busca da nova Configuração.

A ordem de caminhamento é dada conforme a necessidade identificada pelo 
controlador, que vai definir se precisa de uma Configuração mais ou menos potente.
Porém, a Heurística é quem define, através de sua estratégia, qual será a próxima
Configuração usada.

Assim, a Heurística deve prover ao controlador duas operações para escolha da 
próxima Configuração: uma para Elevar o Nível de Configuração, ou seja, escolher
uma Configuração de capacidade superior, e outra para Reduzir o Nível de 
Configuração, isto é, escolher uma Configuração de capacidade inferior. Em ambos
os casos, a Heurística deverá usar como dado de entrada o Resultado da última 
Execução.

As Heurísticas são livres para usar os dados do Resultado como melhor lhe 
aprouverem, desde que a saída seja uma Configuração que ainda não tenha sido 
usada nos testes anteriores. 

\subsection{Selecionar Nova Carga de Trabalho}
De maneira similar à escolha de uma nova Configuração, o controlador pode usar
a Heurística para selecionar uma nova Carga de Trabalho, de acordo com o 
resultado com a última Execução.

As Heurísticas devem, então, prover operações que permitam a navegação pela
faixa de valores de Cargas de Trabalho estudada para a Aplicação sob Teste. 
Portanto, uma Heurística compatível deve fornecer duas operações de controle do
nível de Carga de Trabalho: uma operação para que seja reduzido e outra operação 
para que seja elevado o nível de Carga de Trabalho.

Aqui também as Heurísticas são livres para criarem suas próprias lógicas de 
avaliação dos dados do Resultado e, a partir daí, definirem qual o tamanho do
passo no caminhamento sobre a faixa de Cargas de Trabalho.
 
% ----------------------------------------------------------

% ----------------------------------------------------------

% ----------------------------------------------------------
% Cloud Capacitor
% ----------------------------------------------------------
\chapter{Cloud Capacitor}
\label{chap:capacitor}
% ----------------------------------------------------------
A fim de confirmar as hipóteses de eficácia e eficiência do emprego do Processo
de Avaliação de Capacidade descrito no capítulo anterior, bem como da técnica de
Inferência de Desempenho e das Heurísticas de Seleção que dão suporte a esse Processo, 
criamos uma implementação concreta de sua especificação na forma de uma biblioteca
extensível e de um sistema computacional que demonstra seu funcionamento fazendo
a avaliação de uma aplicação real em um provedor de nuvem de infraestrutura.

Demos o nome de CloudCapacitor à biblioteca, implementada como uma \emph{gem} da
linguagem Ruby~\cite{ruby}. Desenvolvemos o sistema computacional Capacitor Web 
para ser uma interface visual para a utilização do CloudCapacitor, usando 
o \emph{framework} Ruby on Rails~\cite{rails}.  

Descrevemos a seguir os detalhes da implementação de cada um e como ambos se
relacionam para oferecer ao usuário a experiência da avaliação de capacidade
de baixo custo e alta precisão prevista pelo Processo proposto, com uma interface
amigável e de fácil utilização.

\section{CloudCapacitor}
CloudCapacitor é uma biblioteca para criação de sistemas de avaliação de 
capacidade em ambientes de nuvem de infraestrutura como serviço. É a implementação
completa da especificação do Processo de Avaliação de Capacidade do 
Capítulo~\ref{chap:processo}, permitindo que sejam customizadas as atividades 
definidas pelo Processo como pontos de extensão, como as Estratégias de Avaliação
e o disparo e controle da execução da Aplicação sob Teste.

Vamos iniciar a apresentação do CloudCapacitor pelas classes que compõem a biblioteca
e suas responsabilidades. Em seguida, veremos como CloudCapacitor auxilia 
desenvolvedores de software na criação de sistemas de avaliação de capacidade,
mostrando o fluxo de utilização da biblioteca através de sua interface de programação.
Depois, falaremos sobre alguns detalhes de implementação da biblioteca, como 
a solução para representação do Espaço de Implantação e seu papel na execução do
Processo de Avaliação de Capacidade. Por fim, apresentaremos os pontos de extensão
da biblioteca, notadamente como implementar um Executor, a classe responsável pelo
controle de execução da Aplicação sob Teste, e como sobrescrever a Estratégia de
Avaliação fornecida pela biblioteca a fim de alterar seu comportamento padrão.

Para concluir a apresentação do CloudCapacitor, será mostrada a saída de dados
fornecida pela biblioteca com as Configurações capazes de executar a
Aplicação sob Teste em cada uma das Cargas de Trabalho respeitando o SLA definido.

\subsection{Classes e Responsabilidades}
\label{subsec:classes}
CloudCapacitor é formado por um conjunto de classes que, juntas representam os
componentes envolvidos na avaliação de capacidade de aplicações em ambientes de
nuvem, de acordo com os conceitos e o Processo definidos anteriormente, nos
Capítulos~\ref{chap:formalizacao} e~\ref{chap:processo}. A Figura~\ref{fig:classes}
mostra as principais classes cujas responsabilidades e cooperação levam ao 
resultado final de uma Avaliação. 

Ao utilizar a biblioteca CloudCapacitor na construção de um software para avaliação
de capacidade, o desenvolvedor tem à sua disposição uma classe principal, chamada
\emph{Cacitor}. Essa classe fornece o fluxo principal do Processo de Avaliação, com todos
os seus pontos de decisão e de extensão.

\begin{figure}[htb]
  \caption{\label{fig:classes}Principais classes que compõem o CloudCapacitor e suas relações}
  \begin{center}
    \includegraphics[scale=0.75]{img/CapacitorClasses}
  \end{center}
\end{figure}

Entre os pontos de extensão, destacamos o uso da classe \emph{Strategy}. Como 
vemos na Figura~\ref{fig:classes}, essa classe é fornecida pela biblioteca e a
notação de linhas pontilhadas denota a sua possibilidade de especialização, onde,
sobrescrevendo alguns de seus métodos, é possível criar Estratégias com comportamento
diversificado.

A classe \emph{DefaultExecutor} é o outro ponto de extensão oferecido pela biblioteca.
Porém, neste caso, o desenvolvedor deve mandatoriamente implementar uma subclasse
que forneça a lógica necessária ao controle da execução do teste de desempenho.
Na figura, a classe a ser implementada pelo desenvolvedor é representada com o 
nome de \emph{RealExecutor}.

Para que a Avaliação de Capacidade possa ser efetuada, a classe \emph{Capacitor}
deve conhecer o resultado de cada execução da Aplicação sob Teste, de modo que
possa tomar as decisões corretas na indicação das Configurações Candidatas e
Rejeitadas. Esses resultados são encapsulados na classe \emph{Result}, cujos 
objetos são fornecidos pela subclasse responsável pela execução dos testes de 
desempenho.

E, finalmente, durante a execução da Avaliação de Capacidade, a classe \emph{Capacitor} 
precisa ter conhecimento do Espaço de Implantação disponibilizado. Essa é a 
responsabilidade da classe \emph{DeploymentSpace}, que implementa uma estrutura 
de dados em memória para representar os diversos Níveis de Capacidade formados 
entre as Configurações. A construção dessa estrutura é responsabilidade da classe 
\emph{DeploymentSpaceBuilder}, que contém os algoritmos necessários à preparação 
do grafo usado para navegação pelos Níveis de Capacidade. 

***RASCUNHO***
CloudCapacitor
OK  Diagrama de Classes (UML) / Arquitetura de Componentes
OK  Enumeração das classes e suas responsabilidades
  Fluxo de utilização da biblioteca (caixa preta, interface com o Capacitor)
  Especificação e Carregamento do Deployment Space
  Executor - Implementação
  Estratégia - Descrição da interface
  Estratégia - Como sobrescrever (Gof Strategy Pattern)
  Apresentação do resultado da avaliacao
Capacitor Web
  Apresentação da interface de entrada
  Resultados
  Trace
  Full Trace
Resumo e transicao para o proximo capitulo
***RASCUNHO***



PorémAo instanciar um objeto da classe Capacitor, o desenvolvedor deve atribuir  
Por padrão, o CloudCapacitor oferece uma Estratégia de Avaliação que implementa 
uma lógica simples para as Heurísticas de Seleção de Configurações definidas no
capítulo anterior. Essa lógica pode ser facilmente sobrescrita conforme a necessidade
do usuário ou o perfil da Aplicação sob Teste.


****RASCUNHO
\subsection{Heuristicas}
Para que uma Heurística de Avaliação de Capacidade seja compatível no âmbito deste trabalho, 
deve apresentar um conjunto mínimo de operações esperadas para que a lógica da
avaliação se complete e o resultado final obtido possa ser considerado válido e
comparável com os resultados obtidos por outras Heurísticas.

Além disso, as operações constituem a interface pela qual o controlador das 
sessões de avaliação pode configurar as Heurísticas e informar-lhe os dados 
necessários ao controle da sua execução.
 
Apresentamos esse conjunto mínimo de operações nas subseções a seguir, que 
representam o arcabouço necessário para a construção de uma Heurística de 
Avaliação de Capacidade.
 
% ----------------------------------------------------------

% ----------------------------------------------------------

% ----------------------------------------------------------
% Resultados
% ----------------------------------------------------------
\chapter{Experimentos e Resultados}
\label{chap:resultados}
% ----------------------------------------------------------
Este capítulo vem apresentar os experimentos realizados como forma de
verificação do Processo de Avaliação de Capacidade por Inferência de 
Desempenho proposto no Capítulo~\ref{chap:processo}. Inicialmente é
apresentada a metodologia utilizada para construção dos experimentos, 
com a descrição da Aplicação sob Teste escolhida, como foi implantada e 
como foram realizadas as execuções para coleta de dados de desempenho. Depois
são apresentados os resultados obtidos por cada uma das 9 Heurísticas ao
fazer a Avaliação de Capacidade da Aplicação. Esses resultados são usados 
para uma comparação qualitativa das Heusrísticas entre si e para atestar
a eficiência do Processo de Avaliação de Capacidade proposto e sua técnica 
de Inferência de Desempenho tanto quanto à economia de tempo e custo como  
quanto à precisão de acerto de seuas predições.

\section{Metodologia}
\label{sec:resultados_metodologia}
A fim de validar a eficiência da Inferência de Desempenho no apoio
ao Planejamento de Capacidade, foram realizadas seções de avaliação de
capacidade de uma aplicação implantada em um provedor de nuvem de
infraestrutura como serviço.

A aplicação escolhida foi o WordPress \cite{wordpress}, um motor de construção 
e administração de \emph{blogs}. Sua escolha foi motivada por ser uma aplicação
bem conhecida, de utilização via web, ideal para implantação em ambiente de
nuvem, e com componentes arquiteturais escaláveis. Além disso, o fluxo de 
utilização típico apresenta características bem diversificadas quanto ao uso
de recursos de CPU e memória, rede, sistema de arquivos e banco de dados.

\begin{figure}[htb]
  \caption{\label{fig:implantacao}Implantação do WordPress na AWS EC2 para Avaliação de Capacidade}
  \begin{center}
    \includegraphics[scale=0.5]{img/ImplantacaoWordPress}
  \end{center}
\end{figure}

O Provedor escolhido foi a Amazon, com seu serviço de infraestrutura AWS EC2
\cite{ec2}, onde o WordPress foi implantado em duas camadas: uma para o banco de 
dados MySQL, e outra para a aplicação, executada pelo servidor Apache HTTPD. Como
balanceador de carga, foi utilizada uma máquina executando o servidor web Nginx. A
Figura~\ref{fig:implantacao} mostra um panorama geral dessa implantação. 

%Na imagem,
%podem ser vistos que na camada do WordPress algumas máquinas aparecem esmaecidas.
%Isso foi feito para representar que os testes variam


% ----------------------------------------------------------

% ----------------------------------------------------------

% ----------------------------------------------------------
% Conclusão
% ----------------------------------------------------------
\chapter{Conclusão}
\label{chap:conclusao}
% ----------------------------------------------------------
Este capítulo encerra este trabalho, apresentando uma síntese dos resultados 
obtidos, as contribuições e sugestões para trabalhos futuros.

\section{Resultados e Contribuições}
A oportunidade de adoção da computação em nuvem como alternativa para implantação 
de aplicações corporativas, ante a expectativa de economia gerada pela tarifação 
sob demanda e a elasticidade de recursos, traz consigo a necessidade de uma
avaliação prévia da viabilidade financeira da migração dessas aplicações para
o novo ambiente. Essa análise de viabilidade passa pela identificação de quais 
combinações de recursos, entre os existentes na nuvem, são potencialmente capazes 
de manter ou, preferencialmente, suplantar o desempenho original da aplicação em 
questão. Entre esses recursos, as máquinas virtuais geralmente representam os
maiores custos de operação para aplicações implantadas na nuvem. Ao mesmo tempo,
é muito difícil prever qual será o custo operacional total da aplicação quando
implantada na nuvem, exatamente devido à característica de escalabilidade e 
elasticidade proporcionada pelo modelo de implantação em infraestrutura como
serviço.

Nesse contexto, este trabalho propôs uma nova abordagem para a pesquisa das 
melhores combinações de recursos de máquinas virtuais em provedores de nuvem IaaS,
baseada no Processo de Inferência de Desempenho para Planejamento de
Capacidade, como descrito no Capítulo~\ref{chap:processo}. O Processo se apoia
na inferência de desempenho para determinadas configurações a partir da observação
do desempenho real de uma configuração ao executar de fato a aplicação no
ambiente de nuvem. A inferência é realizada pela interpretação do mapeamento das 
relações de capacidade existentes entre as diversas configurações que o provedor 
disponibiliza. A alta acurácia das inferências, acima de 99\% em média, conforme 
verificada pelos experimentos demonstrados no Capítulo~\ref{chap:resultados}, denota 
a confiabilidade das respostas dadas pela utilização do Processo de Inferência 
no apoio às atividades de planejamento de capacidade para migração de aplicações 
para o ambiente de nuvem. 

Paralelamente à confiabilidade, outro diferencial do Processo de Inferência é a 
economia financeira e operacional, em termos de tempo, na realização da fase de 
testes do planejamento de capacidade. Dado o enorme número de combinações de 
configurações possíveis entre recursos do Provedor, o uso de estratégias inteligentes,
com a aplicação das Heurísticas propostas para a seleção das combinações mais 
promissoras na busca pelas configurações mais baratas, confere ao Processo 
uma eficiência notável, destacada pela economia de até 96\% em custos e até 88\% 
no tempo de execução dos testes.  
  
\section{Sugestões para Trabalhos Futuros}
O Processo de Inferência de Desempenho como apoio ao planejamento de capacidade,
proposto e estudado neste trabalho, oferece diversas oportunidades de continuidade. 
Apresentamos algumas sugestões de trabalhos que podem ser criados como extensão 
à pesquisa realizada:

\begin{description}
\item[Implementação de novas Heurísticas] \hfill \\
O Processo de Inferência de Desempenho é extensível do ponto de vista das
Estratégias de Avaliação, responsáveis por implementar as Heurísticas de Seleção
de Configurações. Esta primeira implementação concreta do Processo utilizou 
Heurísticas focadas apenas no valor da Métrica analisada nos experimentos, isto é,
o tempo de resposta total das requisições. A implementação de Heurísticas mais
criteriosas, que levem em consideração outros dados, podem influenciar a eficiência
de custo do Processo. Por exemplo, Heurísticas podem confrontar o tempo de resposta
com a carga de comprometimento de CPU e Memória das Configurações a cada execução
e, dependendo da análise, optar por um salto maior nos Níveis de Capacidade e, 
com isso, economizar ainda mais no número de execuções reais. 

\item[Análise do impacto da flutuação de desempenho] \hfill \\
Observou-se através da análise da acurácia do Processo que a flutuação de 
desempenho do Provedor, em um certo período durante a execução dos experimentos,
causou uma leve perturbação nos resultados obtidos. Faz-se oportuno estudar o 
impacto da flutuação de desempenho do Provedor sobre o comportamento das Heurísticas
e do Processo como um todo, visando a melhor compreender-se até que ponto a 
confiabilidade das respostas é imune a esses eventos.
 
\item[Investigação de resultados para outras aplicações] \hfill \\
A aplicação escolhida para os experimentos deste trabalho representa em boa
medida o perfil comum de aplicações comerciais ou corporativas, isto é, aplicações
em ambiente Web, multicamadas, suportadas por bancos de dados relacionais. Por
outro lado, há novos paradigmas de aplicações em franco desenvolvimento, apoiadas
em protocolos de comunicação assíncrona, bancos de dados não relacionais, 
transcodificação multimídia, webservices, entre outros. É de grande interesse
o estudo da efetividade do uso do Processo de Inferência para o planejamento de
capacidade de aplicações com perfis diferenciados de implantação. Adicionalmente,
outras camadas, além da camada de aplicação, podem ser estudadas com o uso do
Processo. Há ainda a oportunidade de se estudarem outros provedores de IaaS, com
perfis de Configurações e relações de capacidade potencialmente distintos dos já
estudados.

\item[Evolução do Cloud Capacitor] \hfill \\
Por fim, existe a possibilidade de que o Cloud Capacitor seja evoluído para agregar
funções de implantação e avaliação de aplicações, atividades que hoje são 
realizadas através de chamadas a ferramentas externas especializadas nessas 
tarefas. A agregação dessas funcionalidades ao Cloud Capacitor formaria o alicerce
mínimo para o desenvolvimento de uma solução completa de planejamento de capacidade 
para aplicações em nuvens IaaS.

\end{description}

% ----------------------------------------------------------

% ----------------------------------------------------------

% ----------------------------------------------------------
% ELEMENTOS PÓS-TEXTUAIS
% ----------------------------------------------------------
\postextual
% ----------------------------------------------------------

% ----------------------------------------------------------
% Referências bibliográficas
% ----------------------------------------------------------
\bibliography{dissertacao}

\end{document}
