% resumo em português
\setlength{\absparsep}{18pt} % ajusta o espaçamento dos parágrafos do resumo
\begin{resumo}
Um dos principais desafios enfrentados pelos usuários de nuvens que oferecem 
infraestrutura-como-serviço (IaaS) é planejar adequadamente a capacidade dos 
recursos da nuvem necessários às suas aplicações. Este trabalho propõe uma nova 
abordagem para apoiar o planejamento da capacidade de aplicações em nuvens IaaS. 
A nova a abordagem tem como premissa a definição de uma relação de capacidade 
entre as diferentes configurações de recursos oferecidas por um provedor de nuvem, 
com a qual é possível prever (ou ``inferir''), com alto grau de precisão, o 
desempenho esperado de uma aplicação para determinadas configurações de recursos. 
A predição é realizada com base no desempenho observado para outras configurações 
de recursos do mesmo provedor. Dessa forma, a abordagem consegue reduzir, de 
forma significativa, o número total de configurações que precisam ser de fato 
testadas na nuvem (resultados preliminares, obtidos em um ambiente real de nuvem, 
mostram uma redução de mais de 80\% no número total de configurações avaliadas), 
implicando em menores custo e tempo para o processo de planejamento.

 \textbf{Palavras-chaves}: Computação em Nuvem. Planejamento de Capacidade. Inferência de Desempenho.
\end{resumo}

% resumo em inglês
\begin{resumo}[Abstract]
 \begin{otherlanguage*}{english}
One of the main challenges faced by users of infrastructure-as-a-service (IaaS) 
clouds is to correctly plan the resource capacity required for their applications'
needs. This work proposes a new approach to support application capacity planning 
in IaaS clouds. This new approach is based on the definition of a capacity relation 
between different resource configurations offered by a cloud provider which enables 
to predict (or ``infer''), with a high level of accuracy, the expected performance 
of an application for certain resource configurations. The prediction is made based
upon the observed performance for other resource configurations within the same 
provider. The approach significantly reduces the total number of configurations 
effectively tested in the cloud (preliminary results show reductions of over 80\% 
on the number of total tested configurations) resulting in lower costs and time 
for the capacity planning process.

   \vspace{\onelineskip}
 
   \noindent 
   \textbf{Key-words}: Cloud Computing. Capacity Planning. Performance Inference.
 \end{otherlanguage*}
\end{resumo}
