\chapter{Conclusão}
\label{chap:conclusao}
% ----------------------------------------------------------
Este capítulo encerra este trabalho, apresentando uma síntese dos resultados 
obtidos, as contribuições e sugestões para trabalhos futuros.

\section{Resultados e Contribuições}
A oportunidade de adoção da computação em nuvem como alternativa para implantação 
de aplicações corporativas, ante a expectativa de economia gerada pela tarifação 
sob demanda e a elasticidade de recursos, traz consigo a necessidade de uma
avaliação prévia da viabilidade financeira da migração dessas aplicações para
o novo ambiente. Essa análise de viabilidade passa pela identificação de quais 
combinações de recursos, entre os existentes na nuvem, são potencialmente capazes 
de manter ou, preferencialmente, suplantar o desempenho original da aplicação em 
questão. Entre esses recursos, as máquinas virtuais geralmente representam os
maiores custos de operação para aplicações implantadas na nuvem. Ao mesmo tempo,
é muito difícil prever qual será o custo operacional total da aplicação quando
implantada na nuvem, exatamente devido à característica de escalabilidade e 
elasticidade proporcionada pelo modelo de implantação em infraestrutura como
serviço.

Nesse contexto, este trabalho propôs uma nova abordagem para a pesquisa das 
melhores combinações de recursos de máquinas virtuais em provedores de nuvem IaaS,
baseada no Processo de Inferência de Desempenho para Planejamento de
Capacidade, como descrito no Capítulo~\ref{chap:processo}. O Processo se apoia
na inferência de desempenho para determinadas configurações a partir da observação
do desempenho real de uma configuração ao executar de fato a aplicação no
ambiente de nuvem. A inferência é realizada pela interpretação do mapeamento das 
relações de capacidade existentes entre as diversas configurações que o provedor 
disponibiliza. A alta acurácia das inferências, acima de 99\% em média, conforme 
verificada pelos experimentos demonstrados no Capítulo~\ref{chap:resultados}, denota 
a confiabilidade das respostas dadas pela utilização do Processo de Inferência 
no apoio às atividades de planejamento de capacidade para migração de aplicações 
para o ambiente de nuvem. 

Paralelamente à confiabilidade, outro diferencial do Processo de Inferência é a 
economia financeira e operacional, em termos de tempo, na realização da fase de 
testes do planejamento de capacidade. Dado o enorme número de combinações de 
configurações possíveis entre recursos do Provedor, o uso de estratégias inteligentes,
com a aplicação das Heurísticas propostas para a seleção das combinações mais 
promissoras na busca pelas configurações mais baratas, confere ao Processo 
uma eficiência notável, destacada pela economia de até 96\% em custos e até 88\% 
no tempo de execução dos testes.  
  
\section{Sugestões para Trabalhos Futuros}
O Processo de Inferência de Desempenho como apoio ao planejamento de capacidade,
proposto e estudado neste trabalho, oferece diversas oportunidades de continuidade. 
Apresentamos algumas sugestões de trabalhos que podem ser criados como extensão 
à pesquisa realizada:

\begin{description}
\item[Implementação de novas Heurísticas] \hfill \\
O Processo de Inferência de Desempenho é extensível do ponto de vista das
Estratégias de Avaliação, responsáveis por implementar as Heurísticas de Seleção
de Configurações. Esta primeira implementação concreta do Processo utilizou 
Heurísticas focadas apenas no valor da Métrica analisada nos experimentos, isto é,
o tempo de resposta total das requisições. A implementação de Heurísticas mais
criteriosas, que levem em consideração outros dados, podem influenciar a eficiência
de custo do Processo. Por exemplo, Heurísticas podem confrontar o tempo de resposta
com a carga de comprometimento de CPU e Memória das Configurações a cada execução
e, dependendo da análise, optar por um salto maior nos Níveis de Capacidade e, 
com isso, economizar ainda mais no número de execuções reais. 

\item[Análise do impacto da flutuação de desempenho] \hfill \\
Observou-se através da análise da acurácia do Processo que a flutuação de 
desempenho do Provedor, em um certo período durante a execução dos experimentos,
causou uma leve perturbação nos resultados obtidos. Faz-se oportuno estudar o 
impacto da flutuação de desempenho do Provedor sobre o comportamento das Heurísticas
e do Processo como um todo, visando a melhor compreender-se até que ponto a 
confiabilidade das respostas é imune a esses eventos.
 
\item[Investigação de resultados para outras aplicações] \hfill \\
A aplicação escolhida para os experimentos deste trabalho representa em boa
medida o perfil comum de aplicações comerciais ou corporativas, isto é, aplicações
em ambiente Web, multicamadas, suportadas por bancos de dados relacionais. Por
outro lado, há novos paradigmas de aplicações em franco desenvolvimento, apoiadas
em protocolos de comunicação assíncrona, bancos de dados não relacionais, 
transcodificação multimídia, webservices, entre outros. É de grande interesse
o estudo da efetividade do uso do Processo de Inferência para o planejamento de
capacidade de aplicações com perfis diferenciados de implantação. Adicionalmente,
outras camadas, além da camada de aplicação, podem ser estudadas com o uso do
Processo. Há ainda a oportunidade de se estudarem outros provedores de IaaS, com
perfis de Configurações e relações de capacidade potencialmente distintos dos já
estudados.

\item[Evolução do Cloud Capacitor] \hfill \\
Por fim, existe a possibilidade de que o Cloud Capacitor seja evoluído para agregar
funções de implantação e avaliação de aplicações, atividades que hoje são 
realizadas através de chamadas a ferramentas externas especializadas nessas 
tarefas. A agregação dessas funcionalidades ao Cloud Capacitor formaria o alicerce
mínimo para o desenvolvimento de uma solução completa de planejamento de capacidade 
para aplicações em nuvens IaaS.

\end{description}

% ----------------------------------------------------------
