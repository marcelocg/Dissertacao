\chapter{Conclusão}
\label{chap:conclusao}
% ----------------------------------------------------------
Este capítulo encerra este trabalho, apresentando uma síntese dos resultados 
obtidos, as contribuições e sugestões para trabalhos futuros.

\section{Resultados e Contribuições}
A oportunidade de adoção da computação em nuvem como alternativa para implantação 
de aplicações corporativas ante a expectativa de economia gerada pela tarifação 
sob demanda e a elasticidade de recursos traz consigo a necessidade de uma
avaliação prévia da viabilidade financeira da migração dessas aplicações para
o novo ambiente. Essa análise de viabilidade passa pela identificação de quais 
combinações de recursos entre os existentes na nuvem são potencialmente capazes 
de manter ou, preferencialmente, suplantar o desempenho original da aplicação em 
questão. Entre esses recursos, o de máquinas virtuais geralmente representam os
maiores custos de operação para aplicações implantadas na nuvem. Ao mesmo tempo,
é muito difícil prever qual será o custo operacional total na nuvem, exatamente
devido à característica de escalabilidade e elasticidade proporcionada pelo modelo
de implantação em nuvem IaaS.

Nesse contexto, este trabalho propôs uma nova abordagem para a pesquisa das 
melhores combinações de recursos de máquinas virtuais em provedores de nuvem IaaS.
A proposta se baseia no Processo de Inferência de Desempenho para Planejamento de
Capacidade, como descrito no Capítulo~\ref{chap:processo}. O Processo se apoia
na inferência de desempenho para determinadas configurações a partir da observação
do desempenho real de uma configuração ao executar de fato a aplicação no
ambiente de nuvem. A inferência é realizada pela interpretação do mapeamento das 
relações de capacidade existentes entre as diversas configurações que o provedor 
disponibiliza. A alta acurácia das inferências, acima de 99\% em média, conforme 
verificada pelos experimentos demonstrados no Capítulo~\ref{chap:resultados}, denota 
a confiabilidade das respostas dadas pela utilização do Processo de Inferência 
no apoio às atividades de planejamento de capacidade para migração de aplicações 
para o ambiente de nuvem. 

Paralelamente à confiabilidade, outro diferencial do Processo de Inferência é a 
economia financeira e operacional, em termos de tempo, na realização da fase de 
testes do planejamento de capacidade. Dado o enorme número de combinações de 
configurações possíveis entre recursos do Provedor, o uso de estratégias inteligentes,
com a aplicação das Heurísticas propostas para a seleção das combinações mais 
promissoras na busca pelas mais configurações mais baratas, confere ao Processo 
uma eficiência notável, destacada pela economia de até 96\% em custos e até 88\% 
em tempo de execução dos testes.  
  
\section{Sugestões para Trabalhos Futuros}
- realização de novos experimentos visando investigar até que ponto os resultados obtidos durante o mestrado podem ser generalizados para outras aplicações e provedores de nuvem;

- implementação e avaliação de novas heurísticas de inferência de desempenho, que levem em conta os dados de monitoramento dos recursos utilizados, incluindo memória, CPU e a distância do desempenho observado para o desempenho esperado (o "delta").

-  evolução do Capacitor para oferecer um serviço completo de implantação, avaliação e planejamento de capacidade de aplicações em nuvens IaaS.

% ----------------------------------------------------------
