\chapter[Fundamentação Teórica e Trabalhos Relacionados]{Fundamentação Teórica e Trabalhos\\Relacionados}

% ----------------------------------------------------------
Este capítulo apresenta alguns conceitos básicos ligados à Computação em Nuvem, incluindo sua definição, seus modelos de implantação e seus tipos de serviços. O capítulo também aborda a questão do planejamento da capacidade de recursos na nuvem, descrevendo e analisando criticamente vários trabalhos existentes relacionados ao tema.

\section{Computação em Nuvem}
Embora ainda não haja uma definição oficial, a análise da literatura disponível 
em torno da computação em nuvem permite caracterizá-la como ``um modelo de computação 
distribuída em que usuários podem rapidamente provisionar e liberar, sob demanda, recursos 
computacionais virtualizados configuráveis, de maneira escalável e elástica, através 
de um serviço provido pela rede ou Internet'' 
\cite{foster2009cloud,cearley2010case,mell2011nist}. 

A partir dessa caracterização, é possível destacar três ideias chave por trás da computação em nuvem: \textbf{escalabilidade} --- 
relacionada à percepção de recursos ilimitados; \textbf{serviço} --- similar ao 
fornecimento de serviços utilitários como energia elétrica ou gás, que são pagos 
apenas quando utilizados; \textbf{rede e Internet} --- relativa ao meio através do qual o 
consumidor terá acesso ao serviço; \textbf{rápido provisionamento e liberação de recursos}
 --- como forma de garantir escalabilidade, e o acesso aos recursos através da rede, não
necessariamente a Internet.

Em~\cite{vaquero2008break}, os autores discutem o paradigma da computação em
nuvem através da identificação e estudo de 20 definições disponíveis na literatura, 
para então extrair uma definição mais abrangente, contendo as características comuns 
a várias das definições analisadas:

\begin{citacao}
A computação em nuvem ``consiste em um grande \emph{pool} de recursos virtualizados 
(tais como hardware, plataformas de desenvolvimento e/ou serviços) facilmente 
utilizáveis e acessíveis. Esses recursos podem ser dinamicamente reconfigurados 
para se ajustarem a uma carga variável, permitindo também a sua utilização ótima. 
Esse \emph{pool} de recursos é tipicamente explorado em um modelo de pagar-pelo-uso, 
no qual o provedor de infraestrutura oferece garantias por meio de acordos de nível 
de serviço customizados.''\end{citacao}

Essa última definição menciona o termo \textbf{acordo de nível de serviço} (ou SLA, do Inglês \emph{Service-Level Agreement}), uma característica particularmente importante para aplicações sensíveis a flutuações 
de desempenho. Outra característica mencionada é o modelo de \textbf{pagar-pelo-uso}, 
o que permite que uma aplicação se adapte a variações na demanda com a adição de 
novos recursos ou a liberação de recursos existentes, restringindo seus gastos 
com a nuvem apenas a recursos que de fato necessita e utiliza.

\subsection{Modelos de Implantação}

Um serviço de disponibilização de recursos computacionais ao usuário, característico da computação em nuvem, pode ser implantado de maneiras diferentes,
a depender da sua localização em relação à entidade corporativa representada pelo
usuário e ao meio pelo qual esse usuário tem acesso a esse 
serviço~\cite{armbrust2009above,Zhang2010}. Essa diferenciação na maneira de implantar os serviços disponibilizados ao usuários permite classificar as atuais plataformas de computação em nuvem de acordo com diferentes \emph{modelos de implantação}. 
 
\begin{description}
\item[Nuvem privada] \hfill \\ É construída para ser utilizada por uma única 
organização e, geralmente, a alocação de seus recursos se dá por uma rede interna. 
Esse modelo de nuvem oferece todos os benefícios normalmente atribuídos à computação 
em nuvem, como recursos virtualizados e escalabilidade das aplicações. A exceção 
é o seu alto custo de manutenção, já que a própria organização é responsável pelo 
gerenciamento da infraestrutura física. Atualmente há diversas soluções que permitem 
a criação e o gerenciamento de uma nuvem privada dentro de uma organização, com 
destaque para soluções de código aberto como OpenStack~\cite{openstack}, 
Eucalyptus~\cite{eucalyptus} e Open-Nebula~\cite{opennebula}.

\item[Nuvem pública] \hfill \\ A nuvens públicas constituem o modelo de nuvem 
mais conhecido, e disponibilizam recursos de nuvem para o público em geral. Esses 
recursos normalmente estão disponíveis através da Internet e são gerenciados por 
uma organização (provedor da nuvem), que pode cobrar pela sua utilização. 

\item[Nuvem híbrida] \hfill \\ No modelo de implantação híbrido, recursos de uma 
nuvem pública são alocados em conjunto com os de uma nuvem privada. Um exemplo de 
aplicabilidade desse modelo é o caso dos recursos da nuvem privada terem se exaurido,
possivelmente num momento de pico de utilização de uma aplicação. Neste caso, 
novos recursos podem ser alocados a partir de uma nuvem pública. O modelo de 
nuvem híbrida é implementado em algumas soluções como OpenNebula~\cite{opennebula}, 
que permite a integração de recursos de uma nuvem privada com recursos da nuvem 
da Amazon, e Righscale~\cite{rightscale}, que oferece um serviço de nuvem híbrida 
comercial.
\end{description}

Este trabalho tem seu foco voltado ao planejamento de capacidade em
nuvens públicas e, portanto, este foi o modelo de implantação adotado na execução
dos experimentos descritos no Capítulo~\ref{chap:resultados}. 

\subsection{Tipos de Serviço de Nuvem}

Outra forma de classificar as plataformas de computação em nuvem é quanto aos tipos 
dos serviços oferecidos. Os três tipos de serviços de nuvem mais comumente citados 
são: infraestrutura como serviço (do Inglês \emph{Infrastructure-as-a-Service}, ou IaaS), plataforma como serviço (do Inglês \emph{Platform-as-a-Service}, ou PaaS), e 
software como serviço (do Inglês \emph{Software-as-a-Service}, ou SaaS)~\cite{armbrust2010view,Zhang2010}. 

\begin{description}
\item[Nuvem IaaS] \hfill \\ Nuvens to tipo IaaS oferecem 
aos clientes recursos básicos de infraestrutura, como processamento, armazenamento 
e rede. Nessa modalidade de serviço, o usuário tem à sua disposição um conjunto
de recursos virtuais, geralmente agrupados por sua capacidade ou finalidade, para
que escolha e monte a configuração que mais se adeque às necessidade de desempenho
da sua aplicação. Entre esses recursos estão máquinas virtuais, discos e redes de alta velocidade. Comparado 
aos outros dois tipos de serviço de nuvem, os serviços do tipo IaaS oferecem muito 
mais flexibilidade de configuração pelos clientes, que assim podem implantar e 
executar na nuvem as aplicações de sua escolha, inclusive os próprios sistemas 
corporativos com pouca ou nenhuma modificação. Alguns dos provedores de nuvens 
IaaS mais conhecidos são Amazon, com seu serviço EC2 -- \emph{Elastic Compute 
Cloud}~\cite{ec2}, e Rackspace~\cite{rackspace}.


\item[Nuvem PaaS] \hfill \\ Nas nuvens PaaS, o desenvolvimento 
das aplicações fica sob responsabilidade dos clientes, cabendo ao provedor da 
nuvem oferecer uma plataforma de execução apropriada ao cliente. Nesse tipo de 
nuvem, o provedor é responsável por manter o ambiente operacional, garantindo a 
disponibilidade dos serviços e realizando os ajustes necessários para que as 
aplicações desenvolvidas pelos clientes atendam à demanda de seus usuários. 
Dois dos provedores de nuvem PaaS mais conhecidos são Microsoft e Google, que 
oferecem os serviços Azure~\cite{azure} e Google AppEngine~\cite{appengine}, 
respectivamente. No Azure, o cliente da nuvem tem acesso às plataformas 
.NET, Node.js, Java, PHP, Python e Ruby, além das plataformas móveis iOS, Android
e Windows Phone. Já no Google App Engine, o desenvolvedor pode escolher entre 
as linguagens Java, Python, PHP e Go. No entanto, uma aplicação Java ou Phyton 
já existente dificilmente poderia ser portada para um desses serviços sem sofrer 
modificações, pois em ambos há restrições ou a necessidade de configurações específicas
em suas APIs quando comparadas às plataformas tradicionais.

\item[Nuvem SaaS] \hfill \\ Nuvens SaaS caracterizam-se por ter como cliente o
próprio usuário final do serviço disponibilizado, que normalmente consiste em uma
aplicação hospedada na nuvem. Nesse tipo de nuvem, é o provedor da nuvem quem se
responsabiliza pelo desenvolvimento e atualização dos serviços (aplicações) 
fornecidos aos usuários. Os exemplos mais comuns são serviços de email, discos virtuais, 
CRM e editores online, que muitas vezes são ofertados sem custo inicial para seus 
usuários. No caso do email, um dos serviços mais conhecidos é o GMail, que oferece 
uma determinada capacidade de armazenamento gratuitamente, com o usuário tendo a 
opção de pagar para estender essa capacidade inicial caso necessite de mais espaço. 
Serviços de discos virtuais, como Dropbox e GoogleDrive, também seguem uma política 
de preços similar. Já serviços de CRM (do Inglês \emph{Customer Relationship Management}), 
como Salesforce1 \cite{salesforce} e Zoho Creator \cite{zoho}, oferecem um período de teste gratuito 
para o usuário, após o qual o serviço passa a ser pago. O GitHub \cite{github} é um 
serviço gratuito de controle de versionamento de arquivos para repositórios 
públicos. Entretanto, o usuário tem a opção de pagar pelo serviço para ter direito 
a repositórios privados.



\end{description}

Este trabalho foca na modalidade de serviço de nuvem IaaS, mais especificamente na questão do planejamento da capacidade dos recursos virtuais a serem alocados às aplicações que serão executadas na nuvem.

Apresentamos a seguir uma breve descrição de alguns dos diferentes tipos de recursos oferecidos pelo serviço EC2,
da Amazon, um dos mais conhecidos serviços de nuvem IaaS disponíveis publicamente, o qual também foi utilizado para a realização dos experimentos que fizeram parte deste trabalho. 

\subsection{Exemplo de Nuvem Pública IaaS: Amazon EC2}

O serviço EC2 caracteriza-se principalmente pelo fornecimento de máquinas virtuais
que o usuário pode criar, ligar e desligar livremente, pagando pelo tempo de
utilização, ou seja, pelo tempo em que as máquinas estiverem ligadas. Essas 
máquinas são instanciadas a partir de configurações cujo poder computacional é 
pré-determinado pelo provedor, mas especificamente, quantidade de memória, 
quantidade de núcleos virtuais de CPU e poder de processamento total da CPU.

Os tipos de máquinas virtuais do serviço EC2 são categorizados segundo sua 
finalidade e segundo seus atributos de hardware. Há categorias para máquinas
que priorizam mais memória em detrimento de poder computacional, outras para 
máquinas cujo característica é o grande poder de processamento e até categorias
específicas de máquinas voltadas ao processamento gráfico ou em \emph{clusters}.
A Tabela~\ref{table:maquinas_ec2} mostra uma parte dos tipos de máquinas virtuais
oferecidos pelo serviço EC2 da Amazon, juntamente com os preços cobrados por hora e suas
características de hardware.

\begin{table}[t]
  \centering
  \begin{tabular}{|l|c|c|c|c|}
    \hline
%    \multirow{2}{*}{Tipos de Máqunas} & Núcleos & ECU\footnotemark & \parbox[m]{2cm}{\center{Memória\\(GiB)}} & US\$ / hora\\
    \textbf{Tipos de Máquina} & \textbf{Núcleos} & \textbf{ECU}\footnotemark & \parbox[m]{2cm}{\center{\textbf{Memória\\(GiB)}}} & \parbox[m]{2.3cm}{\center{\textbf{Preço\\(US\$/hora)}}}\\
%    \cline{2-5}
    \hline    
      \multicolumn{5}{|c|}{Máquinas de Propósito Geral} \\
      \hline      
      t2.micro  & 1 & Variável & 1 & 0.013 \\
      t2.small  & 1 & Variável & 2 & 0.026 \\
      t2.medium & 2 & Variável & 4 & 0.052 \\
      m3.medium & 1 & 3 & 3.75 & 0.070 \\
      m3.large  & 2 & 6.5 & 7.5 & 0.140 \\
      m3.xlarge & 4 & 13  & 15  & 0.280 \\
      m3.2xlarge & 8 & 26 & 30  & 0.560 \\
      \hline      
      \multicolumn{5}{|c|}{Máquinas Otimizadas para Computação} \\
      \hline      
      c3.large  & 2 & 7 & 3.75 & 0.105 \\
      c3.xlarge & 4 & 14 & 7.5 & 0.210 \\
      c3.2xlarge & 8 & 28 & 15 & 0.420 \\
      c3.4xlarge & 16 & 55 & 30 & 0.840 \\
      c3.8xlarge & 32 & 108 & 60 & 1.680 \\
      \hline      
      \multicolumn{5}{|c|}{Máquinas Otimizadas para Processamento Gráfico (GPU)} \\
      \hline      
      g2.2xlarge & 8 & 26 & 15 & 0.650 \\
      \hline      
      \multicolumn{5}{|c|}{Máquinas Otimizadas para Memória} \\
      \hline      
      r3.large & 2 & 6.5 & 15 & 0.175 \\
      r3.xlarge & 4 & 13 & 30.5 & 0.350 \\
      r3.2xlarge & 8 & 26 & 61 & 0.700 \\
      r3.4xlarge & 16 & 52 & 122 & 1.400 \\
      r3.8xlarge & 32 & 104 & 244 & 2.800 \\
    \hline    
  \end{tabular}
  \caption{\label{table:maquinas_ec2}Relação parcial de recursos e preços do serviço de nuvem EC2 na região \emph{US East}, tal como publicados pelo provedor em dezembro de 2014.}
\end{table}

\footnotetext{ECU (acrônimo para \emph{EC2 Compute Unit}) é uma medida própria criada pela Amazon 
para relativizar o poder de processamento de suas máquinas, e assim permitir uma 
comparação entre os vários tipos de máquinas oferecidos.}

Cada máquina pode ser individualmente configurada com opções diferentes de armazenamento
em massa. Há opções de armazenamento volátil, armazenamento em volumes magnéticos
comuns ou discos de estado sólido -- \emph{SSDs}. Essas opções, embora não alterem
a categorização da máquina, alteram o preço final da hora de utilização, acarretando
um custo adicional ao valor base da hora da configuração padrão da máquina. Além
disso, a utilização de alguns recursos acarreta mais um custo adicional, como o
tráfego de dados pela rede. A Amazon cobra uma tarifa por \emph{gigabyte} trafegado
para fora da rede local do provedor. Isso significa que mesmo o tráfego de rede 
entre centros de dados diferentes pode ser cobrado.

Os preços cobrados pela Amazon para a utilização de uma máquina virtual variam 
conforme o centro de dados escolhido pelo usuário no momento da criação da máquina.
Essa variação ocorre tanto para o preço base das máquinas como para recursos adicionais
e consumíveis, como tráfego de dados.

A Amazon possui centros de dados em todos os continentes, oferecendo redundância
e buscando minimizar problemas com latência de rede. Atualmente, existem 3 centros de
dados nos Estados Unidos, sendo 1 no estado da Virginia, 1 no Oregon e 1 na Califórnia; 2 
centros de dados na Europa, sendo 1 em Frankfurt (Alemanha) e 1 em Dublin (Irlanda); 3 centros na Ásia,
sendo 1 em Cingapura, 1 em Tóquio (Japão) e 1 em Sydney (Austrália); e 1 centro de dados na América do
Sul, em São Paulo (Brasil). Por motivos de custo, para os testes executados nos experimentos 
deste trabalho foi utilizado o centro de dados da região \emph{US East} (Virginia), o
mais barato entre todos, inclusive mais barato que o centro de dados no Brasil.   

%-- faltou falar dos diferentes modelos de precificação !!

% ----------------------------------------------------------

\section{Planejamento de Capacidade na Nuvem}

A questão do planejamento de capacidade de recursos na nuvem pode ser vista a partir de dois ângulos distintos: do ponto de vista do provedor do serviço e do ponto de vista do usuário do serviço \cite{Menasce2009}. Do ponto de vista do provedor do serviço, planejar a capacidade dos recursos significa monitorar a utilização e, sempre que necessário, adicionar novos  recursos computacionais à infraestrutura física onde são implantados os serviços disponibilizados aos usuários da nuvem. Nesse aspecto, o planejamento de capacidade precisa ser realizado pelo provedor independentemente do modelo de implantação ou do tipo de serviço oferecido aos usuários.

Já do ponto de vista do usuário do serviço, o planejamento de capacidade envolve o correto dimensionamento dos recursos virtuais a serem alocados às diferentes aplicações que serão executadas na nuvem. Nesse caso, o planejamento de capacidade é necessário apenas em nuvens da modalidade IaaS, uma vez que em nuvens das modalidades PaaS e SaaS a alocação dos recursos às aplicações é feita automaticamente, com parte dos benefícios oferecidas pelo provedor do serviço, ficando totalmente transparente ao usuário. 

Nos dois casos, um planejamento inadequado dos recurso pode levar a situações indesejadas. No caso do provedor do serviço, um mal planejamento pode resultar tanto em uma infraestrutura subutilizada, na qual a quantidade de recursos físicos disponíveis é muito superior à demanda, quanto em uma infraestrutura saturada, com qual o provedor não consegue garantir os requisitos mínimos de qualidade prometidos aos usuários em seus acordos de nível de serviço. Ambas as situações podem acarretar em prejuízos financeiros, seja pelo desperdício de recursos, seja pela insatisfação dos usuários.

O problema do mal planejamento de recursos é ainda mais sutil no caso do usuário do serviço, que é o foco deste trabalho. Isso porque a provisão de recursos virtuais além do estritamente necessário costuma passar despercebida à maioria dos usuários de nuvens IaaS, que muitas vezes focam apenas no monitoramento do desempenho de suas aplicações, e não no nível de utilização do recursos que lhes foram alocados. Esse problema existe mesmo diante da utilização de serviços IaaS que oferecem escalabilidade automática de aplicações, como é o caso dos serviços Amazon CloudWatch \cite{amazon_watch} e RightScale \cite{rightscale}. A razão é que esses tipos de serviço são tipicamente restritos à escalabilidade horizontal da aplicação, na qual alguns dos componentes da aplicação são executados paralelamente em um conjunto de duas ou mais máquinas virtuais do mesmo tipo. Dependendo do nível de utilização das máquinas virtuais em uso, mais máquinas virtuais podem ser adicionadas ao grupo ou algumas das máquinas virtuais em uso podem ser desligadas. De qualquer maneira, a variação da capacidade de recursos alocados à aplicação fica limitada à variação da quantidade de máquinas virtuais presentes em grupos de máquinas homogêneos, cujos tipos também precisam ser cuidadosamente determinados pelo usuário como parte do processo de planejamento de capacidade. 

A próxima seção apresenta e analisa criticamente alguns dos principais trabalhos propostos para apoiar usuários de serviços de nuvem IaaS a melhor planejar a capacidade dos recursos a serem alocados a suas aplicações.

\section{Abordagens para Apoiar o Planejamento de Capacidade em Nuvens IaaS}

Apresentamos  neste capítulo alguns trabalhos cujos objetivos estão alinhados 
com a ideia da avaliação de desempenho de aplicações executadas em ambientes de
computação em nuvem. Esses trabalhos estão agrupados de acordo com a abordagem utilizada para a realização da avaliação de desempenho. São duas abordagens, a primeira é a abordagem preditiva, nesta, os trabalhos não executam diretamente a aplicação alvo no ambiente onde se deseja implantá-la. Já a segunda abordagem é a empírica, nesta as aplicações alvo são implantadas na nuvem e então submetidas a testes de carga. 


Cada abordagem será apresentada em uma subseção, bem como os diversos trabalhos da área que serão resumidos separadamente. Ao final da subseção, será apresentada uma análise crítica que será baseada nos critérios descritos acima.

\subsection{Abordagem Preditiva}

As ferramentas que serão apresentadas nesta seção têm em comum o fato de
indicarem a configuração de implantação na nuvem sem executarem avaliações de
desempenho diretamente na aplicação que será migrada para a nuvem. Dessa forma,
elas prevêem a configuração de implantação, por isso estão sendo chamadas de
abordagens preditivas. Muitas das soluções de abordagem preditiva realizam as
suas predições após a caracterização da performance dos recurso da nuvem, o que
é realizado através da execução de \textit{benchmarks}. Já o
\textit{CloudProphet}, apresentado em~\cite{li2011cloudprophet}, coleta como a
aplicação alvo faz uso dos recursos computacionais em um ambiente controlado e
então repete essa utilização em uma nuvem candidata. Dessa forma, não necessita
caracterizar a performance dos recursos da nuvem.

A seguir, apresentaremos seis trabalhos que se destacam na avaliação de
desempenho de aplicações na nuvem usando uma abordagem preditiva, são
eles:~\cite{cloudharmony},~\cite{malkowski2010cloudxplor},~\cite{li2011},~\cite{jung2013cloudadvisor},~\cite{fittkau2012cdosim}
e~\cite{li2011cloudprophet}.

\subsubsection{Cloud Harmony}
O projeto {\em CloudHarmony}, cujo
objetivo é ``tornar-se a principal fonte independente, imparcial e útil de
métricas de desempenho de provedores de nuvem''~\cite{cloudharmony}, agrega
dados de testes de desempenho realizados desde 2009 em mais de 60 provedores de
nuvem. Conforme descrito em~\cite{cunha2012ambiente}, além do histórico das
avaliações, o {\em CloudHarmony} disponibiliza uma ferramenta para executar novas avaliações de desempenho a qualquer momento, denominada \textit{Cloud Speed Test},\footnote{\url{http://cloudharmony.com/speedtest}}, a qual permite realizar quatro tipos de teste:

\begin{description}
  \item[\em Download a few large files] --- objetiva determinar o melhor provedor
  para descarregar arquivos grandes, sendo útil para aplicações como {\em video
  streaming};
  \item[\em Download many small files] --- objetiva determinar o melhor provedor
  para descarregar arquivos pequenos, podendo ser útil para hospedar uma página
  web, por exemplo;
  \item[\em Upload] --- útil para avaliar serviços que serão utilizados para
  envio de arquivos;
  \item[\em Test network latency] --- a latência afeta o tempo de resposta da
  aplicação e geralmente está relacionada com a região de onde o teste está
  partindo.
\end{description}

Os resultados disponibilizados pelo {\em CloudHarmony} têm como pontos fortes a
grande quantidade de dados de testes de desempenho disponíveis, além da possibilidade do cliente da
nuvem poder executar novos testes a qualquer tempo. Por outro lado, os testes estão limitados àqueles implementados pela ferramenta de teste, não podendo ser facilmente modificados para contemplar novas métricas ou cenários de avaliação.

\subsubsection{{\em CloudXplor}}

{\em CloudXplor}~\cite{malkowski2010cloudxplor} é uma ferramenta para
planejamento de configuração de recursos da nuvem baseada em dados empíricos. A ferramenta
foi desenvolvida tomando como base um modelo de planejamento de configuração de
recursos de Tecnologia da Informação (TI), com foco explícito em aspectos econômicos.
Por essa razão, a ferramenta se utiliza de acordos de nível de serviço baseados na relação do custo
da infraestrutura de TI com o valor dos recursos do provedor do serviço. Esse
valor será maior quando o tempo de resposta da aplicação for plenamente atendido
pelo provedor do serviço, e vai diminuindo à medida em que esse tempo de resposta
não é alcançado.

Os dados empíricos precisam ser coletados, previamente, através da execução de diversos experimentos de avaliação de desempenho. Esses dados
são compostos por métricas de sistema (uso de CPU, memória utilizada, tráfego na rede e E/S de disco) e métricas de mais alto nível (tempo de resposta e \textit{throughput}). Após a coleta, os dados dos experimentos são submetidos e analisados pela ferramenta, utilizando um de seus quatro módulos: análise de tempo de
resposta, análise de \textit{throughput}, análise do valor agregado e do custo,
e análise do lucro. Cada um desses módulos filtra os dados, fazendo uso apenas
das informações necessárias para a execução da sua análise. Após a análise dos
 dados, a ferramenta pode ser utilizada para produzir gráficos que ilustram o comportamento da aplicação ao se variar
parâmetros como carga de trabalho e configuração dos componentes da aplicação.


%A ferramenta proposta no trabalho faz uso de dados previamente coletados
%e só então realiza a análise do desempenho de aplicações na nuvem
%em diversos recursos. Por isso deixa a cargo do cliente da nuvem todas as
%atividades para a avaliação de desempenho, pois depende dos dados empíricos que
%são coletados após a avaliação. O CloudXplor é capaz de plotar um gráfico com o
%comportamento da aplicação à medida em que é exposta a variação na demanda, do
%custo e do valor, mas só o faz por que o cliente da nuvem executou cada uma das
%avaliações manualmente. Da mesma forma a variação na demanda vai depender do
%conjunto de avaliações executadas pelo cliente da nuvem, a ferramenta apenas
%plota gráficos do que foi executado pelo cliente. Com relação a definição de
%cenários de avaliação e a definição de parâmetros de desempenho, a ferramenta
%não dá nenhum suporte, uma vez que os cenários dependem das avaliações
%realizadas e que os módulos de análise de custo, valor, tempo de resposta e
% {\em throughput} possuem uma lista dos parâmetros de desempenho possíveis de
%avaliação.


\subsubsection{CloudCmp}
\citeonline{li2011} apresentam uma ferramenta para apoiar a avaliação e a comparação do
desempenho e do custo dos recursos e serviços de diversos provedores de nuvem
pública, de modo a auxiliar o cliente da nuvem a escolher o provedor mais adequado para a sua
aplicação. Essa ferramenta, denominada {\em CloudCmp}, analisa
os serviços de elasticidade, persistência de dados e rede oferecidos pelos provedores
de interesse, com base em resultados previamente coletados a partir da execução de diversos
{\em benchmarks}: uma versão modificada do {\em SPECjvm2008}~\cite{SPECjvm2008}
para avaliar a característica de elasticidade do provedor; um cliente Java para avaliar os serviços de
armazenamento e persistência de dados; e as ferramentas {\em iperf}\footnote{http://iperf.sourceforge.net. }
e {\em ping} para avaliar os serviços de rede. Após a fase inicial da coleta dos dados, a ferramenta pode ser utilizada para gerar gráficos que auxiliem o
cliente da nuvem a comparar o desempenho dos recursos de cada um
dos provedores nos quais as avaliações foram realizadas, que assim poderá escolher o provedor e os recursos mais apropriados para as necessidades e demandas específicas de suas aplicações. 

%Os gráficos gerados nos
%experimentos apresentados neste trabalho foram comparados com os gerados após o
%estudo do comportamento de três aplicações simples implantadas na nuvem. Essas
%comparações mostraram que as previsões do CloudCmp refletiram o comportamento
%das aplicações testadas.

Segundo \citeonline{li2011}, até a época do trabalho não houve
nenhum provedor de nuvem que se destacasse com relação aos demais. Outra constatação foi de que os resultados
obtidos a partir da execução dos {\em benchmarks} em cada provedor apenas refletiam o momento em que foram coletados, uma vez que a estrutura utilizada pelos provedores para hospedar seus serviços sofre frequentes modificações e a demanda por seus recursos computacionais é bastante variável.

%Como essa solução compara serviços e recursos da nuvem através da análise de
% dados previamente coletados a partir da execução de diferentes {\em benchmarks}, a
%escolha dos provedores e recursos mais apropriados para uma determinada
% aplicação só será eficaz se a aplicação utilizar os recursos da nuvem de forma semelhante à dos {\em benchmarks} avaliados. Além disso, a ferramenta não oferece suporte à execução de novos cenários de avaliação na nuvem, estando limitada àqueles previamente definidos para os respectivos {\em bechmarks}.

\subsubsection{CloudAdvisor}
O trabalho apresentado em~\cite{jung2013cloudadvisor} ``introduz uma nova plataforma de recomendação de nuvem, chamada {\em CloudAdvisor}''. Essa plataforma destina-se a auxiliar o seu usuário na tarefa de capturar as implicações monetárias e financeiras das configurações de implantação das suas aplicações. Para recomendar a configuração, a platforma recebe como entrada parâmetros de configuração de alto nível como, orçamento, expectativa de performance e economia de energia, os quais estão limitados a uma escala discreta que vai de 0 até 10, onde 0 significa baixa influência e 10 significa alta influência. Uma vez informados os parâmetros de configuração, o {\em CloudAdvisor} irá caracterizar a performance da aplicação alvo em termos de uso dos recursos computacionais e em seguida executará o {\em benchmark} {\em CloudMeter},~\cite{jung2013cloudadvisor}, nas nuvens candidatas, a fim de caracterizar a performance dos recursos dessas nuvens.

Para ilustrar o uso do {\em CloudAdvisor}, os autores implantaram a solução em servidores locais e em três provedores de nuvem pública, foram eles: Windows Azure~\cite{azure}, Rackspace~\cite{rackspace}, and Amazon EC2~\cite{ec2}. Como resultado, foi observado que a taxa de erro da configuração para uma determinada carga foi de 10~\%. No entanto, quando o usuário do ambiente escolhe parâmetros de configuração extremos (por exemplo, configurar o máximo de orçamento, de economia de energia e de nível de performance), essa taxa de erro elevou para 18~\%.

Os autores concluem que o usuário da solução pode explorar diversas opções de configuração da sua aplicação na nuvem utilizando uma interface amigável e sem a necessidade de informar detalhes específicos de configuração. Além disso, mostraram que é possível utilizar uma técnica de caracterização de performance, para uma dada carga, baseada na execução de {\em benchmarks} em nuvens candidatas.

\subsubsection{CDOSim}\label{subsec:CDOSim}
Uma solução para o desafio da escolha da configuração de implantação de uma aplicação na nuvem foi apresentada em~\cite{fittkau2012cdosim} com o nome de \textit{CDOSim}. Essa solução auxilia o usuário no processo de escolha do que os autores chamaram de opção de implantação na nuvem --- do inglês \textit{Cloud Deployment Option} (CDO)---, uma vez que a análise manual das ``potenciais CDOs é intratável, custosa e consome tempo, devido à heterogeneidade dos ambientes de nuvem'',~\cite{fittkau2012cdosim}. Para a escolha das CDOs, são realizadas simulações que se baseiam no custo e nas propriedades de performance de cada CDO. O custo é informado pelo provedor de nuvem, já a caracterização da performance é realizada através da execução de um \textit{benchmark} para medir a quantidade de \textit{mega integer plus instructions per second} (MIPIPS)~\cite{fittkau2012cdosim}, por opção de configuração. O código desse \textit{benchmark} deve ser gerado para cada linguagem de programação utilizada pela aplicação alvo. Esta, por sua vez, deve passar por um processo de engenharia reversa para modelos KDM, que é descrito em~\cite{perez2011knowledge}, para que o simulator consiga escolher a opção de configuração mais adequada.

Para ilustrar o uso da solução, os autores executaram três tipos de experimentos, um para validar o uso de MIPIPS para caracterizar a performance das opções de configuração, outro para comparar os resultados da simulação com dados reais, e um último experimento para verificar a possibilidade da predição da performance de um provedor de nuvem com base nos dados de outro provedor de nuvem. Esses experimentos foram conduzidos em dois ambientes de nuvem, sendo uma pública, a Amazon EC2, e outra privada. Na nuvem pública foi evidenciado que os valores de MIPIPS dependem da região onde o \textit{benchmark} foi realizado e da carga sobre a máquina física que hospeda a máquina virtual. Já na comparação dos resultados da simulação com os dados reais, os autores mostraram que a taxa de error da utilização de CPU simulada com a utilização de CPU medida, chegou a 30,86~\%. Contudo, a taxa de erro médio global foi abaixo de 22,75~\%, portanto abaixo do limiar estabelecido pelos autores que era de 30~\%. Já a predição da performance de uma instância da Amazon EC2 a partir da performance de uma instância da nuvem privada gerou 15,76~\% como taxa de erro global.

Finalmente, os autores concluem que a simulação pode auxiliar no processo de escolha da opção de implantação com maior performance e menor custo e que os resultados da simulação são razoavelmente próximos dos valores reais.

\subsubsection{CloudProphet}\label{subsec:CloudProphet}
Em \cite{li2011cloudprophet} os autores apresentam o \textit{CloudProphet}, um sistema de
predição de desempenho de aplicações em ambiente de nuvem computacional baseado 
na metodologia de ``rastrear e reproduzir'' (\textit{trace and replay}).

O \textit{CloudProphet} não testa a aplicação do cliente de fato no ambiente de nuvem. De
modo contrário, ele injeta na implantação original da aplicação um módulo que 
registra um rastreamento detalhado dos eventos de utilização de recursos de CPU,
armazenamento e rede em cada componente da aplicação durante um período de 
execução habitual em seu ambiente de produção.

Em um passo seguinte, outro módulo faz uma extração das relações de dependência 
entre os eventos coletados, ordenando as transações executadas nos diversos 
componentes.

O terceiro passo é a reprodução dos eventos coletados durante a fase de 
rastreamento. Essa reprodução consiste fazer com que o ambiente de nuvem 
computacional que se deseja avaliar execute as transações representadas nos dados
do rastreamento a partir de requisições que partem de clientes simulados.
   
O objetivo do \textit{CloudBench}, segundo os autores é eliminar o custo e o trabalho 
envolvidos na migração da aplicação real para a nuvem para a execução de testes 
antes que seja de fato tomada a decisão em favor dessa migração.

Os autores argumentam que a simples implantação da aplicação no ambiente de um
serviço de nuvem computacional já incorre em custos, que podem ser altos a 
depender do tamanho ou da arquitetura da aplicação. Além disso, a tarefa de
migração pode ser bastante trabalhosa conforme o número e a diversidade dos 
componentes da aplicação, que podem acarretar dificuldades de configuração e
compatibilidade no novo ambiente.

\subsection{Abordagem Empírica}
As ferramentas que serão apresentadas nesta seção têm em comum o fato de utilizarem como aplicação alvo a prória aplicação que se deseja implantar na nuvem. Portanto, é preciso inicialmente realizar uma implantação na nuvem para que seja dado início ao processo de análise de desempenho. Por isso, cada ferramenta oferece um mecanismo para que o seu usuário possa definir como a aplicação deve ser implantada e configurada. Além da definição da aplicação e dos recursos da nuvem que serão utilizados, essas ferramentas também permitem que sejam definidas a demanda que será imposta a cada aplicação e o acordo de nível de serviço. Dessa forma, é possível definir, por exemplo, o número de usuários simultâneos e o tempo de resposta esperado para uma transação.

Uma vez que que as ferramentas que realizam a abordagem empírica possibilitam ao usuário muita liberdade na definição da aplicação, demanda, recurso da nuvem e SLA, essas soluções têm o mais alto grau de completude. Além disso, como faz-se uso da própria aplicação alvo para a avaliação de desempenho, a acurácia dos resultados apresentados pelas ferramentas é a mais elevada. A seguir, apresentaremos quatro trabalhos que se destacam na avaliação de desempenho de aplicações na nuvem usando a abordagem empírica, são eles:~\cite{jayasinghe2012},~\cite{silva2013cloudbench},~\cite{cunhacloud} e~\cite{scheuner2014cloud}.

\subsubsection{Expertus}
Devido à complexidade e às implicações da escolha da configuração para a implantação de uma aplicação na nuvem, em~\cite{jayasinghe2012} os autores apresentam o \textit{Expertus}, que é descrito como ``um framework flexível de geração de código para automatizar testes de performance de aplicações distribuídas em nuvem de infraestrutura''. Essa geração automática de código é realizada a partir de {\em templates} especificados na forma de documentos XML~\cite{jayasinghe2012}. Os templates utilizados nas avaliações de desempenho devem ser escritos pelo usuário e servem de entrada para o ambiente, que realiza diversas transformações nessa entrada até a forma de \textit{shell scripts}. Esses \textit{scripts}, por fim, possuem os comandos para a configuração da avaliação de desempenho na aplicação alvo.

Como demonstração da usabilidade da ferramenta, os autores apresentaram em~\cite{jayasinghe2012} resultados de experimentos realizados com duas aplicações alvo. Cada uma das aplicações foi avaliada com duas opções de sistemas de gerenciamento de bancos de dados, o que demonstrou também como diferentes opções de configuração poderiam ser utilizadas nas aplicações. Além da demonstração da usabilidade, os autores realizaram experimentos para evidenciar a magniture e tipos de \textit{scripts} que podem ser gerados pela ferramenta. Como exemplo da magnitute, para a realização de experimentos com 48 nós, o total de linhas de scripts geradas pelo \textit{Expertus} girou em torno de 15 mil. Por fim, os autores demonstraram o que chamaram de ``riqueza da ferramenta'', que foi comprovada através da execução de experimentos em 5 nuvens (por exemplo, Amazon EC2 e Open Cirrus~\cite{avetisyan2010open}).

Dessa forma, pode-se concluir que a ferramenta apresentada minimiza a ocorrência de falhas humanas na avaliação de desempenho de uma aplicação implantada em diversos nós. Além disso, os mesmo experimentos podem ser repetidos em diferentes provedores de nuvem pública. De modo que mais cenários de implantação podem ser considerados para a escolha do mais adequado para a aplicação.

\subsubsection{CloudBench}
\cite{silva2013cloudbench} descreve o \textit{CloudBench} como um arcabouço para automação da avaliação de desempenho de ambientes de nuvem computacional sob o modelo IaaS. As abstrações apresentadas neste trabalho permitem que um experimento seja especificado através de uma lista de diretivas as quais descrevem os itens que compõem o experimento. São exemplos desses itens, objetos como a aplicação alvo, as instâncias de máquinas virtuais utilizadas, e as métricas de desempenho que são tanto relativas à aplicação alvo, quanto ao serviço do provedor de nuvem (por exemplo, latência de provisionamento).
%Para a análise dos resultados, o ambiente coleta dados relativos às métricas de desempenho observadas, através %da abstração de experimentos, aplicações e máquinas virtuais.
%Além disso, o ambiente prevê métricas que dizem respeito não só à aplicação, mas também ao provedor, como latência de provisionamento.
%, bem como a execução dos 
%testes e a coleta de dados relativos às métricas de desempenho observadas, através 
%da abstração de experimentos, aplicações e máquinas virtuais.   

Para a realização dos experimentos, o \textit{CloudBench} faz a implantação automática da aplicação a ser executada para efeito de testes. Portanto, o acompanhamento é realizado desde a criação da máquina virtual no ambiente até a coleta dos dados de desempenho e desligamento das máquinas. Essas características fazem do CloudBench uma ferramenta muito poderosa para a
automação de testes e coleta de dados para análise das execuções. Suas ferramentas
de monitoramento fornecem informações com grandes níveis de detalhamento a respeito
de cada componente implantado e usado nos testes, proporcionando excelente embasamento
para a tomada de decisão.

Entretanto, embora o CloudBench tenha um escopo de solução muito mais amplo, 
voltado para a avaliação de desempenho tanto da aplicação do cliente como do 
provisionamento de máquinas pelo provedor, seu alvo no momento da execução de 
testes está restrito a \textit{benchmarks} pré-definidos, não permitindo a execução de 
uma aplicação real no ambiente testado.

%Neste sentido, o arcabouço que propomos com este trabalho se diferencia pelo fato
%de ser agnóstico em relação à aplicação que deverá ser testada, assim como quanto 
%às métricas que a ela concernem, conferindo ao usuário da solução a oportunidade 
%de avaliar o comportamento da aplicação de seu interesse implantada no ambiente
%pretendido e sob a perspectiva que lhe for mais conveniente, inclusive em termos
%de arquitetura de implantação. 

\subsubsection{Cloud Crawler}
Este trabalho apresenta um ambiente programável para apoiar os usuários de nuvens IaaS na realização de testes automáticos de desempenho de aplicações na nuvem. As principais contribuições do ambiente são: a linguagem declarativa {\em Crawl}, com a qual os usuários podem especificar, através de uma notação simples e de alto nível de abstração, uma grande variedade de cenários de avaliação de desempenho de uma aplicação na nuvem; e o motor de execução {\em Crawler}, que automaticamente executa e coleta os resultados dos cenários descritos em {\em Crawl} em um ou mais provedores. Essas duas ferramentas são denominadas conjuntamente de {\em Cloud Crawler}~\cite{cunhacloud}.

Para iniciar os testes de desempenho de uma aplicação através do ambiente {\em Cloud Crawler}, os componentes dessa aplicação precisam ser declarados em um \textit{script} da linguagem {\em Crawl}. Compõem esse \textit{script Crawl}, por exemplo, o provedor de nuvem, os tipos de máquinas virtuais e as máquinas virtuais que serão utilizadas nas avaliações, além disso, métricas de desempenho e a demanda imposta à aplicação também irão compor o cenário de avaliação que é declarado no \textit{script Crawl}. Finalizada essa etapa de declaração, o usuário do ambiente irá submeter o \textit{script crawl} para o motor de execução {\em Crawler}. Esse motor irá iniciar todas as máquinas virtuais, caso seja necessário, irá proceder com a modificação do tipo de máquina virtual, de acordo com o que estiver declarado. Após a inicialização de cada máquina virtual, o motor pode executar alguma configuração nessa máquina, por exemplo, a configuração do endereço ip de um banco de dados, ou a configuração do total de memória utilizado por uma máquina virtual java. Todas as configurações necessárias para a aplicação executar na nuvem devem estar declaradas no {\em script Crawl} que foi submetido para o motor. Quando a última máquina virtual é configurada, o motor {\em Crawler} executa um por um os cenários de avaliação, com suas respectivas demandas, e ao mesmo tempo coleta as métricas de desempenho especificadas. As métricas de desempenho podem ser tanto métricas de sistema, como percentual de CPU utilizado e de memória RAM, quanto métricas de aplicação, como o tempo de resposta de uma aplicação WEB.

A fase de mapeamento dos componentes da aplicação é realizada apenas uma vez, enquanto que a submissão para o motor de execução pode ser repetida ao critério do usuário. Ambientes como o {\em Cloud Crawler} permitem que os seus usuários testem suas aplicaçãoes em difirentes cenário de implantação e possibilitam que o mesmo entenda o comportamento da sua aplicação à medida em que ela é submetida a diferentes demandas e implantada em diferentes configurações, porém, a qualidade da avaliação de desempenho dependerá da qualidade dos cenários de testes que os usuários declararem, uma vez que o ambiente não decide qual será a nova configuração testada. O ambiente apenas segue aquilo que foi declarado pelo usuário.

\subsubsection{Cloud WorkBench}
Uma vez que a escolha da infraestrutura computacional ótima para hospedar uma determinada aplicação na nuvem se trata de uma tarefa que ``exige a avaliação de custos e performances de diferentes combinações de configurações''~\cite{scheuner2014cloud}. Onde os autores propõem uma arquitetura e uma implementação concreta dessa arquitetura para automatizar a realização de avaliações em serviços da nuvem. O {\em Cloud WorkBench}, nome dado à solução apresentada neste trabalho, adota noções de Infraestrutura como Código, do inglês {\em Infrastructure-as-Code} (IaC)~\cite{huttermann2012devops}, para a realização dessas avalições. Dessa forma, as ações necessárias para o provisionamento dos recursos utilizados pela aplicação encontram-se todas codificadas.

Para ilustrar o uso do \textit{Cloud WorkBench}, foi realizado um pequeno experimento para avaliar a velocidade de escrita sequencial em disco de recursos da nuvem. Nesse experimento, foram utilizados três perfis de recursos computacionais da Amazon EC2 na região Irlanda (\textit{eu-west-1}), em servidores utilizando o sistema operacional Ubuntu 14.04. Para cada perfil de recurso, foram realizadas entre 8 e 12 execuções do benchmark FIO\footnote{http://git.kernel.org/cgit/linux/kernel/git/axboe/fio.git} 2.1.10. Conforme descrito em~\cite{scheuner2014cloud}, o experimento evidenciou que há diferenças na performance dos perfis de recursos utilizados. Esse diferença poderia se refletir na performance de uma aplicação que fizesse muita escrita em disco.

Após a análise dos resultados, os autores concluem que o \textit{Cloud WorkBench} suporta a realização de experimentos em nuvens de infraestrutura, e que toda a complexidade da configuração do ambiente pode ficar codificada. O que diminui a ocorrência de erros decorrentes de eventuais intervenções manuais.

\subsection{Análise Crítica}

\subsubsection{Critérios de Comparação}

Para apoiar a análise crítica de cada trabalho e suas abordagens, definimos um conjunto de critérios a partir dos quais poderemos tanto descrever, quanto comparar as soluções existentes de apoio a avaliação de desempenho do ponto de vista do usuário. Seguem os critérios:

\begin{enumerate}
  \item \textbf{Completude da solução}
  \begin{enumerate}
    \item \textbf{Definição da aplicação} --- flexibilidade da solução para
    definir a aplicação a ser avaliada;
    \item \textbf{Definição da demanda} --- flexibilidade da solução para
    definir os níveis de carga de trabalho sobre os quais a aplicação será
    submetida;
	\item \textbf{Definição dos recursos da nuvem} --- flexibilidade da solução
	para definir as configurações de recursos do provedor sobre as quais a aplicação
	será executada;
	\item \textbf{Definição do acordo de nível de serviço (SLA)} --- flexibilidade
	da solução para definir o SLA desejado.
  \end{enumerate}
  \item \textbf{Efetividade da solução}  
  \begin{enumerate}
    \item \textbf{Eficiência} --- tempo e custo necessários para a solução
    resolver o problema;
    \item \textbf{Acurácia} --- confiabilidade das respostas oferecidas pela
    solução;
	\item \textbf{Complexidade} --- grau de complexidade/esforço exigido do usuário
	da solução.
  \end{enumerate}
\end{enumerate}


\subsubsection{Abordagem Preditiva}
As soluções de abordagem preditiva possuem desempenho bastante variados nos
critérios de desempenho estabelecidos no início deste capítulo, porém elas
acabam convergindo na efetividade, que comproente eficiência, acurácia e
complexidade. No critério de eficência, as soluções se destacam, possuem
alta eficiência, uma vez que não requerem a alocação de recursos de nuvem para
realizarem as predições. No entanto, o \textit{CloudProphet} é a única solução
de abordagem preditiva com baixa eficiência, pois requer a alocações de recursos da nuvem para
avaliar todas as demandas e configurações. Já com relação à acurácia, as
soluções tem um desempenho moderado, com distinção para a solução apresentada
em~\cite{fittkau2012cdosim}, que possui baixa acurácia, conforme fica
evidenciado nos resultados apresentados na subseção~\ref{subsec:CDOSim}.
Finalmente, no que diz respeito à complexidade, a solução com menor
complexidade, portanto com maior destaque, é a \textit{CloudHarmony} a qual
permite que os testes sejam iniciados e que as pesquisas de resultados
anteriores sejam realizadas através de uma interface amigável, sem a necessidade de intervenções do usuário.
Já as demais possuem complexidade moderada.
 
Ainda com base nos critérios de avaliação, as soluções apresentadas
em~\cite{malkowski2010cloudxplor,cloudharmony} possuem baixa completude, pois
não permitem que sejam definidos aplicação, demanda, recurso da nuvem e SLA. Já
o \textit{CloudAdvisor} apresentando em~\cite{jung2013cloudadvisor}, permite a
definição da aplicação, e da carga, porém não permite a escolha do SLA e não
faz referência a respeito do uso de nuvens públicas diferentes das apresentadas
nos resultados. Dessa forma, com relação à completude, o \textit{CloudAdvisor}
tem um desempenho moderado. Das abordagens preditivas, o \textit{CloudProphet}
é a solução que se destaque no critério da completude, pois permite que o
usuário defina aplicação, demanda, recurso da nuvem e SLA desejado.

\subsubsection{Abordagem Empírica}
As soluções de abordagem empírica, por sua vez, possuem desempenhos muito
parecidos tanto na completude, quanto na efetividade. Com relação à
completude, que abrange a definição da aplicação, da demanda, dos recursos da
nuvem e do SLA, as soluções possuem alto desempenho. Pois permitem muita
liberdade nessas definições. Por exemplo, na solução apresentada
em~\cite{cunhacloud}, o usuário pode definir toda a pilha de componentes
da aplicação alvo, todos cenários utilizados na avaliação de desempenho, as
demandas que serão submetidas a cada um dos cenários e o critério que define se
o cenário suportou a demanda, ou seja, o SLA. Da mesma forma, as soluções
apresentadas em ~\cite{jayasinghe2012,silva2013cloudbench,scheuner2014cloud},
possuem os seus mecanismos para a realização dessas definições.

Apesar das soluções de abordagem empírica terem destaque na completude, no que
diz respeito à efetividade, que abrange eficiência, acurácia e complexidade,
essas soluções possuem desempenho moderado. No critério da eficiência, essas
ferramentas deveriam fazer uso de resultados anteriores para evitar a execução
de testes que claramente poderiam ser evitados. Por exemplo, em uma situação na qual uma demanda é submetida à aplicação que está sendo
executada em uma máquina virtual com baixo poder computacional, seria coerente
afirmar que essa mesma demanda pode ser atendida por máquinas com perfis
computacionais mais robustos. Devido à essa necessidade de executar os testes em
cada uma das configurações, sem fazer uso de resultados anteriores, as soluções
de abordagem empírica possuem baixa eficiência. Por outro lado, como irão
executar de fato a aplicação e submetê-la à demanda especificada, essas soluções
possuem alta acurácia. Já no que diz respeito à complexidade, o desempenho é
considerado moderado. Uma vez que cada trabalho faz uso de uma estratégia
de uso particular, e que as experiências anteriores do usuário irão ser
determinantes na percepção da complexidade. Esse usuário precisa se adaptar à
sintaxe de cada solução, e eventualmente, configurar imagens contendo os
componentes da aplicação que será avaliada.







