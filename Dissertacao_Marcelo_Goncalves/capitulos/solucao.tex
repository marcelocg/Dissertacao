\chapter[Esquema de Solução]{Esquema de Solução}
% ----------------------------------------------------------

***** DESCONSIDERAR POR ENQUANTO *****

O objetivo deste trabalho é estudar heurísticas de preenchimento da matriz P 
descrita no capítulo anterior sem necessariamente ter que executar de fato todos
os testes necessários. Esse preenchimento deverá ser feito, assim, por meio de 
um motor de predições que executará uma ou mais estratégias para preencher a 
matriz P com resultados calculados a partir dos dados de entrada.

Consideraremos como dados de entrada os resultados de uma ou mais execuções 
reais para um ou mais cenários de teste. A precisão dos resultados obtidos pelo 
motor de predição vai depender da qualidade da estratégia escolhida e da 
quantidade de dados de entrada. Quanto mais diversificadas quanto aos cenários a
que forem aplicadas e quanto maior o número de execuções reais usadas para 
alimentar inicialmente o motor de predições, mais preciso tenderá a ser o 
resultado da predição, porém mais caro se tornará o processo, uma vez que os 
testes executados no ambiente de nuvem incorrem em custo financeiro.

Apresentaremos um motor capaz de executar heurísticas de predição e um arcabouço
de implementação dessas heurísticas, de forma que nova inteligência de predição 
possa ser agregada ao trabalho futuramente. Apresentaremos também, como forma de
validar a proposta do arcabouço e do motor de execuções, duas heurísticas para 
geração da matriz P preenchida e sugestão da configuração de menor custo para 
executar a aplicação alvo em ambiente de nuvem de infraestrutura.

