\chapter[Formalização do Problema]{Formalização do Problema}
% ----------------------------------------------------------
A fim de que uma análise criteriosa possa ser feita, apresentamos a seguir um 
conjunto de definições e terminologias que possam basear o entendimento da 
construção do trabalho e também da avaliação dos resultados e de sua eficiência 
e eficácia. Os conceitos aqui explicitados são tomados de forma a permitir um 
estudo agnóstico quanto a aplicações, plataformas e provedores utilizados 
durante a execução das ferramentas desenvolvidas neste trabalho.

Apresentamos também um formalismo que visa à generalização da metodologia 
utilizada neste trabalho e a permitir a melhor descrição do raciocínio lógico 
envolvido no desenvolvimento das heurísticas criadas.

\section{Definições e Terminologias}
Apresentamos a seguir as definições que permeiam o conhecimento necessário para 
a análise dos problemas estudados e soluções propostas. Mostramos também as 
terminologias ou nomenclaturas que criamos para designar esses conceitos a fim 
de facilitar a comunicação e o entendimento por parte do leitor.

\subsection{Aplicação sob Teste}
A Aplicação sob Teste é um sistema computacional, possivelmente implementado em 
arquitetura multicamadas, para o qual se deseja observar o comportamento em um 
ambiente de computação em nuvem e ao estão ligadas uma ou mais Métricas de 
Desempenho.

\subsection{Métrica de Desempenho}
Uma característica ou comportamento mensurável de forma automatizada e 
comparável a um Valor de Referência capaz de indicar o grau de sucesso de uma 
execução da Aplicação. Ex. tempo de resposta, quadros por segundo, etc. Métricas 
podem ser minimizáveis ou maximizáveis, a depender do objetivo da métrica quanto 
ao resultado desejado. Por exemplo, “tempo de resposta” é uma métrica 
minimizável, uma vez que geralmente se deseja que uma Aplicação responda a uma 
requisição com o menor tempo de resposta possível nos resultados. Contrariamente, 
uma métrica “quadros renderizados por segundo”, no domínio da computação gráfica, 
é uma métrica maximizável, pois quanto mais quadros são renderizados por unidade 
de tempo, maior a qualidade percebida pelo usuário.

\subsection{Valor de Referência}
Um valor predefinido como minimamente aceitável como resultado apresentado por 
uma Métrica após uma Execução da Aplicação sob Teste. Este valor, também 
referenciado neste trabalho como SLA (Service Level Agreement), serve como base 
de comparação para que as Heurísticas desenvolvidas e apresentadas saibam se a 
Aplicação é capaz de ser executada sobre um determinado arranjo de máquinas 
virtuais e sob uma determinada Carga de Trabalho a ela imposta.

\subsection{Provedor}
Consideramos neste trabalho a figura do provedor como representando uma empresa 
que fornece infraestrutura computacional como serviço cobrado financeiramente 
por fração de tempo de utilização. Alguns provedores fornecem conjuntamente a 
modalidade de plataforma como serviço. Nós, porém, não estamos considerando essa 
modalidade neste trabalho, interessando-nos apenas os serviços de infraestrutura, 
notadamente a disponibilização de máquinas virtuais.

\subsection{Tipos de Máquinas Virtuais}
Provedores costumam classificar as máquinas virtuais fornecidas conforme suas 
características, de modo a manter uma linha de produtos discreta e finita. 
Normalmente essa classificação se dá em termos de quantidade de memória RAM, 
quantidade de espaço em disco e capacidade computacional, neste caso, quer seja 
em termos relativos a um valor padrão tomado como base, quer seja em termos 
absolutos, como número de CPUs virtuais.

\subsection{Categorias de Máquinas Virtuais}
Tipos de Máquinas Virtuais podem ser agrupados em Categorias, conforme suas 
características físicas, plataforma e/ou arquitetura de hardware e a natureza do 
uso a que se destinam. Dentro de uma mesma Categoria, os Tipos de Máquinas 
Virtuais variam apenas na quantidade de cada um dos recursos especificados para 
a Categoria e no preço cobrado pelo uso das máquinas virtuais.

Como exemplo, podemos citar uma Categoria de máquinas destinadas a armazenamento 
de arquivos, onde as máquinas devem privilegiar o espaço de armazenamento em 
massa. Dentro dessa categoria, a principal diferença entre os Tipos de Máquinas 
Virtuais se dá em função da quantidade de espaço em disco disponibilizado, 
enquanto características como memória RAM e CPU teriam pequenas variações. 
Outras Categorias podem enfatizar o consumo de banda de rede ou processamento 
paralelo de alto desempenho.

\subsection{Configurações}
Chamamos de Configuração um conjunto de máquinas virtuais pertencentes ao mesmo 
Tipo de Máquinas Virtuais (e, portanto, de uma mesma Categoria) e que será 
aplicado a uma camada arquitetural da aplicação sob estudo (apresentação, 
negócio, persistência, etc.). Uma Configuração representa o estado de uma 
determinada camada da aplicação quanto à sua escalabilidade, vertical ou horizontal.

Por exemplo, poderíamos avaliar o comportamento de uma Aplicação sob Teste 
implantada com sua camada de aplicação em arquitetura de cluster, composta por 
duas, três ou quatro instâncias em paralelo, caracterizando a variação dessa 
camada em diferentes níveis de escalabilidade horizontal. Para esse teste, 
teríamos então três Configurações diferentes, a primeira com duas instâncias, a 
segunda com três e a terceira com quatro instâncias de máquinas do mesmo Tipo de 
Máquina Virtual.

As heurísticas desenvolvidas e apresentadas por este trabalho comparam 
implantações de diferentes Configurações em uma determinada camada da Aplicação 
sob Teste estudada. Isso permite que sejam feitas avaliações como a viabilidade 
financeira da escalabilidade vertical face ao desempenho possivelmente obtido 
com a escalabilidade horizontal.

\subsection{Espaço de Implantação}
Chamamos de Espaço de Implantação o conjunto limitado de Configurações tomadas 
para execução da Aplicação sob Teste em uma sessão de avaliação.
 
Idealmente, uma Aplicação deveria ser testada sob todos os Tipos de Máquinas 
Virtuais fornecidos pelo Provedor (cobrindo todo o espaço de escalabilidade 
vertical) com o maior número possível de combinações de quantidade de instâncias 
(cobrindo o espaço de escalabilidade horizontal). Porém, se muitos Tipos de 
Máquinas Virtuais forem necessários e se o intervalo de número de instâncias 
solicitado for muito grande, o tempo de duração da sessão e o custo da muitas 
execuções podem se tornar proibitivos.

Assim, o processo de especificação de um Espaço de Implantação consiste em 
selecionar uma lista de Tipos de Máquinas Virtuais entre os oferecidos pelo 
Provedor e designar o melhor valor para o número máximo de instâncias que serão 
usadas na criação das Configurações. Isso faz com que ambos os espaços de 
escalabilidade vertical e horizontal sejam limitados, de forma a controlar 
melhor os custos e permitir que sejam executados testes mais objetivos e de 
acordo com a meta de Carga de Trabalho a ser atendida pela Aplicação.

\subsection{Carga de Trabalho}
A Carga de Trabalho, ou Workload, representa o tamanho da demanda que será 
imposta à Aplicação sob Teste em uma execução. A unidade de medida da Carga é 
dependente do domínio da Aplicação,  mas, para efeito deste trabalho, é 
irrelevante, uma vez que a responsabilidade pela execução dos testes e, por 
conseguinte, pela geração da carga, é delegada a um módulo à parte dentro 
sistema de avaliação.

\subsection{Execução}
Damos o nome de Execução ao evento de utilização de uma Configuração para 
executar a Aplicação sob Teste submetida a uma determinada Carga de Trabalho. 
Dessa forma, a avaliação dos Resultados de uma Execução nos dará uma ideia de 
como a Aplicação responderá às requisições de certo número de usuários (workload) 
após ser implantada num ambiente de nuvem com certo grau de escalabilidade 
horizontal (número de máquinas virtuais usadas).

\section{Formalismos}
Dada uma Aplicação sob Teste A, precisamos identificar, dentre um conjunto de 
cenários de implantação e execução da aplicação em ambiente de nuvem 
computacional sob a modalidade de infraestrutura como serviço, sob quais desses 
cenários a aplicação é executada com sucesso e, dentre esses cenários, quais os 
de menor custo.

De modo a facilitar uma descrição formal do raciocínio lógico empregado na
solução proposta, mostramos a seguir a notação que representa os conceitos 
descritos na seção anterior. Ao final, será apresentada uma descrição formal
do modelo vislumbrado como solução para o problema apresentado.

\subsection{Métricas de Desempenho}
Seja $M(A)$ um conjunto de Métricas de Desempenho de interesse da Aplicação sob 
Teste $A$. Seja $|M(A)|$ a quantidade de Métricas de Desempenho $\in M(A)$ avaliadas 
para a Aplicação sob Teste $A$.

$M(A) = \{m_1, m_2, \dotsc, m_{|P|}\}\; | \; m_n \in M, 1 \leq n \leq |M| \iff m_n $ 
é uma Métrica de Desempenho a ser avaliada para a Aplicação sob Teste $A$.

\subsection{Provedor}
Seja $P$ um conjunto de Provedores. Seja $|P|$ a quantidade de Provedores 
pertencentes a $P$.

$P = \{p_1, p_2, \dotsc, p_{|P|}\}\; | \; p_n \in P, 1 \leq n \leq |P| \iff p_n $ 
é um Provedor de infraestrutura como serviço.

\subsection{Tipos de Máquinas Virtuais}
Seja $T(P)$ um conjunto de Tipos de Máquinas Virtuais fornecidos pelo provedor $P$.

Seja $|T(P)|$ a quantidade de Tipos de Máquinas Virtuais pertencentes a $T(P)$.

$T(P) = \{t_1, t_2, \dotsc, t_{|T(P)|}\}\; | \; t_n \in T(P), 1 \leq n \leq |T(P)|
 \iff t_n $ é um Tipo de Máquina Virtual fornecido pelo provedor P.

Definimos $i(t)$ como uma instância de Máquina Virtual do Tipo $t \in T(P)$. 

Definimos $t(i)$ como o Tipo de Máquina Virtual $t \in T(P)$ usado para criar a 
instância $i$.

Definimos \$$(t)$ como o valor de custo unitário por hora de utilização de uma 
instância do Tipo $t$.

Definimos $cpu(t)$ como o tamanho da capacidade computacional do Tipo $t$.

Definimos $mem(t)$ como o tamanho da capacidade de memória do Tipo $t$.

Dados dois Tipos de Máquinas Virtuais $t_1$ e $t_2$:

$t_1 = t_2 \iff cpu(t_1) = cpu(t_2) \land mem(t_1) = mem(t_2)$.

$t_1 < t_2 \iff cpu(t_1) < cpu(t_2) \land mem(t_1) < mem(t_2)$.

$t_1 > t_2 \iff cpu(t_1) > cpu(t_2) \land mem(t_1) > mem(t_2)$.  
  
\subsection{Categorias de Máquinas Virtuais}
Seja $Cat(P)$ um conjunto de Categorias de Máquinas Virtuais fornecidas pelo 
provedor $P$.

$Cat(P) = \{cat_1, cat_2, \dotsc, cat_{|Cat(P)|}\}\; | \; cat_n \in Cat(P), 
1 \leq n \leq |Cat(P)| \iff cat_n $ é uma Categoria de Máquina Virtual fornecida 
pelo provedor P.

$cat_n = \{T(P)_1, T(P)_2, \dotsc\, T(P)_{|cat_n|}\}\; | \; T(P)_m \subset T(P), 
1 \leq m \leq |cat_n| \iff T(P)_m $ é um grupo de Tipos de Máquina Virtual
fornecidos pelo provedor P.

Definimos $cat(t_n) \in Cat(P)$ como a Categoria atribuída ao Tipo $t_n \in T(P)$ 
pelo Provedor $P$.

\subsection{Configurações}
Seja $c$ um grupo com um certo número de Máquinas Virtuais $i(t)$.

Dizemos que $c$ é uma Configuração se $\forall \; i(t) \in c, t$ é constante.

Definimos $t(c)$ como o Tipo de Máquina Virtual usado para criar as instâncias 
que compõem $c$. Portanto, temos que $t(c) = t(i), \forall \; i(t) \in c$. 

Definimos $\#(c)$ como o tamanho da Configuração $c$, ou seja, a quantidade de 
instâncias $i(t)$ usadas para formar $c$.

Definimos \$$(c)$ como o custo total da configuração, dado por \$$(t) \times \#(c)$.

O conjunto $C(P)$ é o conjunto de todas as Configurações $c$ possíveis de serem 
formadas a partir dos Tipos $t \in T(P)$ e com um número arbitrário de instâncias 
$i(t)$.

Dadas duas Configurações $c_1$ e $c_2$:

$c_1 = c_2 \iff t(c_1) = t(c_2) \land \#(c_1) = \#(c_2)$.

$ c_1 > c_2 \iff \left\{
  \begin{array}{l l}
    t(c_1) = t(c_2) \land \#(c_1) > \#(c_2)\\
    ou\\
    t(c_1) > t(c_2) \land \#(c_1) = \#(c_2)
  \end{array} \right.$
  
$ c_1 < c_2 \iff \left\{
  \begin{array}{l l}
    t(c_1) = t(c_2) \land \#(c_1) < \#(c_2)\\
    ou\\
    t(c_1) < t(c_2) \land \#(c_1) = \#(c_2)
  \end{array} \right.$

\subsection{Espaço de Implantação}
Seja um conjunto de Tipos de Máquinas Virtuais $T(DS_x^y) \subset T(P)$.

Seja um conjunto de Configurações $C_x^y$ tal que:
 
$\forall \; t \in T(DS_x^y), \; \forall \; i \; | \; x \leq i \leq y, \; \exists \; c \in C_x^y \; | \; t(c) = t \; \land \; \#(c) = i$.
 
O Espaço de Implantação $DS_x^y(P)$ é um grafo cujo conjunto de vértices é dado 
por $C_x^y$ e cujas arestas são dadas pelas seguintes operações:

\begin{enumerate}
  \item Ordenar $C_x^y$ pelo atributo $preço$ de cada elemento do conjunto (\$$(c)$).
  \item $preço_{\to} = \forall\; c_1, c_2 \in C_x^y,\; c_1 \to c_2 \iff \$(c_1) < \$(c_2) \land \nexists\; c_3\;|\; \$(c_1) < \$(c_3) < \$(c_2)$
  \item Ordenar $C_x^y$ pelo atributo $CPU$ de cada elemento do conjunto ($cpu(c)$).
  \item $cpu_{\to} = \forall\; c_1, c_2 \in C_x^y,\; c_1 \to c_2 \iff cpu(c_1) < cpu(c_2) \land \nexists\; c_3\;|\; cpu(c_1) < cpu(c_3) < cpu(c_2)$
  \item Ordenar $C_x^y$ pelo atributo $Memória$ de cada elemento do conjunto ($mem(c)$).
  \item $mem_{\to} = \forall\; c_1, c_2 \in C_x^y,\; c_1 \to c_2 \iff mem(c_1) < mem(c_2) \land \nexists\; c_3\;|\; mem(c_1) < mem(c_3) < mem(c_2)$
\end{enumerate}

\subsection{Cargas de Trabalho}
Seja $W$ um conjunto finito de valores que representam Cargas de Trabalho. Seja 
$|W|$ a quantidade de Cargas de Trabalho pertencentes a $W$.

$W = \{w_1, w_2, \dotsc, w_{|W|}\}\; | \; w_n \in W, 1 \leq n \leq |W| \iff w_n $ 
é um valor válido como carga a ser submetida dentro do domínio de conhecimento 
da Aplicação sob Teste.

\subsection{Resultados}
Um Resultado $r$ é uma quádrupla no formato $\{m, v, cpu, mem\}$ onde:

$m \in M(A)$ é a Métrica de Desempenho medida no Resultado;

$v$ é o valor medido para a Métrica;

$cpu$ é o percentual de CPU utilizada para a obtenção de $v$; e

$mem$ é o percentual de memória RAM utilizada para a obtenção de $v$.

\subsection{Execuções}
Seja $E(A)$ um conjunto de Execuções da Aplicação sob Teste $A$. Seja $|E|$ a 
quantidade de Execuções pertencentes a $E(A)$.

$E(A) = \{e_1, e_2, \dotsc, e_{|E|}\}\; | \; e_n \in E(A), 1 \leq n \leq |E(A)| \iff e_n $ 
é uma tripla no formato $\{c, w, r\}\; |\; c \in DS_x^y(P), w \in W$ e onde $r$ é 
o Resultado obtido a partir da Execução da Aplicação na configuração $c$, 
submetida à Carga de Trabalho $w$.

\subsection{Avaliação da Execução}
Seja $\alpha$ um valor de referência definido como parâmetro de sucesso para uma 
Métrica de Desempenho quando da Execução de um teste da Aplicação.

Seja $e \in E(A)$ uma Execução da Aplicação $A$

Seja $r \in e$ o Resultado da Execução $e$

Seja $m \in r$ a Métrica de Desempenho avaliada no Resultado $r$

Seja $v \in r$ o valor medido para a Métrica $m$ no Resultado $r$

Seja $atende(r, \alpha)$ uma função tal que:

$m\; é\; minimizável \iff atende(r,\alpha) = \left\{
  \begin{array}{l l}
    \{1, \delta\}, v \leq \alpha\\
    \{0, \delta\}, v > \alpha
  \end{array} \right.$

$m\; é\; maximizável \iff atende(r,\alpha) = \left\{
  \begin{array}{l l}
    \{0, \delta\}, v < \alpha\\
    \{1, \delta\}, v \geq \alpha
  \end{array} \right.$

Em ambos os casos, $\delta$ representa a distância entre o valor de Resultado 
obtido para a Métrica $m$ na Execução $e$ (ou seja, o valor $e[r[v]]$) e o Valor
de Referência ou SLA $\alpha$. 

A representação dessa distância é dada por $\delta = \{grau\; de\; desvio,
\; direção\; do\; desvio\}$, onde o $grau\;de\;desvio$ indica se esse desvio é 
$alto$, $médio$ ou $baixo$, e a $direção\;do\;desvio$ indica se o resultado 
obtido está $abaixo$ ou $acima$ do SLA.  

Os limites de comparação do Resultado com o SLA para a classificação do grau de 
desvio são parâmetros e podem ser modificados conforme a tolerância da Aplicação
a flutuações de desempenho do ambiente de nuvem. Os valores desses parâmetros 
são dados em termos percentuais.

As informações trazidas por $\delta$ podem influenciar o modo como as Heurísticas
de Avaliação de Capacidade caminham sobre o Espaço de Implantação na busca pelas 
melhores Configurações. Seu uso, porém, é opcional e fica a critério da própria 
Heurística.

Apresentamos a seguir uma especificação de funcionamento das Heurísticas, 
detalhando seu papel dentro deste trabalho, insumos de entrada e as operações
que devem fornecer como interface para sua manipulação e controle.