\chapter[Introdução]{Introdução}
% ----------------------------------------------------------
O processo de decisão pela migração de aplicações para o ambiente de nuvens computacionais 
envolve uma série de análises que buscam, entre outras coisas, identificar que vantagens a 
mudança trará de fato. Ao comparar os serviços de provedores de computação em nuvem com a 
administração de um centro de dados próprio, a melhoria dos indicadores de desempenho e de 
custo, como redução de tempo de resposta, redução/otimização de custo de operação e melhores 
ferramentas com mais facilidades de gerenciamento, está entre os principais benefícios buscados 
a partir da adoção do ambiente de infraestrutura como serviço (IaaS – Infrastructure as a Service)
\cite{li2011cloudprophet, rodero2010infrastructure}.

Em geral, provedores de IaaS cobram um valor em função do tempo de ocupação de uma máquina 
virtual, normalmente medido em horas, e esse valor unitário varia conforme o tamanho da máquina 
virtual (capacidade de processamento, memória e espaço de armazenamento). Dessa forma, a apuração 
do custo de operação da aplicação em um determinado período de tempo leva em conta a quantidade 
de máquinas virtuais utilizadas bem como seu perfil, ou seja, o tamanho e quantidade de recursos 
usados em cada uma.
 
Para prever o custo de operação de uma aplicação na nuvem, é preciso estimar ou medir como a 
aplicação responderá à demanda submetida em termos de indicadores de desempenho. A aplicação deve 
manter ou superar na nuvem o nível de desempenho apresentado quando executada em centro de dados 
próprio e com indicadores de custo menores, a fim de que se justifique o investimento feito na 
migração de ambiente. Para se chegar a essa conclusão, faz-se necessário conhecer o comportamento 
da aplicação na nova implantação para que se identifiquem quais perfis de máquinas virtuais 
oferecidos pelo provedor são capazes de executar a aplicação com níveis satisfatórios de desempenho. 
Somem-se a isso as variações da demanda exercida sobre a aplicação e as diversas possibilidades de 
variação de arquitetura de implantação por meio de procedimentos de escalabilidade.
 
Ao se tomarem procedimentos de escalabilidade vertical (variando-se a quantidade de recursos de 
cada máquina) e/ou de escalabilidade horizontal (variando-se a quantidade de máquinas em uma ou 
mais camadas, como dados, apresentação e negócio) chega-se a níveis de desempenho e de custo muito 
diversos. A variação da demanda exige que a aplicação também varie em tamanho da implantação, 
vertical ou horizontalmente, conforme a carga aplicada. Quanto mais acentuadas e mais frequentes 
as variações na demanda, mais variações de custo e desempenho serão observadas.

Assim, o custo apresenta-se entre os mais difíceis de prever, uma vez que depende necessariamente 
do tamanho da demanda exercida sobre a aplicação além do desempenho oferecido e preços cobrados 
pelo provedor de nuvem de infraestrutura contratado \cite{cunha2012ambiente}. Estrategicamente, 
torna-se interessante identificar, entre as possíveis composições de máquinas virtuais ofertadas 
em um ou vários provedores, quais são as configurações de menor custo capazes de executar a 
aplicação mantendo-se os níveis satisfatórios para os indicadores de desempenho.

Para saber se uma determinada configuração de recursos do provedor é capaz de atender a uma 
demanda específica, é preciso antes enumerar os indicadores de desempenho que mais interessam 
à aplicação e a partir daí estabelecer os valores aceitáveis para esses indicadores. Uma vez 
estabelecidos os valores aceitáveis, pode-se implantar a aplicação sob essa configuração de 
recursos e então aplicar diferentes níveis de carga de trabalho sobre a aplicação. Ao comparar 
a resposta da aplicação com os valores dados como aceitáveis para os indicadores, é possível 
determinar se aquela configuração de recursos escolhida é capaz de executar a aplicação a 
contento e ainda calcular o custo mensal dessa implantação.

Porém, partindo-se do pressuposto de que o desempenho da aplicação foi satisfatório, o que se 
tem até agora é o custo de uma única configuração capaz de executar a aplicação estudada sob 
um único nível de carga de trabalho. Mas cargas de trabalho costumam variar em função do tempo 
em implantações reais, fazendo-se necessário, portanto, que esse efeito seja contemplado nos 
testes por meio da medição do desempenho da aplicação submetida a diferentes níveis de carga 
de trabalho.

Analogamente, as diversas configurações de máquinas e recursos, ainda que no mesmo provedor, 
podem responder de maneira muito diferente sob o mesmo nível de carga de trabalho a depender 
do momento em que sejam ativados \cite{cunha2011investigating, iosup2011performance, 
jayasinghe2011variations}. Independente do motivo que leve a esse comportamento de certa 
forma imprevisível, é preciso levar em conta nos ensaios de avaliação de desempenho essa 
variabilidade e isso pode ser alcançado através da repetição dos cenários de teste em horários 
e dias diferentes.

Um grande problema começa a se desenhar ao seguir essa abordagem: a fase de ensaios pode atingir 
patamares elevados de custo, a depender das necessidades de variação da demanda, da arquitetura 
de implantação e das configurações utilizadas em cada arquitetura implantada 
\cite{silva2013cloudbench}. Ainda que certos provedores IaaS ofereçam descontos ou pacotes de 
horas grátis para novos clientes, em geral esses incentivos são suficientes para custear apenas 
um mês de utilização de uma única máquina virtual muito pequena, provavelmente incapaz suportar 
a carga de uma aplicação real em produção. Assim, executar uma aplicação real, tipicamente 
implantada em arquitetura de várias camadas, em máquinas virtuais de tamanho considerável e por 
longos períodos de tempo apenas para estudar o seu comportamento, pode se traduzir em um custo 
alto que inviabilize o próprio projeto de migração dessa aplicação para a nuvem. Para evitar 
que sejam feitos testes com todas as combinações de provedores, configurações, horários, cargas 
de trabalho e métricas avaliadas, é possível lançar mão de técnicas de predição.

Através da predição, é possível estimar com razoável aproximação o desempenho que a aplicação 
apresentará ao ser executada em vários perfis de configuração diferentes, permitindo a 
determinação de qual configuração de menor custo capaz de executar a aplicação e sem a 
necessidade da realização completa dos testes. Como consequência, o custo dessa fase de 
ensaios pode ser reduzido sensivelmente.

Este trabalho propõe heurísticas de predição de custo mínimo para execução de aplicações 
em ambientes de nuvem de infraestrutura, bem como um arcabouço de programação que apoia a 
implementação dessas heurísticas. Além disso, o trabalho estuda os resultados apresentados 
pela aplicação das heurísticas propostas quanto ao custo total de execução da fase de ensaios 
para escolha da melhor configuração capaz de executar uma aplicação e quanto à acuidade dos 
resultados da predição em si.
