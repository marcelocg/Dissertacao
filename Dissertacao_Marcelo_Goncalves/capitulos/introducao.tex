\chapter[Introdução]{Introdução}
% ----------------------------------------------------------

\section{Motivação}

A computação em nuvem é um paradigma computacional que está
transformando a forma de desenvolver e gerenciar aplicações e serviços de
tecnologia da informação. Diversas organizações passaram a adotar este paradigma
atraídas pelo seu modelo de negócios onde recursos computacionais (ex:
computação, armazenamento e transferência de dados) podem ser consumidos, sob-demanda, 
como um serviço e pagos de acordo com o consumo efetuado. Um dos principais
desafios enfrentados pelos usuários de nuvens que oferecem 
infraestrutura-como-serviço (IaaS) é planejar adequadamente a capacidade 
dos recursos da nuvem necessários às suas aplicações.  

O processo de decisão pela migração de aplicações para o ambiente de nuvens computacionais 
envolve uma série de análises que buscam, entre outras coisas, identificar que vantagens a 
mudança trará de fato \cite{li2011cloudprophet, rodero2010infrastructure}, 
além de tentar descobrir a melhor maneira de configurar a aplicação com os
vários recursos oferecidos pelo provedor de nuvem.

Um dos principais recursos computacionais oferecidos por provedores de nuvem
IaaS, que normalmente representam a parte mais significativa dos gastos dos
clientes, é o serviço de máquina virtual. Em geral, provedores de IaaS cobram um
valor em função do tempo de utilização deste recurso, normalmente medido em horas, e esse valor 
unitário varia conforme o tamanho da máquina virtual (capacidade de processamento, memória e espaço de armazenamento).
Dessa forma, a apuração do custo de operação de uma aplicação em um determinado
período de tempo leva em conta a quantidade de máquinas virtuais utilizadas bem como seu perfil, ou seja, o tamanho e quantidade de recursos 
usados em cada uma.
 
Para prever o custo de operação de uma aplicação na nuvem, é preciso estimar ou medir como a 
aplicação responderá à demanda submetida em termos de indicadores de desempenho. Para se chegar a essa 
conclusão, faz-se necessário conhecer o comportamento 
da aplicação no ambiente de nuvem para que se identifiquem quais perfis de
máquinas virtuais oferecidos pelo provedor são capazes de executar a aplicação com níveis satisfatórios de desempenho. 
Somem-se a isso as variações da demanda exercidas sobre a aplicação e as
diversas possibilidades de variação da arquitetura de implantação por meio de
procedimentos de escalabilidade.
 
Ao se tomarem procedimentos de escalabilidade vertical (variando-se a quantidade de recursos de 
cada máquina) e/ou de escalabilidade horizontal (variando-se a quantidade de máquinas em uma ou 
mais camadas da aplicação, como dados, apresentação e negócio) chega-se a níveis de desempenho e de custo muito 
diversos. A variação da demanda exige que a aplicação também varie em tamanho da implantação, 
vertical ou horizontalmente, conforme a carga aplicada. Quanto mais acentuadas e mais frequentes 
as variações na demanda, mais variações de custo e desempenho serão observadas.

Assim, o custo apresenta-se entre os mais difíceis de prever, uma vez que depende necessariamente 
do tamanho da demanda exercida sobre a aplicação além do desempenho oferecido e preços cobrados 
pelo provedor de nuvem de infraestrutura contratado \cite{cunha2012ambiente}. Estrategicamente, 
torna-se interessante identificar, entre as possíveis composições de máquinas virtuais ofertadas 
em um ou vários provedores, quais são as configurações de menor custo capazes de executar a 
aplicação mantendo-se os níveis satisfatórios para os indicadores de desempenho.

Para facilitar o entendimento do problema suponha o seguinte exemplo. Uma
determinada organização quer utilizar um provedor de nuvem para implantar uma
determinada aplicação web multicamadas. Para saber se uma determinada 
configuração de recursos do provedor (ex: uma máquina virtual de tamanho
pequeno para a camada de aplicação e outra para a a camada de dados) é capaz de
atender a uma demanda específica (ex: carga imposta por 100 usuários
concorrentes), é preciso antes enumerar os indicadores de desempenho que mais
interessam à aplicação (ex: tempo de resposta) e a partir daí estabelecer os
valores aceitáveis para esses indicadores (ex: tempo de resposta abaixo de 10
segundos para uma determinado conjunto de operações com 100 usuários).
Uma vez estabelecidos os valores aceitáveis, pode-se implantar a aplicação 
sob essa configuração de recursos no ambiente da nuvem e então testar o
desempenho da aplicação. Ao comparar a resposta da aplicação com os valores
dados como aceitáveis para os indicadores, é possível determinar se aquela configuração de recursos 
escolhida é capaz de executar a aplicação a 
contento e ainda calcular o custo mensal dessa implantação.

Porém, partindo-se do pressuposto de que o desempenho da aplicação foi satisfatório, o que se 
tem até agora é o custo de uma única configuração capaz de executar a aplicação estudada sob 
um único nível de carga de trabalho. No entanto, cargas de trabalho costumam variar em função do tempo 
em implantações reais, fazendo-se necessário, portanto, que esse efeito seja contemplado nos 
testes por meio da medição do desempenho da aplicação submetida a diferentes níveis de carga 
de trabalho.

Analogamente, as diversas configurações de máquinas e recursos, ainda que no mesmo provedor, 
podem responder de maneira diferente sob o mesmo nível de carga de trabalho a depender 
do momento em que sejam ativados \cite{cunha2011investigating, iosup2011performance, 
jayasinghe2011variations}. Independente do motivo que leve a esse comportamento de certa 
forma imprevisível, é preciso levar em conta nos ensaios de avaliação de desempenho essa 
variabilidade e isso pode ser alcançado através da repetição dos cenários de teste em horários 
e dias diferentes.

Um grande problema começa a se desenhar ao seguir essa abordagem: a fase de
experimentação pode atingir patamares elevados de custo, a depender das
necessidades de variação da demanda, da arquitetura de implantação e das configurações utilizadas 
em cada arquitetura implantada 
\cite{silva2013cloudbench}. Ainda que certos provedores IaaS ofereçam descontos ou pacotes de 
horas grátis para novos clientes, em geral esses incentivos são suficientes para custear apenas 
um mês de utilização de uma única máquina virtual muito pequena, provavelmente incapaz de suportar 
a carga de uma aplicação real em produção. Assim, executar uma aplicação real, tipicamente 
implantada em arquitetura de várias camadas, em máquinas virtuais de tamanho considerável e por 
longos períodos de tempo apenas para estudar o seu comportamento, pode se traduzir em um custo 
alto que inviabilize o próprio projeto de migração dessa aplicação para a nuvem. 

Com o intuito de resolver este problema algumas abordagens (ref XXX) já
foram desenvolvidas e se organizam em duas categorias: preditiva e empírica.
Normalmente baseadas em simuladores, estas abordagens apresentam uma baixa
eficiência (Colocar referencia XXX) com relação a sua capacidade de prever as
melhores configurações dos recursos da nuvem para variadas cargas impostas na aplicação. Isto se deve
normalmente a fatores como: diferenças conceituais entre o modelo de simulação e
o modelo real de comportamento da aplicação e instabilidade dos serviços de
nuvem. Uma vantagem das abordagens preditivas é não onerar os clientes com os
testes pois não exigem implantação real no ambiente da nuvem.

As abordagens empíricas são baseadas na implantação das aplicações alvo na nuvem 
e em testes de carga. As abordagens normalmente oferecem suporte aos usuários
para automatizar as taferas de instalar, configurar e testar as aplicações no
ambiente da nuvem eliminando uma série de trabalhos manuais custosos. Por
executarem no ambuiente de nuvem real elas conseguem ajudar a atingir resultados
significativamente melhores no que diz respeito a seleção de recursos
computacionais para cargas de trabalho específicas. No entanto, um problema ainda existente nessas abordagens
é a necessidade de se testar exaustivamente uma grande quantidade de
configurações de recursos e cargas de trabalho implicando em altos custos
durante a fase de experimentação.

Com o intuito de combinar as vantagens das duas abordagens, este trabalho
propõe uma nova maneira de apoiar o planejamento da capacidade de aplicações
em nuvens IaaS. A nova abordagem tem como premissa a definição de uma relação 
de capacidade entre as diferentes configurações de recursos oferecidas por um provedor 
de nuvem, com a qual é possível prever (ou ``inferir''), com alto grau de precisão, 
o desempenho esperado de uma aplicação para determinadas configurações de recursos. A predição 
é realizada com base no desempenho observado (em testes na nuvem) para outras
configurações de recursos do mesmo provedor.

\section{Objetivos}

Este trabalho tem como principal objetivo o de propor uma abordagem de
planejamento de capacidade baseado nas relações de capacidade existentes em
recursos da nuvem e também em um processo de inferência de desempenho 
que utiliza como insumo resultados de execuções reais na nuvem. Os objetivos
específicos do trabalho são:

\begin{itemize}
  \item Definir uma forma de representar as relações de capacidade entre
  configurações de recurso de um provedor de nuvem;
  \item Propor um processo de planejamento de capacidade que seja capaz de 
  prever as melhores configurações de recurso capazes de suportar variadas
  cargas de trabalho;
  \item Fornecer uma implementação concreta do processo proposto;
  \item Avaliar a eficiência e efetividade do processo através de um estudo
  baseado em uma aplicação concreta em um provedor de nuvem real.
\end{itemize}
 
\section{Contribuições}

Com o intuito de resolver a problemática do planejamento de capacidade na nuvem
atendendo aos objetivos propostos algumas contribuições podem ser destacadas
deste trabalho.

Primeiramente propusemos um modelo que representa as relações de capacidade de
configurações de uma nuvem. Este modelo é chamado de Espaço de Implantação
e consiste de uma estrutura em grafo que representa as relações entre os
Tipos de Máquinas Virtuais oferecidos pelo Provedor de nuvem. O grafo é
construído de forma a indicar as relações de capacidade entre os tipos de
instância indicando quais tipos são maiores, menores ou não relacionados com
outros.

Outra contribuição significativa é o processo de inferencia de desempenho ..
falar das inovacoes, heuristicas, 

capacitor --> falar das inovacoes,das



\section{Estrutura da Dissertação}
