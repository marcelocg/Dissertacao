\chapter[Introdução]{Introdução}
% ----------------------------------------------------------
\section{Motivação}
A computação em nuvem é um paradigma computacional que está
transformando a forma de desenvolver e gerenciar aplicações e serviços de
Tecnologia da Informação (TI) \cite{Murugesan2014}. Diversas organizações passaram 
a adotar este paradigma atraídas pelo seu modelo de negócios onde recursos 
computacionais (ex: CPU, espaço de armazenamento e volume de dados transferidos) 
podem ser consumidos, sob-demanda, como um serviço, e pagos de acordo com o consumo 
efetuado. Um dos principais desafios enfrentados pelos usuários de nuvens que oferecem 
infraestrutura-como-serviço (IaaS) é planejar adequadamente a capacidade 
dos recursos da nuvem necessários às suas aplicações \cite{Menasce2009}.  

O processo de decisão pela migração de aplicações para o ambiente de nuvens 
computacionais envolve uma série de análises que buscam, entre outras coisas, 
identificar as vantagens que esta eventual migração trará de fato \cite{beserra2012cloudstep, 
rodero2010infrastructure}, além de tentar descobrir a melhor maneira de configurar 
a aplicação com os vários recursos oferecidos pelo provedor de nuvem.

Um dos principais recursos computacionais oferecidos por provedores de nuvem de 
infraestrutura como serviço (IaaS), que tipicamente representam a parte mais 
significativa dos gastos dos
clientes, é o serviço de máquina virtual. Em geral, provedores de IaaS cobram um
valor em função do tempo de utilização deste recurso, normalmente medido em horas, 
e esse valor unitário varia conforme o tamanho da máquina virtual (capacidade de 
processamento, memória e espaço de armazenamento). Dessa forma, a apuração do 
custo de operação de uma aplicação em um determinado período de tempo leva em 
conta a quantidade de máquinas virtuais utilizadas bem como seu perfil, ou seja, 
o tamanho e quantidade de recursos usados em cada uma.
 
A fim de prever o custo de operação de uma aplicação na nuvem, é preciso estimar ou 
medir como a aplicação responderá à demanda submetida em termos de indicadores de 
desempenho. Para se chegar a essa conclusão, faz-se necessário conhecer o 
comportamento da aplicação no ambiente de nuvem de modo que se identifiquem quais 
perfis de máquinas virtuais oferecidos pelo provedor são capazes de executar a 
aplicação com níveis satisfatórios de desempenho. Somem-se a isso as variações 
da demanda exercidas sobre a aplicação e as diversas possibilidades de variação 
da arquitetura de implantação por meio de procedimentos de escalabilidade.
 
Ao se tomarem procedimentos de escalabilidade vertical (variando-se a quantidade 
de recursos de cada máquina) e/ou de escalabilidade horizontal (variando-se a 
quantidade de máquinas em uma ou mais camadas da aplicação, como dados, apresentação 
e negócio) chega-se a níveis de desempenho e de custo muito diversos. A variação 
da demanda exige que a aplicação também varie em tamanho da implantação, vertical 
ou horizontalmente, conforme a carga aplicada. Quanto mais acentuadas e mais 
frequentes as variações na demanda, mais variações de custo e desempenho serão 
observadas.

Assim, o custo apresenta-se entre os mais difíceis de prever, uma vez que depende 
necessariamente do tamanho da demanda exercida sobre a aplicação além do desempenho 
oferecido e preços cobrados pelo provedor de nuvem de infraestrutura contratado 
\cite{cunha2012ambiente}. Estrategicamente, torna-se fundamental identificar, 
entre as possíveis composições de máquinas virtuais ofertadas em um ou vários 
provedores, quais são as configurações de menor custo capazes de executar a 
aplicação mantendo-se os níveis considerados satisfatórios para os indicadores de desempenho.

Para facilitar o entendimento do problema, suponha o seguinte exemplo. Uma
determinada organização quer utilizar um provedor de nuvem para implantar uma
determinada aplicação web multicamadas. Para saber se uma determinada 
configuração de recursos do provedor (ex: uma máquina virtual de tamanho
pequeno para a camada de aplicação e outra para a a camada de dados) é capaz de
atender a uma demanda específica (ex: carga imposta por 100 usuários
concorrentes), é preciso antes enumerar os indicadores de desempenho que mais
interessam à aplicação (ex: tempo de resposta) e a partir daí estabelecer os
valores aceitáveis para esses indicadores (ex: tempo de resposta abaixo de 10
segundos para uma determinado conjunto de operações requisitadas por 100 usuários concorrentes).
Uma vez estabelecidos os valores aceitáveis, pode-se implantar a aplicação 
sob essa configuração de recursos no ambiente da nuvem e então testar o
desempenho da aplicação. Ao comparar a resposta da aplicação com os valores
dados como aceitáveis para os indicadores, é possível determinar se aquela 
configuração de recursos escolhida é capaz de executar a aplicação a contento e 
ainda calcular o custo por período (por exemplo, mensal ou anual) associado ao uso dessa configuração.

Porém, partindo-se do pressuposto de que o desempenho da aplicação foi satisfatório, 
o que se tem até agora é o custo de uma única configuração capaz de executar a 
aplicação estudada sob um único nível de carga de trabalho. No entanto, cargas de 
trabalho costumam variar em função do tempo em implantações reais, fazendo-se 
necessário, portanto, que esse efeito seja contemplado nos testes por meio da 
medição do desempenho da aplicação submetida a diferentes níveis de carga de 
trabalho.

Analogamente, as diversas configurações de máquinas e recursos, ainda que no mesmo 
provedor, podem responder de maneira diferente sob o mesmo nível de carga de 
trabalho, a depender do momento em que sejam ativados \cite{cunha2011investigating, 
iosup2011performance, jayasinghe2011variations}. Independente do motivo que leve 
a esse comportamento de certa forma imprevisível, é preciso levar em conta nos 
ensaios de avaliação de desempenho essa variabilidade, o que pode pode ser alcançado 
através da repetição dos cenários de teste em horários e dias diferentes.

Um grande problema começa a se desenhar ao seguir essa abordagem: a fase de
experimentação pode atingir patamares elevados de custo, a depender das
necessidades de variação da demanda, da arquitetura de implantação e das 
configurações utilizadas em cada arquitetura implantada \cite{silva2013cloudbench}. 
Ainda que certos provedores IaaS ofereçam descontos ou pacotes de horas grátis 
para novos clientes, em geral esses incentivos são suficientes para custear apenas 
um mês de utilização de uma única máquina virtual muito pequena, provavelmente 
incapaz de suportar a carga de uma aplicação real em produção. Assim, executar 
uma aplicação real, tipicamente implantada em arquitetura de várias camadas, em 
máquinas virtuais de tamanho considerável e por longos períodos de tempo apenas 
para estudar o seu comportamento, pode se traduzir em um custo alto que dificulte ou até mesmo inviabilize 
o próprio projeto de migração dessa aplicação para a nuvem. 

Com o intuito de ajudar a resolver o problema do planejamento de capacidade de aplicações na nuvem, vários trabalhos foram propostos, o quais podem ser classificados de acordo com duas abordagens: preditiva \cite{cloudharmony,
malkowski2010cloudxplor, fittkau2012cdosim, li2011cloudprophet} e empírica 
\cite{jayasinghe2012, silva2013cloudbench, cunhacloud, scheuner2014cloud}.
Trabalhos que adotam a abordagem preditiva têm como objetivo tentar prever o desempenho esperado da aplicação alvo quando executada na nuvem sob determinada configuração de recursos e determinado nível de carga. As técnicas de predição variam conforme o trabalho, sendo as mais comuns o uso de simuladores de nuvem \cite{fittkau2012cdosim} e a analogia com resultados previamente obtidos através da utilização de \emph{benchmarks} \cite{cloudharmony,
malkowski2010cloudxplor}. Uma desvantagem dos trabalhos que seguem a abordagem preditiva é a sua ainda limitada capacidade para prever as melhores configurações dos recursos da nuvem para atender variadas cargas impostas à 
aplicação. Isto se deve principalmente a fatores como: diferenças conceituais entre o comportamento real da aplicação e  
os modelos de simulação, a ausência de \emph{benchmarks} com funcionalidades e demandas similares aos da aplicação, além da instabilidade, comuns aos serviços de nuvem. Por outro lado, trabalhos da abordagem preditiva têm a grande vantagem de não onerar os usuários, pois não exigem a implantação e execução da aplicação no ambiente de nuvem real.

Trabalhos que utilizam a abordagem empírica são baseados na implantação das aplicações alvo na nuvem 
e em testes de carga. Esses trabalhos normalmente oferecem suporte aos usuários
para automatizar as tarefas de instalar, configurar e testar as aplicações no
ambiente da nuvem, eliminando uma série de procedimentos manuais custosos. Por
executarem em um ambiente de nuvem real, eles conseguem ajudar a atingir resultados
significativamente melhores no que diz respeito a seleção de recursos
computacionais para cargas de trabalho específicas. No entanto, um problema ainda 
existente nos trabalhos dessa abordagem é a necessidade de se testar exaustivamente uma grande 
quantidade de configurações de recursos e cargas de trabalho, implicando em altos 
custos durante a fase de experimentação.

Visando combinar as vantagens das duas abordagens, este trabalho
propõe uma nova maneira de apoiar o planejamento da capacidade de aplicações
em nuvens IaaS. A nova abordagem tem como premissa a definição de uma relação 
de capacidade entre as diferentes configurações de recursos oferecidas por um 
provedor de nuvem, com a qual é possível prever (ou ``inferir''), com alto grau 
de precisão, o desempenho esperado de uma aplicação para determinadas configurações 
de recursos. A predição é realizada com base no desempenho observado (em testes 
na nuvem) para outras configurações de recursos do mesmo provedor.

\section{Objetivos}
Este trabalho tem como principal objetivo propor uma abordagem de
planejamento de capacidade baseado nas relações de capacidade existentes em
recursos da nuvem e também em um processo de inferência de desempenho 
que utiliza como insumo resultados de execuções reais na nuvem. Os objetivos
específicos do trabalho são:

\begin{itemize}
  \item Definir uma forma de representar as relações de capacidade entre
  configurações de recurso de um provedor de nuvem;
  \item Propor um processo de avaliação de capacidade que seja capaz de 
  prever as melhores configurações de recurso capazes de suportar variadas
  cargas de trabalho;
  \item Fornecer uma implementação concreta do processo proposto;
  \item Avaliar a eficiência e efetividade do processo através de um estudo
  baseado em uma aplicação concreta em um provedor de nuvem real.
\end{itemize}
 
\section{Contribuições}
Com o intuito de resolver a problemática do planejamento de capacidade na nuvem,
atendendo aos objetivos propostos, algumas contribuições podem ser destacadas
deste trabalho.

Primeiramente, é proposto um modelo que representa as relações de capacidade de
configurações de uma nuvem. Este modelo é chamado de Espaço de Implantação
e consiste de uma estrutura em grafo que representa as relações entre os
Tipos de Máquinas Virtuais oferecidos pelo Provedor de nuvem. O grafo é
construído de forma a indicar as relações de capacidade entre os tipos de
instância indicando quais tipos são maiores, menores ou não relacionados com
outros.

Outra contribuição significativa é o Processo de Avaliação de Capacidade por 
Inferência de Desempenho. Esse processo define uma nova abordagem para o 
planejamento de capacidade com base na inferência do comportamento de máquinas 
virtuais. A inferência se dá sob a luz da análise da relação de capacidade 
representada pelo Espaço de Implantação e a partir do resultado da execução de 
testes reais de desempenho em relativamente poucas máquinas dentre o conjunto completo de 
configurações avaliadas.

Para que o Processo saiba identificar quais configurações devem ser efetivamente
testadas e em que ordem, foram definidas Heurísticas para a seleção dessas 
configurações. A cada passo da iteração do Processo, a Heurística selecionada é
acionada para que, a depender do resultado obtido no passo anterior, uma nova
configuração de máquinas virtuais seja levada a testes.

As Heurísticas conferem grande flexibilidade ao Processo, que pode se adaptar a
diferentes comportamentos da Aplicação sob Teste. Além de já ter definido um 
conjunto inicial, o Processo prevê a possibilidade de que novas Heurísticas possam
ser escritas e aplicadas como pontos de extensão à implementação original.

Quando um estudo de capacidade está sendo executado por um profissional, é 
intuitivo que não sejam testadas todas as combinações possíveis de configurações
e cargas de trabalho. Porém, no momento da realização dos testes, a decisão de
qual será a próxima configuração a ser testada é um problema de difícil solução.
Com custos altos envolvidos já durante a fase de execução de testes e com 
possivelmente dezenas de configurações a serem examinadas, como saber qual a
configuração a ser executada a fim de minimizar o custo e maximizar as chances
de obter respostas conclusivas quanto ao planejamento de capacidade em andamento?

Essa é outra contribuição consequente da aplicação da Processo de Inferência de
Desempenho: a minimização do custo de execução de testes e a maximização das 
respostas para o planejamento de capacidade, com altíssimo nível de confiança na
acurácia dessas respostas.

Este trabalho apresenta ainda uma implementação concreta usada em experimentos para 
validar o Processo proposto. Os resultados desses experimentos comprovam sua eficiência 
na redução de custos e sua acurácia ao apontar quais configurações conseguem 
atender a um acordo de nível de serviço estipulado previamente. 

\section{Estrutura da Dissertação}
Além desta Introdução, esta dissertação está organizada em mais cinco capítulos, 
descritos brevemente a seguir.

\begin{description}
  \item[Capítulo 2 -- Fundamentação Teórica e Trabalhos Relacionados] \hfill \\
  Aborda alguns conceitos fundamentais da computação em nuvem e planejamento de
  capacidade, de modo a nivelar o conhecimento do leitor sobre o tema. Além disso, 
  são descritos e analisados criticamente diversos trabalhos relacionados a esta pesquisa.
  \item[Capítulo 3 -- Processo de Avaliação de Capacidade] \hfill \\
  Apresenta os conceitos criados especificamente no âmbito desta pesquisa, explicando 
  a terminologia usada no texto e que permeiam a solução proposta. Descreve em 
  detalhes o Processo proposto para a avaliação de capacidade baseado em
  inferência de desempenho.
  \item[Capítulo 4 -- Implementação do Processo] \hfill \\
  Descreve os detalhes da implementação concreta do processo proposto, mostrando
  como foi construída a biblioteca de programação que fornece a funcionalidade
  necessária à execução de rotinas de avaliação de capacidade. Apresenta também
  o sistema construído com a utilização da biblioteca, validando sua 
  aplicabilidade. 
  \item[Capítulo 5 -- Experimentos e Resultados] \hfill \\
  Aqui são apresentados os experimentos criados como parte deste estudo a fim de
  verificar a aplicabilidade do processo de inferência de desempenho. Os resultados 
  dos experimentos são analisados e vêm comprovar a eficiência e acurácia do 
  Processo. 
  \item[Capítulo 6 -- Conclusão] \hfill \\
  Faz o fechamento do trabalho, apontando as contribuições trazidas e sugestões
  para trabalhos futuros. 
\end{description}
% ----------------------------------------------------------
