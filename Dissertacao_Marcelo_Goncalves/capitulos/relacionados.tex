\chapter[Trabalhos Relacionados]{Trabalhos Relacionados}
% ----------------------------------------------------------
Apresentamos  neste capítulo alguns trabalhos cujos objetivos estejam alinhados 
com a ideia da avaliação de desempenho de aplicações executadas em ambientes de
computação em nuvem. Fazemos então uma análise de seus objetivos e resultados 
alcançados, bem como traçamos as semelhanças com este trabalho, avaliando quais são
nossas contribuições diferenciais.

\section{CloudBench}
\cite{silva2013cloudbench} descrevem o CloudBench como um arcabouço para 
automação dos testes e avaliação do desempenho de ambientes de nuvem computacional 
sob o modelo IaaS. Através da abstração de experimentos, aplicações e máquinas 
virtuais, CloudBench a execução dos testes e a coleta de dados relativos
às métricas de desempenho observadas. CloudBench prevê métricas que dizem respeito
não só à aplicação, mas também ao provedor, como latência de provisionamento. 

As abstrações criadas pelo CloudBench permitem que um experimento seja 
especificado através de uma lista de diretivas. Essas diretivas descrevem os itens
que compõem o experimento, definindo objetos como a aplicação e asinstâncias de 
máquinas virtuais utilizadas.

******* CloudBench executa apenas benchmarks! 
(página 3 = The applications used in the experiments are predefined
benchmarks that fall into one of the categories
previously discussed, making CloudBench a metabenchmark
(or benchmark harness).)


\cite{li2011cloudprophet}