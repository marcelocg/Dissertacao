\chapter[Trabalhos Relacionados]{Trabalhos Relacionados}
% ----------------------------------------------------------
Apresentamos  neste capítulo alguns trabalhos cujos objetivos estejam alinhados 
com a ideia da avaliação de desempenho de aplicações executadas em ambientes de
computação em nuvem. Fazemos então uma análise de seus objetivos e resultados 
alcançados, bem como traçamos as semelhanças com este trabalho, avaliando quais são
nossas contribuições diferenciais.

\section{Prediction Tools}
\subsection{Cloud Harmony}
O projeto {\em CloudHarmony}, cujo
objetivo é ``tornar-se a principal fonte independente, imparcial e útil de
métricas de desempenho de provedores de nuvem''~\cite{cloudharmony}, agrega
dados de testes de desempenho realizados desde 2009 em mais de 60 provedores de
nuvem. Além do histórico das avaliações, o {\em CloudHarmony} disponibiliza uma ferramenta para executar novas avaliações de desempenho a qualquer momento, denominada
\textit{Cloud Speed Test},\footnote{\url{http://cloudharmony.com/speedtest}}, a qual permite realizar quatro tipos de teste:

\begin{description}
  \item[\em Download a few large files] --- objetiva determinar o melhor provedor
  para descarregar arquivos grandes, sendo útil para aplicações como {\em video
  streaming};
  \item[\em Download many small files] --- objetiva determinar o melhor provedor
  para descarregar arquivos pequenos, podendo ser útil para hospedar uma página
  web, por exemplo;
  \item[\em Upload] --- útil para avaliar serviços que serão utilizados para
  envio de arquivos;
  \item[\em Test network latency] --- a latência afeta o tempo de resposta da
  aplicação e geralmente está relacionada com a região de onde o teste está
  partindo.
\end{description}

Os resultados disponibilizados pelo {\em CloudHarmony} têm como pontos fortes a
grande quantidade de dados de testes de desempenho disponíveis, além da possibilidade do cliente da
nuvem poder executar novos testes a qualquer tempo. Por outro lado, os testes estão limitados àqueles implementados pela ferramenta de teste, não podendo ser facilmente modificados para contemplar novas métricas ou cenários de avaliação.

\subsection{{\em CloudXplor}}

{\em CloudXplor}~\cite{malkowski2010cloudxplor} é uma ferramenta para
planejamento de configuração de recursos da nuvem baseada em dados empíricos. A ferramenta
foi desenvolvida tomando como base um modelo de planejamento de configuração de
recursos de Tecnologia da Informação (TI), com foco explícito em aspectos econômicos.
Por essa razão, a ferramenta se utiliza de acordos de nível de serviço baseados na relação do custo
da infraestrutura de TI com o valor dos recursos do provedor do serviço. Esse
valor será maior quando o tempo de resposta da aplicação for plenamente atendido
pelo provedor do serviço, e vai diminuindo à medida em que esse tempo de resposta
não é alcançado.

Os dados empíricos precisam ser coletados, previamente, através da execução de diversos experimentos de avaliação de desempenho. Esses dados
são compostos por métricas de sistema (uso de CPU, memória utilizada, tráfego na rede e E/S de disco) e métricas de mais alto nível (tempo de resposta e \textit{throughput}). Após a coleta, os dados dos experimentos são submetidos e analisados pela ferramenta, utilizando um de seus quatro módulos: análise de tempo de
resposta, análise de \textit{throughput}, análise do valor agregado e do custo,
e análise do lucro. Cada um desses módulos filtra os dados, fazendo uso apenas
das informações necessárias para a execução da sua análise. Após a análise dos
 dados, a ferramenta pode ser utilizada para produzir gráficos que ilustram o comportamento da aplicação ao se variar
parâmetros como carga de trabalho e configuração dos componentes da aplicação.


A ferramenta proposta no trabalho faz uso de dados previamente coletados
e só então realiza a análise do desempenho de aplicações na nuvem
em diversos recursos. Por isso deixa a cargo do cliente da nuvem todas as
atividades para a avaliação de desempenho, pois depende dos dados empíricos que
são coletados após a avaliação. O CloudXplor é capaz de plotar um gráfico com o
comportamento da aplicação à medida em que é exposta a variação na demanda, do
custo e do valor, mas só o faz por que o cliente da nuvem executou cada uma das
avaliações manualmente. Da mesma forma a variação na demanda vai depender do
conjunto de avaliações executadas pelo cliente da nuvem, a ferramenta apenas
plota gráficos do que foi executado pelo cliente. Com relação a definição de
cenários de avaliação e a definição de parâmetros de desempenho, a ferramenta
não dá nenhum suporte, uma vez que os cenários dependem das avaliações
realizadas e que os módulos de análise de custo, valor, tempo de resposta e {\em
throughput} possuem uma lista dos parâmetros de desempenho possíveis de
avaliação.


\subsection{CloudCmp}
\citeonline{li2011} apresentam uma ferramenta para apoiar a avaliação e a comparação do
desempenho e do custo dos recursos e serviços de diversos provedores de nuvem
pública, de modo a auxiliar o cliente da nuvem a escolher o provedor mais adequado para a sua
aplicação. Essa ferramenta, denominada {\em CloudCmp}, analisa
os serviços de elasticidade, persistência de dados e rede oferecidos pelos provedores
de interesse, com base em resultados previamente coletados a partir da execução de diversos
{\em benchmarks}: uma versão modificada do {\em SPECjvm2008}~\cite{SPECjvm2008}
para avaliar a característica de elasticidade do provedor; um cliente Java para avaliar os serviços de
armazenamento e persistência de dados; e as ferramentas {\em iperf}\footnote{http://iperf.sourceforge.net. }
e {\em ping} para avaliar os serviços de rede. Após a fase inicial da coleta dos dados, a ferramenta pode ser utilizada para gerar gráficos que auxiliem o
cliente da nuvem a comparar o desempenho dos recursos de cada um
dos provedores nos quais as avaliações foram realizadas, que assim poderá escolher o provedor e os recursos mais apropriados para as necessidades e demandas específicas de suas aplicações. 

%Os gráficos gerados nos
%experimentos apresentados neste trabalho foram comparados com os gerados após o
%estudo do comportamento de três aplicações simples implantadas na nuvem. Essas
%comparações mostraram que as previsões do CloudCmp refletiram o comportamento
%das aplicações testadas.

Segundo \citeonline{li2011}, até a época do trabalho não houve
nenhum provedor de nuvem que se destacasse com relação aos demais. Outra constatação foi de que os resultados
obtidos a partir da execução dos {\em benchmarks} em cada provedor apenas refletiam o momento em que foram coletados, uma vez que a estrutura utilizada pelos provedores para hospedar seus serviços sofre frequentes modificações e a demanda por seus recursos computacionais é bastante variável.

Como essa solução compara serviços e recursos da nuvem através da análise de dados
previamente coletados a partir da execução de diferentes {\em benchmarks}, a
escolha dos provedores e recursos mais apropriados para uma determinada aplicação só será eficaz se a aplicação utilizar os recursos da nuvem de forma semelhante à dos {\em benchmarks} avaliados. Além disso, a
ferramenta não oferece suporte à execução de novos cenários de avaliação na nuvem, estando limitada àqueles previamente definidos para os respectivos {\em bechmarks}.

\subsection{CloudAdvisor}
\cite{jung2013cloudadvisor}

\subsection{CDOSim}
\cite{fittkau2012cdosim}

\subsection{CloudProphet}
Em \cite{li2011cloudprophet} os autores apresentam o CloudProphet, um sistema de
predição de desempenho de aplicações em ambiente de nuvem computacional baseado 
na metodologia de ``rastrear e reproduzir'' (\textit{trace and replay}).

O CloudProphet não testa a aplicação do cliente de fato no ambiente de nuvem. De
modo contrário, ele injeta na implantação original da aplicação um módulo que 
registra um rastreamento detalhado dos eventos de utilização de recursos de CPU,
armazenamento e rede em cada componente da aplicação durante um período de 
execução habitual em seu ambiente de produção.

Em um passo seguinte, outro módulo faz uma extração das relações de dependência 
entre os eventos coletados, ordenando as transações executadas nos diversos 
componentes.

O terceiro passo é a reprodução dos eventos coletados durante a fase de 
rastreamento. Essa reprodução consiste fazer com que o ambiente de nuvem 
computacional que se deseja avaliar execute as transações representadas nos dados
do rastreamento a partir de requisições que partem de clientes simulados.
   
O objetivo do CloudBench, segundo os autores é eliminar o custo e o trabalho 
envolvidos na migração da aplicação real para a nuvem para a execução de testes 
antes que seja de fato tomada a decisão em favor dessa migração.

Os autores argumentam que a simples implantação da aplicação no ambiente de um
serviço de nuvem computacional já incorre em custos, que podem ser altos a 
depender do tamanho ou da arquitetura da aplicação. Além disso, a tarefa de
migração pode ser bastante trabalhosa conforme o número e a diversidade dos 
componentes da aplicação, que podem acarretar dificuldades de configuração e
compatibilidade no novo ambiente.

\section{Trial-And-Test Tools}
\subsection{Expertus}
Um outro trabalho recente foi descrito em~\cite{jayasinghe2012}, onde os autores propõem um ambiente para execução automática de testes de desempenho em nuvens IaaS.  Nesse ambiente, chamado Expertus, os usuários implementam testes de desempenho através da geração e customização de \textit{scripts} Shell a partir de {\em templates} especificados na forma de documentos XML~\cite{jayasinghe2012}. 

\subsection{CloudBench}
\cite{silva2013cloudbench} descrevem o CloudBench como um arcabouço para 
automação dos testes e avaliação do desempenho de ambientes de nuvem computacional 
sob o modelo IaaS. CloudBench prevê métricas que dizem respeito não só à aplicação,
mas também ao provedor, como latência de provisionamento, bem como a execução dos 
testes e a coleta de dados relativos às métricas de desempenho observadas, através 
da abstração de experimentos, aplicações e máquinas virtuais.   

As abstrações criadas pelo CloudBench permitem que um experimento seja 
especificado através de uma lista de diretivas que descrevem os itens
que compõem o experimento, definindo objetos como a aplicação e as instâncias de 
máquinas virtuais utilizadas.

CloudBench faz a implantação automática da aplicação a ser executada para
efeito de testes, desde a criação da máquina virtual no ambiente até a coleta dos
dados de desempenho e desligamento das máquinas, com suporte a diversos provedores.

Essas características fazem do CloudBench uma ferramenta muito poderosa para a
automação de testes e coleta de dados para análise das execuções. Suas ferramentas
de monitoramento fornecem informações com grandes níveis de detalhamento a respeito
de cada componente implantado e usado nos testes, proporcionando excelente embasamento
para a tomada de decisão.

Entretanto, embora o CloudBench tenha um escopo de solução muito mais amplo, 
voltado para a avaliação de desempenho tanto da aplicação do cliente como do 
provisionamento de máquinas pelo provedor, seu alvo no momento da execução de 
testes está restrito a benchmarks pré-definidos, não permitindo a execução de 
uma aplicação real no ambiente testado.

Neste sentido, o arcabouço que propomos com este trabalho se diferencia pelo fato
de ser agnóstico em relação à aplicação que deverá ser testada, assim como quanto 
às métricas que a ela concernem, conferindo ao usuário da solução a oportunidade 
de avaliar o comportamento da aplicação de seu interesse implantada no ambiente
pretendido e sob a perspectiva que lhe for mais conveniente, inclusive em termos
de arquitetura de implantação. 

\subsection{Cloud Crawler}
Este trabalho apresenta um ambiente programável para apoiar os usuários de nuvens IaaS na realização de testes automáticos de desempenho de aplicações na nuvem. As principais contribuições do ambiente são: a linguagem declarativa {\em Crawl}, com a qual os usuários podem especificar, através de uma notação simples e de alto nível de abstração, uma grande variedade de cenários de avaliação de desempenho de uma aplicação na nuvem; e o motor de execução {\em Crawler}, que automaticamente executa e coleta os resultados dos cenários descritos em {\em Crawl} em um ou mais provedores. Essas duas ferramentas são denominadas conjuntamente de {\em Cloud Crawler}~\cite{cunhacloud}.

Para iniciar os testes de desempenho de uma aplicação através do ambiente {\em Cloud Crawler}, os componentes dessa aplicação precisam ser declarados em um \textit{script} da linguagem {\em Crawl}. Compõem esse \textit{script Crawl}, por exemplo, o provedor de nuvem, os tipos de máquinas virtuais e as máquinas virtuais que serão utilizadas nas avaliações, além disso, métricas de desempenho e a demanda imposta à aplicação também irão compor o cenário de avaliação que é declarado no \textit{script Crawl}. Finalizada essa etapa de declaração, o usuário do ambiente irá submeter o \textit{script crawl} para o motor de execução {\em Crawler}. Esse motor irá iniciar todas as máquinas virtuais, caso seja necessário, irá proceder com a modificação do tipo de máquina virtual, de acordo com o que estiver declarado. Após a inicialização de cada máquina virtual, o motor pode executar alguma configuração nessa máquina, por exemplo, a configuração do endereço ip de um banco de dados, ou a configuração do total de memória utilizado por uma máquina virtual java. Todas as configurações necessárias para a aplicação executar na nuvem devem estar declaradas no {\em script Crawl} que foi submetido para o motor. Quando a última máquina virtual é configurada, o motor {\em Crawler} executa um por um os cenários de avaliação, com suas respectivas demandas, e ao mesmo tempo coleta as métricas de desempenho especificadas. As métricas de desempenho podem ser tanto métricas de sistema, como percentual de CPU utilizado e de memória RAM, quanto métricas de aplicação, como o tempo de resposta de uma aplicação WEB.

A fase de mapeamento dos componentes da aplicação é realizada apenas uma vez, enquanto que a submissão para o motor de execução pode ser repetida ao critério do usuário. Ambientes como o {\em Cloud Crawler} permitem que os seus usuários testem suas aplicaçãoes em difirentes cenário de implantação e possibilitam que o mesmo entenda o comportamento da sua aplicação à medida em que ela é submetida a diferentes demandas e implantada em diferentes configurações, porém, a qualidade da avaliação de desempenho dependerá da qualidade dos cenários de testes que os usuários declararem, uma vez que o ambiente não decide qual será a nova configuração testada. O ambiente apenas segue aquilo que foi declarado pelo usuário.

\subsection{Cloud WorkBench}   
Em~\cite{scheuner2014cloud}



\section{Outros Trabalhos e Considerações}

******

\textbf{Caracteristicas gerais do Cloud Capacitor}

Foco em minimização dos custos de execução dos testes

Proposta de análise do impacto da escalabilidade horizontal e/ou vertical nos 
resultados dos testes

Julgamento de qual tipo de máquina apresenta melhor custo x desempenho para cada
faixa de workload

Flexibilidade de implementação da lógica de avaliação através do arcabouço de
heurísticas

******

