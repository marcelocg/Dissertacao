\chapter[Trabalhos Relacionados]{Trabalhos Relacionados}
% ----------------------------------------------------------
Apresentamos  neste capítulo alguns trabalhos cujos objetivos estejam alinhados 
com a ideia da avaliação de desempenho de aplicações executadas em ambientes de
computação em nuvem. Esses trabalhos estão agrupados de acordo com a abordagem utilizada para a realização da avaliação de desempenho. São duas abordagens, a primeira é a abordagem preditiva, nesta, os trabalhos não executam diretamente a aplicação alvo no ambiente onde se deseja implantá-la. Já a segunda abordagem é a empírica, nesta as aplicações alvo são implantadas na nuvem e então submetidas a testes de carga. 

Para apoiar a análise crítica de cada trabalho e suas abordagens, definimos um conjunto de critérios a partir dos quais poderemos tanto descrever, quanto comparar as soluções existentes de apoio a avaliação de desempenho do ponto de vista do usuário. Seguem os critérios:

\begin{enumerate}
  \item Completude da solução
  \begin{enumerate}
    \item Definição da aplicação --- flexibilidade da solução para
    definir a aplicação a ser avaliada;
    \item Definição da demanda --- flexibilidade da solução para
    definir os níveis de carga de trabalho sobre os quais a aplicação será
    submetida;
	\item Definição dos recursos da nuvem --- flexibilidade da solução para
	definir as configurações de recursos do provedor sobre as quais a aplicação
	será executada;
	\item Definição do acordo de nível de serviço (SLA) --- flexibilidade da
	solução para definir o SLA desejado.
  \end{enumerate}
  \item Efetividade da solução  
  \begin{enumerate}
    \item Eficiência --- tempo e custo necessários para a solução resolver o
    problema;
    \item Acurácia --- confiabilidade das respostas oferecidas pela solução;
	\item Complexidade --- grau de complexidade/esforço exigido do usuário da
	solução.
  \end{enumerate}
\end{enumerate}

Cada abordagem será apresentada em uma subseção, bem como os diversos trabalhos da área que serão resumidos separadamente. Ao final da subseção, será apresentada uma análise crítica que será baseada nos critérios descritos acima.

\section{Ferramentas com Abordagem Preditiva}
\subsection{Cloud Harmony}
O projeto {\em CloudHarmony}, cujo
objetivo é ``tornar-se a principal fonte independente, imparcial e útil de
métricas de desempenho de provedores de nuvem''~\cite{cloudharmony}, agrega
dados de testes de desempenho realizados desde 2009 em mais de 60 provedores de
nuvem. Além do histórico das avaliações, o {\em CloudHarmony} disponibiliza uma ferramenta para executar novas avaliações de desempenho a qualquer momento, denominada
\textit{Cloud Speed Test},\footnote{\url{http://cloudharmony.com/speedtest}}, a qual permite realizar quatro tipos de teste:

\begin{description}
  \item[\em Download a few large files] --- objetiva determinar o melhor provedor
  para descarregar arquivos grandes, sendo útil para aplicações como {\em video
  streaming};
  \item[\em Download many small files] --- objetiva determinar o melhor provedor
  para descarregar arquivos pequenos, podendo ser útil para hospedar uma página
  web, por exemplo;
  \item[\em Upload] --- útil para avaliar serviços que serão utilizados para
  envio de arquivos;
  \item[\em Test network latency] --- a latência afeta o tempo de resposta da
  aplicação e geralmente está relacionada com a região de onde o teste está
  partindo.
\end{description}

Os resultados disponibilizados pelo {\em CloudHarmony} têm como pontos fortes a
grande quantidade de dados de testes de desempenho disponíveis, além da possibilidade do cliente da
nuvem poder executar novos testes a qualquer tempo. Por outro lado, os testes estão limitados àqueles implementados pela ferramenta de teste, não podendo ser facilmente modificados para contemplar novas métricas ou cenários de avaliação.

\subsection{{\em CloudXplor}}

{\em CloudXplor}~\cite{malkowski2010cloudxplor} é uma ferramenta para
planejamento de configuração de recursos da nuvem baseada em dados empíricos. A ferramenta
foi desenvolvida tomando como base um modelo de planejamento de configuração de
recursos de Tecnologia da Informação (TI), com foco explícito em aspectos econômicos.
Por essa razão, a ferramenta se utiliza de acordos de nível de serviço baseados na relação do custo
da infraestrutura de TI com o valor dos recursos do provedor do serviço. Esse
valor será maior quando o tempo de resposta da aplicação for plenamente atendido
pelo provedor do serviço, e vai diminuindo à medida em que esse tempo de resposta
não é alcançado.

Os dados empíricos precisam ser coletados, previamente, através da execução de diversos experimentos de avaliação de desempenho. Esses dados
são compostos por métricas de sistema (uso de CPU, memória utilizada, tráfego na rede e E/S de disco) e métricas de mais alto nível (tempo de resposta e \textit{throughput}). Após a coleta, os dados dos experimentos são submetidos e analisados pela ferramenta, utilizando um de seus quatro módulos: análise de tempo de
resposta, análise de \textit{throughput}, análise do valor agregado e do custo,
e análise do lucro. Cada um desses módulos filtra os dados, fazendo uso apenas
das informações necessárias para a execução da sua análise. Após a análise dos
 dados, a ferramenta pode ser utilizada para produzir gráficos que ilustram o comportamento da aplicação ao se variar
parâmetros como carga de trabalho e configuração dos componentes da aplicação.


A ferramenta proposta no trabalho faz uso de dados previamente coletados
e só então realiza a análise do desempenho de aplicações na nuvem
em diversos recursos. Por isso deixa a cargo do cliente da nuvem todas as
atividades para a avaliação de desempenho, pois depende dos dados empíricos que
são coletados após a avaliação. O CloudXplor é capaz de plotar um gráfico com o
comportamento da aplicação à medida em que é exposta a variação na demanda, do
custo e do valor, mas só o faz por que o cliente da nuvem executou cada uma das
avaliações manualmente. Da mesma forma a variação na demanda vai depender do
conjunto de avaliações executadas pelo cliente da nuvem, a ferramenta apenas
plota gráficos do que foi executado pelo cliente. Com relação a definição de
cenários de avaliação e a definição de parâmetros de desempenho, a ferramenta
não dá nenhum suporte, uma vez que os cenários dependem das avaliações
realizadas e que os módulos de análise de custo, valor, tempo de resposta e {\em
throughput} possuem uma lista dos parâmetros de desempenho possíveis de
avaliação.


\subsection{CloudCmp}
\citeonline{li2011} apresentam uma ferramenta para apoiar a avaliação e a comparação do
desempenho e do custo dos recursos e serviços de diversos provedores de nuvem
pública, de modo a auxiliar o cliente da nuvem a escolher o provedor mais adequado para a sua
aplicação. Essa ferramenta, denominada {\em CloudCmp}, analisa
os serviços de elasticidade, persistência de dados e rede oferecidos pelos provedores
de interesse, com base em resultados previamente coletados a partir da execução de diversos
{\em benchmarks}: uma versão modificada do {\em SPECjvm2008}~\cite{SPECjvm2008}
para avaliar a característica de elasticidade do provedor; um cliente Java para avaliar os serviços de
armazenamento e persistência de dados; e as ferramentas {\em iperf}\footnote{http://iperf.sourceforge.net. }
e {\em ping} para avaliar os serviços de rede. Após a fase inicial da coleta dos dados, a ferramenta pode ser utilizada para gerar gráficos que auxiliem o
cliente da nuvem a comparar o desempenho dos recursos de cada um
dos provedores nos quais as avaliações foram realizadas, que assim poderá escolher o provedor e os recursos mais apropriados para as necessidades e demandas específicas de suas aplicações. 

%Os gráficos gerados nos
%experimentos apresentados neste trabalho foram comparados com os gerados após o
%estudo do comportamento de três aplicações simples implantadas na nuvem. Essas
%comparações mostraram que as previsões do CloudCmp refletiram o comportamento
%das aplicações testadas.

Segundo \citeonline{li2011}, até a época do trabalho não houve
nenhum provedor de nuvem que se destacasse com relação aos demais. Outra constatação foi de que os resultados
obtidos a partir da execução dos {\em benchmarks} em cada provedor apenas refletiam o momento em que foram coletados, uma vez que a estrutura utilizada pelos provedores para hospedar seus serviços sofre frequentes modificações e a demanda por seus recursos computacionais é bastante variável.

Como essa solução compara serviços e recursos da nuvem através da análise de dados
previamente coletados a partir da execução de diferentes {\em benchmarks}, a
escolha dos provedores e recursos mais apropriados para uma determinada aplicação só será eficaz se a aplicação utilizar os recursos da nuvem de forma semelhante à dos {\em benchmarks} avaliados. Além disso, a
ferramenta não oferece suporte à execução de novos cenários de avaliação na nuvem, estando limitada àqueles previamente definidos para os respectivos {\em bechmarks}.

\subsection{CloudAdvisor}
O trabalho apresentado em~\cite{jung2013cloudadvisor} ``introduz uma nova plataforma de recomendação de nuvem, chamada {\em CloudAdvisor}''. Essa plataforma destina-se a auxiliar o seu usuário na tarefa de capturar as implicações monetárias e financeiras das configurções de implantação das suas aplicações. Para recomendar a configuração, a platforma recebe como entrada parâmetros de configuração de alto nível como, orçamento, expectativa de performance e economia de energia, os quais estão limitados a uma escala discreta que vai de 0 até 10, onde 0 significa baixa influência e 10 significa alta influência. Uma vez informados os parâmetros de configuração, o {\em CloudAdvisor} irá caracterizar a performance da aplicação alvo em termos de uso dos recursos computacionais e em seguida executará o {\em benchmark} {\em CloudMeter},~\cite{jung2013cloudadvisor}, nas nuvens candidatas, a fim de caracterizar a performance dos recursos dessas nuvens.

Para ilustrar o uso do {\em CloudAdvisor}, os autores implantaram a solução em servidores locais e em três provedores de nuvem pública, foram eles: Windows Azure, Rackspace, and Amazon EC2. Como resultado, foi observado que a taxa de erro da configuração para uma determinada carga foi de 10~\%. No entanto, quando o usuário do ambiente escolhe parâmetros de configuração extremos (por exemplo, configurar o máximo de orçamento, de economia de energia e de nível de performance), essa taxa de erro elevou para 18~\%.

Os autores concluem que o usuário da solução pode explorar diversas opções de configuração da sua aplicação na nuvem utilizando uma interface amigável e sem a necessidade de informar detalhes específicos de configuração. Além disso, mostraram que é possível utilizar uma técnica de caracterização de performance, para uma dada carga, baseada na execução de {\em benchmarks} em nuvens candidatas.

\subsection{CDOSim}
Uma solução para o desafio da escolha da configuração de implantação de uma aplicação na nuvem foi apresentada em~\cite{fittkau2012cdosim} com o nome de \textit{CDOSim}. Essa ferramenta auxilia o usuário no processo de escolha do que os autores chamaram de opção de implantação na nuvem, do inglês \textit{Cloud Deployment Option} (CDO). Uma vez que a análise manual das ``potenciais CDOs é intratável, custosa e consome tempo, devido à heterogeneidade dos ambientes de nuvem'',~\cite{fittkau2012cdosim}. Para a escolha das CDOs, são realizadas simulações que se baseiam no custo e nas propriedades de performance de cada CDO. Como requisito para a utilização de uma CDO nas simulações, é preciso caracterizar as suas propriedades de performance em termos de \textit{mega integer plus instructions per second} (MIPIPS)~\cite{fittkau2012cdosim}, essa caracterização é realizada através da execução de um \textit{benchmark} cujo código é gerado para cada linguagem de programação utilizada pela aplicação alvo. Esta, por sua vez, deve passar por um processo de engenharia reversa para modelos KDM, que é descrito em~\cite{perez2011knowledge}, para que o simulator consiga escolher a opção de configuração mais adequada.

Para validar o \textit{benchmark} MIPIPS, os resultados da simulação em comparação com dados reais e a predição da performance de um provedor de nuvem com base nos dados de outro provedor de nuvem, os autores conduziram uma série de experimentos em dois ambientes de nuvem, uma pública a Amazon EC2 e outra privada. Na nuvem pública foi evidenciado que os valores de MIPIPS dependem da região onde o \textit{benchmark} foi realizado e da carga sobre a máquina física que hospeda a vitual. Já na comparação dos resultados da simulação com os dados reais, os autores mostraram que a taxa de error da utilização de CPU simulada com a utilização de CPU medida chegou a 30,86~\%, contudo, como a taxa de erro médio global foi abaixo de 22,75~\%, abaixo do threshold estabelecido pelos autores que era de 30~\%. Já a predição da performance de uma instância da Amazon EC2 a partir da performance de uma instância da nuvem privada gerou um 15,76~\% como taxa de erro global.

Finalmente, os autores concluem que a simulação pode auxiliar no processo de escolha da opção de implantação com maior performance e menor custo e que os resultados da simulação são razoavelmente próximos dos experimentos realizados dos valores reais.

\subsection{CloudProphet}
Em \cite{li2011cloudprophet} os autores apresentam o \textit{CloudProphet}, um sistema de
predição de desempenho de aplicações em ambiente de nuvem computacional baseado 
na metodologia de ``rastrear e reproduzir'' (\textit{trace and replay}).

O \textit{CloudProphet} não testa a aplicação do cliente de fato no ambiente de nuvem. De
modo contrário, ele injeta na implantação original da aplicação um módulo que 
registra um rastreamento detalhado dos eventos de utilização de recursos de CPU,
armazenamento e rede em cada componente da aplicação durante um período de 
execução habitual em seu ambiente de produção.

Em um passo seguinte, outro módulo faz uma extração das relações de dependência 
entre os eventos coletados, ordenando as transações executadas nos diversos 
componentes.

O terceiro passo é a reprodução dos eventos coletados durante a fase de 
rastreamento. Essa reprodução consiste fazer com que o ambiente de nuvem 
computacional que se deseja avaliar execute as transações representadas nos dados
do rastreamento a partir de requisições que partem de clientes simulados.
   
O objetivo do \textit{CloudBench}, segundo os autores é eliminar o custo e o trabalho 
envolvidos na migração da aplicação real para a nuvem para a execução de testes 
antes que seja de fato tomada a decisão em favor dessa migração.

Os autores argumentam que a simples implantação da aplicação no ambiente de um
serviço de nuvem computacional já incorre em custos, que podem ser altos a 
depender do tamanho ou da arquitetura da aplicação. Além disso, a tarefa de
migração pode ser bastante trabalhosa conforme o número e a diversidade dos 
componentes da aplicação, que podem acarretar dificuldades de configuração e
compatibilidade no novo ambiente.

\section{Ferramentas com Abordagem Empírica}
As ferramentas que serão apresentadas nesta seção têm em comum o fato de utilizarem como aplicação alvo a prória aplicação que se deseja implantar na nuvem. Portanto, é preciso inicialmente realizar uma implantação na nuvem para que seja dado início ao processo de análise de desempenho. Por isso, cada ferramenta oferece um mecanismo para que o seu usuário possa definir como a aplicação deve ser implantada e configurada. Além da definição da aplicação e dos recursos da nuvem que serão utilizados, essas ferramentas também permitem que sejam definidas a demanda que será imposta a cada aplicação e o acordo de nível de serviço. Dessa forma, é possível definir, por exemplo, o número de usuários simultâneos e o tempo de resposta esperado para uma transação.

Uma vez que que as ferramentas que realizam a abordagem empírica possibilitam ao usuário muita liberdade na definição da aplicação, demanda, recurso da nuvem e SLA, essas soluções têm o mais alto grau de completude. Além disso, como faz-se uso da própria aplicação alvo para a avaliação de desempenho, a acurácia dos resultados apresentados pelas ferramentas é a mais elevada. A seguir, apresentaremos quatro trabalhos que se destacam na avaliação de desempenho de aplicações na nuvem usando a abordagem empírica, são eles:~\cite{jayasinghe2012},~\cite{silva2013cloudbench},~\cite{cunhacloud},~\cite{scheuner2014cloud}.

\subsection{Expertus}
Devido à complexidade e às implicações da escolha da configuração para a implantação de uma aplicação na nuvem, em~\cite{jayasinghe2012} os autores apresentam o \textit{Expertus}, que é descrito como ``um framework flexível de geração de código para automatizar testes de performance de aplicações distribuídas em nuvem de infraestrutura''. Essa geração automática de código é realizada a partir de {\em templates} especificados na forma de documentos XML~\cite{jayasinghe2012}. Os templates utilizados nas avaliações de desempenho devem ser escritos pelo usuário e servem de entrada para o ambiente, que realiza diversas transformações nessa entrada até a forma de \textit{shell scripts}. Esses \textit{scripts}, por fim, possuem os comandos para a configuração da avaliação de desempenho na aplicação alvo.

Como demonstração da usabilidade da ferramenta, os autores apresentaram em~\cite{jayasinghe2012} resultados de experimentos realizados com duas aplicações alvo. Cada uma das aplicações foi avaliada com duas opções de sistemas de gerenciamento de bancos de dados, o que demonstrou também como diferentes opções de configuração poderiam ser utilizadas nas aplicações. Além da demonstração da usabilidade, os autores realizaram experimentos para evidenciar a magniture e tipos de \textit{scripts} que podem ser gerados pela ferramenta. Como exemplo da magnitute, para a realização de experimentos com 48 nós, o total de linhas de scripts geradas pelo \textit{Expertus} girou em torno de 15 mil. Por fim, os autores demonstraram o que chamaram de ``riqueza da ferramenta'', que foi comprovada através da execução de experimentos em 5 nuvens (por exemplo, Amazon EC2 e Open Cirrus~\cite{avetisyan2010open}).

Dessa forma, pode-se concluir que a ferramenta apresentada minimiza a ocorrência de falhas humanas na avaliação de desempenho de uma aplicação implantada em diversos nós. Além disso, os mesmo experimentos podem ser repetidos em diferentes provedores de nuvem pública. De modo que mais cenários de implantação podem ser considerados para a escolha do mais adequado para a aplicação.

\subsection{CloudBench}
\cite{silva2013cloudbench} descreve o \textit{CloudBench} como um arcabouço para automação da avaliação de desempenho de ambientes de nuvem computacional sob o modelo IaaS. As abstrações apresentadas neste trabalho permitem que um experimento seja especificado através de uma lista de diretivas as quais descrevem os itens que compõem o experimento. São exemplos desses itens, objetos como a aplicação alvo, as instâncias de máquinas virtuais utilizadas, e as métricas de desempenho que são tanto relativas à aplicação alvo, quanto ao serviço do provedor de nuvem (por exemplo, latência de provisionamento).
%Para a análise dos resultados, o ambiente coleta dados relativos às métricas de desempenho observadas, através %da abstração de experimentos, aplicações e máquinas virtuais.
%Além disso, o ambiente prevê métricas que dizem respeito não só à aplicação, mas também ao provedor, como latência de provisionamento.
%, bem como a execução dos 
%testes e a coleta de dados relativos às métricas de desempenho observadas, através 
%da abstração de experimentos, aplicações e máquinas virtuais.   

Para a realização dos experimentos, o \textit{CloudBench} faz a implantação automática da aplicação a ser executada para efeito de testes. Portanto, o acompanhamento é realizado desde a criação da máquina virtual no ambiente até a coleta dos dados de desempenho e desligamento das máquinas. Essas características fazem do CloudBench uma ferramenta muito poderosa para a
automação de testes e coleta de dados para análise das execuções. Suas ferramentas
de monitoramento fornecem informações com grandes níveis de detalhamento a respeito
de cada componente implantado e usado nos testes, proporcionando excelente embasamento
para a tomada de decisão.

Entretanto, embora o CloudBench tenha um escopo de solução muito mais amplo, 
voltado para a avaliação de desempenho tanto da aplicação do cliente como do 
provisionamento de máquinas pelo provedor, seu alvo no momento da execução de 
testes está restrito a \textit{benchmarks} pré-definidos, não permitindo a execução de 
uma aplicação real no ambiente testado.

%Neste sentido, o arcabouço que propomos com este trabalho se diferencia pelo fato
%de ser agnóstico em relação à aplicação que deverá ser testada, assim como quanto 
%às métricas que a ela concernem, conferindo ao usuário da solução a oportunidade 
%de avaliar o comportamento da aplicação de seu interesse implantada no ambiente
%pretendido e sob a perspectiva que lhe for mais conveniente, inclusive em termos
%de arquitetura de implantação. 

\subsection{Cloud Crawler}
Este trabalho apresenta um ambiente programável para apoiar os usuários de nuvens IaaS na realização de testes automáticos de desempenho de aplicações na nuvem. As principais contribuições do ambiente são: a linguagem declarativa {\em Crawl}, com a qual os usuários podem especificar, através de uma notação simples e de alto nível de abstração, uma grande variedade de cenários de avaliação de desempenho de uma aplicação na nuvem; e o motor de execução {\em Crawler}, que automaticamente executa e coleta os resultados dos cenários descritos em {\em Crawl} em um ou mais provedores. Essas duas ferramentas são denominadas conjuntamente de {\em Cloud Crawler}~\cite{cunhacloud}.

Para iniciar os testes de desempenho de uma aplicação através do ambiente {\em Cloud Crawler}, os componentes dessa aplicação precisam ser declarados em um \textit{script} da linguagem {\em Crawl}. Compõem esse \textit{script Crawl}, por exemplo, o provedor de nuvem, os tipos de máquinas virtuais e as máquinas virtuais que serão utilizadas nas avaliações, além disso, métricas de desempenho e a demanda imposta à aplicação também irão compor o cenário de avaliação que é declarado no \textit{script Crawl}. Finalizada essa etapa de declaração, o usuário do ambiente irá submeter o \textit{script crawl} para o motor de execução {\em Crawler}. Esse motor irá iniciar todas as máquinas virtuais, caso seja necessário, irá proceder com a modificação do tipo de máquina virtual, de acordo com o que estiver declarado. Após a inicialização de cada máquina virtual, o motor pode executar alguma configuração nessa máquina, por exemplo, a configuração do endereço ip de um banco de dados, ou a configuração do total de memória utilizado por uma máquina virtual java. Todas as configurações necessárias para a aplicação executar na nuvem devem estar declaradas no {\em script Crawl} que foi submetido para o motor. Quando a última máquina virtual é configurada, o motor {\em Crawler} executa um por um os cenários de avaliação, com suas respectivas demandas, e ao mesmo tempo coleta as métricas de desempenho especificadas. As métricas de desempenho podem ser tanto métricas de sistema, como percentual de CPU utilizado e de memória RAM, quanto métricas de aplicação, como o tempo de resposta de uma aplicação WEB.

A fase de mapeamento dos componentes da aplicação é realizada apenas uma vez, enquanto que a submissão para o motor de execução pode ser repetida ao critério do usuário. Ambientes como o {\em Cloud Crawler} permitem que os seus usuários testem suas aplicaçãoes em difirentes cenário de implantação e possibilitam que o mesmo entenda o comportamento da sua aplicação à medida em que ela é submetida a diferentes demandas e implantada em diferentes configurações, porém, a qualidade da avaliação de desempenho dependerá da qualidade dos cenários de testes que os usuários declararem, uma vez que o ambiente não decide qual será a nova configuração testada. O ambiente apenas segue aquilo que foi declarado pelo usuário.

\subsection{Cloud WorkBench}
Uma vez que a escolha da infraestrutura computacional ótima para hospedar uma determinada aplicação na nuvem se trata de uma tarefa que ``exige a avaliação de custos e performances de diferentes combinações de configurações''~\cite{scheuner2014cloud}. Onde os autores propõem uma arquitetura e uma implementação concreta dessa arquitetura para automatizar a realização de avaliações em serviços da nuvem. O {\em Cloud WorkBench}, nome dado à solução apresentada neste trabalho, adota noções de Infraestrutura como Código, do inglês {\em Infrastructure-as-Code} (IaC)~\cite{huttermann2012devops}, para a realização dessas avalições. Dessa forma, as ações necessárias para o provisionamento dos recursos utilizados pela aplicação encontram-se todas codificadas.

Para ilustrar o uso do \textit{Cloud WorkBench}, foi realizado um pequeno experimento para avaliar a velocidade de escrita sequencial em disco de recursos da nuvem. Nesse experimento, foram utilizados três perfis de recursos computacionais da Amazon EC2 na região Irlanda (\textit{eu-west-1}), em servidores utilizando o sistema operacional Ubuntu 14.04. Para cada perfil de recurso, foram realizadas entre 8 e 12 execuções do benchmark FIO\footnote{http://git.kernel.org/cgit/linux/kernel/git/axboe/fio.git} 2.1.10. Conforme descrito em~\cite{scheuner2014cloud}, o experimento evidenciou que há diferenças na performance dos perfis de recursos utilizados. Esse diferença poderia se refletir na performance de uma aplicação que fizesse muita escrita em disco.

Após a análise dos resultados, os autores concluem que o \textit{Cloud WorkBench} suporta a realização de experimentos em nuvens de infraestrutura, e que toda a complexidade da configuração do ambiente pode ficar codificada. O que diminui a ocorrência de erros decorrentes de eventuais intervenções manuais.

\subsection{Análise Técnica da Metodologia}
Apesar das soluções de abordagem empírica terem destaque no critério da completude, no que diz respeito à efetividade, elas possuem desempenho moderado nos critérios relacionados à eficiência e à complexidade, enquanto que possuem alta acurácia. No que diz respeito à eficiência, essas ferramentas poderiam fazer uso de resultados anteriores para evitar a execução de testes que claramente poderiam ser evitados, por exemplo, em uma situação em uma demanda é submetida à aplicação que está sendo executada em uma máquina virtual com o mais baixo nível de recurso computacional, seria coerente afirmar que essa mesma demanda pode ser atendida por máquinas com perfis computacionais mais robustos. Já no que diz respeito à complexidade, cada trabalho faz uso de uma estratégia particular para a definição da aplicação. Dessa forma, o usuário do ambiente precisa se adaptar à sintaxe de cada solução, e eventualmente, configurar imagens contendo os componentes da aplicação que será avaliada.

%\section{Outros Trabalhos e Considerações}

%******

%\textbf{Caracteristicas gerais do Cloud Capacitor}

%Foco em minimização dos custos de execução dos testes

%Proposta de análise do impacto da escalabilidade horizontal e/ou vertical nos
% resultados dos testes

%Julgamento de qual tipo de máquina apresenta melhor custo x desempenho para
% cadafaixa de workload

%Flexibilidade de implementação da lógica de avaliação através do arcabouço
% deheurísticas

%******

